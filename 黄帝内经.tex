\documentclass{article}% 文档类选项
\usepackage{xeCJK} % 声明中文包

\title{黄帝内经$\cdot$素问}
\date{}
\begin{document}
\CJKfontspec{黑体} %! 修改字体保证生僻字的输出
\maketitle
%%%%%%%%%%%%%%% 目录  %%%%%%%%%%%%%%%%%%%%%%%%%%%%%%%%
\tableofcontents
\clearpage

\section{上古天真论}

%%	1	%%
昔在黄帝,生而神灵,弱而能言,幼而徇齐,长而敦敏,成而登天。

乃问于天师曰:余闻上古之人,春秋皆度百岁,而动作不衰;今时之人,年半百而动作皆衰者,时世异耶?人将失之耶?

岐伯对曰:上古之人,其知道者,法于阴阳,和于术数,食饮有节,起居有常,不妄作劳,故能形与神俱,而尽终其天年,度百岁乃去。

今时之人不然也,以酒为浆,以妄为常,醉以入房,以欲竭其精,以耗散其真,不知持满,不时御神,务快其心,逆于生乐,起居无节,故半百而衰也。

夫上古圣人之教下也,皆谓之虚邪贼风,避之有时,恬淡虚无,真气从之,精神内守,病安从来。

是以志闲而少欲,心安而不惧,形劳而不倦,气从以顺,各从其欲,皆得所愿。

故美其食,任其服,乐其俗,高下不相慕,其民故曰朴。

是以嗜欲不能劳其目,淫邪不能惑其心,愚智贤不肖,不惧于物,故合于道。

所以能年皆度百岁而动作不衰者,以其德全不危也。

帝曰:人年老而无子者,材力尽耶?将天数然也?

岐伯曰:女子七岁,肾气盛,齿更发长。

二七,而天癸至,任脉通,太冲脉盛,月事以时下,故有子。

三七,肾气平均,故真牙生而长极。

四七,筋骨坚,发长极,身体盛壮。

五七,阳明脉衰,面始焦,发始堕。

六七,三阳脉衰于上,面皆焦,发始白。

七七,任脉虚,太冲脉衰少,天癸竭,地道不通,故形坏而无子也。

丈夫八岁,肾气实,发长齿更。

二八,肾气盛,天癸至,精气溢泻,阴阳和,故能有子。

三八,肾气平均,筋骨劲强,故真牙生而长极。

四八,筋骨隆盛,肌肉满壮。

五八,肾气衰,发堕齿槁。

六八,阳气衰竭于上,面焦,发鬓斑白。

七八,肝气衰,筋不能动,天癸竭,精少,肾脏衰,形体皆极。

八八,则齿发去。

肾者主水,受五脏六腑之精而藏之,故五脏盛,乃能泻。

今五脏皆衰,筋骨皆堕,天癸尽矣,故发鬓白,身体重,行步不正,而无子耳。

帝曰:有其年已老,而有子者,何也?

岐伯曰:此其天寿过度,气脉常通,而肾气有余也。此虽有子,男子不过尽八八,女子不过尽七七,而天地之精气皆竭矣。

帝曰:夫道者年皆百数,能有子乎?

岐伯曰:夫道者能却老而全形,身年虽寿,能生子也。

黄帝曰:余闻上古有真人者,提挈天地,把握阴阳,呼吸精气,独立守神,肌肉若一,故能寿敝天地,无有终时,此其道生。

中古之时,有至人者,淳德全道,和于阴阳,调于四时,去世离俗,积精全神,游行天地之间,视听八达之外,此盖益其寿命而强者也,亦归于真人。

其次有圣人者,处天地之和,从八风之理,适嗜欲于世俗之间,无恚嗔之心,行不欲离于世,被服章,举不欲观于俗,外不劳形于事,内无思想之患,以恬愉为务,以自得为功,形体不敝,精神不散,亦可以百数。

其次有贤人者,法则天地,象似日月,辨列星辰,逆从阴阳,分别四时,将从上古合同于道,亦可使益寿而有极时。

\section{四气调神大论}
%%	2	%%
昔在黄帝,生而神灵,弱而能言,幼而徇齐,长而敦敏,成而登天。

乃问于天师曰:余闻上古之人,春秋皆度百岁,而动作不衰;今时之人,年半百而动作皆衰者,时世异耶?人将失之耶?

岐伯对曰:上古之人,其知道者,法于阴阳,和于术数,食饮有节,起居有常,不妄作劳,故能形与神俱,而尽终其天年,度百岁乃去。

今时之人不然也,以酒为浆,以妄为常,醉以入房,以欲竭其精,以耗散其真,不知持满,不时御神,务快其心,逆于生乐,起居无节,故半百而衰也。

夫上古圣人之教下也,皆谓之虚邪贼风,避之有时,恬淡虚无,真气从之,精神内守,病安从来。

是以志闲而少欲,心安而不惧,形劳而不倦,气从以顺,各从其欲,皆得所愿。

故美其食,任其服,乐其俗,高下不相慕,其民故曰朴。

是以嗜欲不能劳其目,淫邪不能惑其心,愚智贤不肖,不惧于物,故合于道。

所以能年皆度百岁而动作不衰者,以其德全不危也。

帝曰:人年老而无子者,材力尽耶?将天数然也?

岐伯曰:女子七岁,肾气盛,齿更发长。

二七,而天癸至,任脉通,太冲脉盛,月事以时下,故有子。

三七,肾气平均,故真牙生而长极。

四七,筋骨坚,发长极,身体盛壮。

五七,阳明脉衰,面始焦,发始堕。

六七,三阳脉衰于上,面皆焦,发始白。

七七,任脉虚,太冲脉衰少,天癸竭,地道不通,故形坏而无子也。

丈夫八岁,肾气实,发长齿更。

二八,肾气盛,天癸至,精气溢泻,阴阳和,故能有子。

三八,肾气平均,筋骨劲强,故真牙生而长极。

四八,筋骨隆盛,肌肉满壮。

五八,肾气衰,发堕齿槁。

六八,阳气衰竭于上,面焦,发鬓斑白。

七八,肝气衰,筋不能动,天癸竭,精少,肾脏衰,形体皆极。

八八,则齿发去。

肾者主水,受五脏六腑之精而藏之,故五脏盛,乃能泻。

今五脏皆衰,筋骨皆堕,天癸尽矣,故发鬓白,身体重,行步不正,而无子耳。

帝曰:有其年已老,而有子者,何也?

岐伯曰:此其天寿过度,气脉常通,而肾气有余也。此虽有子,男子不过尽八八,女子不过尽七七,而天地之精气皆竭矣。

帝曰:夫道者年皆百数,能有子乎?

岐伯曰:夫道者能却老而全形,身年虽寿,能生子也。

黄帝曰:余闻上古有真人者,提挈天地,把握阴阳,呼吸精气,独立守神,肌肉若一,故能寿敝天地,无有终时,此其道生。

中古之时,有至人者,淳德全道,和于阴阳,调于四时,去世离俗,积精全神,游行天地之间,视听八达之外,此盖益其寿命而强者也,亦归于真人。

其次有圣人者,处天地之和,从八风之理,适嗜欲于世俗之间,无恚嗔之心,行不欲离于世,被服章,举不欲观于俗,外不劳形于事,内无思想之患,以恬愉为务,以自得为功,形体不敝,精神不散,亦可以百数。

其次有贤人者,法则天地,象似日月,辨列星辰,逆从阴阳,分别四时,将从上古合同于道,亦可使益寿而有极时。

春三月,此谓发陈。天地俱生,万物以荣,夜卧早起,广步于庭,被发缓形,以使志生,生而勿杀,予而勿夺,赏而勿罚,此春气之应,养生之道也;逆之则伤肝,夏为寒变,奉长者少。

夏三月,此谓蕃秀。天地气交,万物华实,夜卧早起,无厌于日,使志无怒,使华英成秀,使气得泄,若所爱在外,此夏气之应,养长之道也;逆之则伤心,秋为痎疟,奉收者少,冬至重病。

秋三月,此谓容平。天气以急,地气以明,早卧早起,与鸡俱兴,使志安宁,以缓秋刑,收敛神气,使秋气平,无外其志,使肺气清,此秋气之应,养收之道也;逆之则伤肺,冬为飧泄,奉藏者少。

冬三月,此谓闭藏。水冰地坼,无扰乎阳,早卧晚起,必待日光,使志若伏若匿,若有私意,若已有得,去寒就温,无泄皮肤,使气亟夺。此冬气之应,养藏之道也;逆之则伤肾,春为痿厥,奉生者少。

天气,清净光明者也,藏德不止,故不下也。

天明则日月不明,邪害空窍。

阳气者闭塞,地气者冒明,云雾不精,则上应白露不下。

交通不表,万物命故不施,不施则名木多死。

恶气不发,风雨不节,白露不下,则菀槁不荣。

贼风数至,暴雨数起,天地四时不相保,与道相失,则未央绝灭。

唯圣人从之,故身无苛病,万物不失,生气不竭。

逆春气则少阳不生,肝气内变。

逆夏气则太阳不长,心气内洞。

逆秋气则太阴不收,肺气焦满。

逆冬气则少阴不藏,肾气独沉。

夫四时阴阳者,万物之根本也。所以圣人春夏养阳,秋冬养阴,以从其根;故与万物沉浮于生长之门。逆其根则伐其本,坏其真矣。

故阴阳四时者,万物之终始也;死生之本也;逆之则灾害生,从之则苛疾不起,是谓得道。

道者,圣人行之,愚者佩之。从阴阳则生,逆之则死;从之则治,逆之则乱。反顺为逆,是谓内格。

是故圣人不治已病,治未病;不治已乱,治未乱,此之谓也。夫病已成而后药之,乱已成而后治之,譬犹渴而穿井,斗而铸锥,不亦晚乎?
\section{生气通天论}
%%	3	%%
黄帝曰:夫自古通天者,生之本,本于阴阳。

天地之间,六合之内,其气九州、九窍、五脏十二节,皆通乎天气。

其生五,其气三,数犯此者,则邪气伤人,此寿命之本也。

苍天之气,清净则志意治,顺之则阳气固,虽有贼邪,弗能害也,此因时之序。

故圣人抟精神,服天气而通神明。失之则内闭九窍,外壅肌肉,卫气散解,此谓自伤,气之削也。

阳气者,若天与日,失其所,则折寿而不彰。故天运当以日光明。是故阳因而上,卫外者也。

因于寒,欲如运枢,起居如惊,神气乃浮。

因于暑,汗,烦则喘喝,静则多言,体若燔炭,汗出而散。

因于湿,首如裹,湿热不攘,大筋緛短,小筋弛长。緛短为拘,驰长为痿。

因于气,为肿,四维相代,阳气乃竭。

阳气者,烦劳则张,精绝,辟积于夏,使人煎厥;目盲不可以视,耳闭不可以听,溃溃乎若坏都,汩汩乎不可止。

阳气者,大怒则形气绝而血菀于上,使人薄厥。

有伤于筋,纵,其若不容。

汗出偏沮,使人偏枯。

汗出见湿,乃生痤疿。

膏粱之变,足生大丁,受如持虚。

劳汗当风,寒薄为皶,郁乃痤。

阳气者,精则养神,柔则养筋。

开阖不得,寒气从之,乃生大偻。

陷脉为瘘,留连肉腠,俞气化薄,传为善畏,及为惊骇。

营气不从,逆于肉理,乃生痈肿。

魄汗未尽,形弱而气烁,穴俞以闭,发为风疟。

故风者,百病之始也,清静则肉腠闭拒,虽有大风苛毒,弗之能害,此因时之序也。

故病久则传化,上下不并,良医弗为。

故阳畜积病死,而阳气当隔。隔者当泻,不亟正治,粗乃败之。

故阳气者,一日而主外。平旦人气生,日中而阳气隆,日西而阳气已虚,气门乃闭。

是故暮而收拒,无扰筋骨,无见雾露,反此三时,形乃困薄。

岐伯曰:阴者,藏精而起亟也,阳者,卫外而为固也。

阴不胜其阳,则脉流薄疾,并乃狂。阳不胜其阴,则五脏气争,九窍不通。

是以圣人陈阴阳,筋脉和同,骨髓坚固,气血皆从。如是则内外调和,邪不能害,耳目聪明,气立如故。

风客淫气,精乃亡,邪伤肝也。

因而饱食,筋脉横解,肠澼为痔。

因而大饮,则气逆。

因而强力,肾气乃伤,高骨乃坏。

凡阴阳之要,阳密乃固,两者不和,若春无秋,若冬无夏。因而和之,是谓圣度。

故阳强不能密,阴气乃绝。

阴平阳秘,精神乃治;阴阳离决,精气乃绝。

因于露风,乃生寒热。

是以春伤于风,邪气留连,乃为洞泄。

夏伤于暑,秋为痎疟。

秋伤于湿,上逆而咳,发为痿厥。

冬伤于寒,春必温病。

四时之气,更伤五脏。

阴之所生,本在五味;阴之五宫,伤在五味。

是故味过于酸,肝气以津,脾气乃绝。

味过于咸,大骨气劳,短肌,心气抑。

味过于甘,心气喘满,色黑,肾气不衡。

味过于苦,脾气不濡,胃气乃厚。

味过于辛,筋脉沮弛,精神乃央。

是故谨和五味,骨正筋柔,气血以流,腠理以密,如是则骨气以精。谨道如法,长有天命。
\section{金匮真言论}
%%	4	%%
黄帝问曰:天有八风,经有五风,何谓?

岐伯对曰:八风发邪以为经风,触五脏,邪气发病。

所谓得四时之胜者,春胜长夏,长夏胜冬,冬胜夏,夏胜秋,秋胜春,所谓四时之胜也。

东风生于春,病在肝,俞在颈项;南风生于夏,病在心,俞在胸胁;西风生于秋,病在肺,俞在肩背;北风生于冬,病在肾,俞在腰股;中央为土,病在脾,俞在脊。

故春气者,病在头;夏气者,病在脏;秋气者,病在肩背;冬气者,病在四肢。

故春善病鼽衄,仲夏善病胸胁,长夏善病洞泄寒中,秋善病风疟,冬善病痹厥。

故冬不按蹻,春不鼽衄;春不病颈项,仲夏不病胸胁;长夏不病洞泄寒中,秋不病风疟,冬不病痹厥、飧泄,而汗出也。

夫精者,身之本也。故藏于精者,春不病温。夏暑汗不出者,秋成风疟,此平人脉法也。

故曰:阴中有阴,阳中有阳。平旦至日中,天之阳,阳中之阳也;日中至黄昏,天之阳,阳中之阴也;合夜至鸡鸣,天之阴,阴中之阴也;鸡鸣至平旦,天之阴,阴中之阳也。

故人亦应之,夫言人之阴阳,则外为阳,内为阴。言人身之阴阳,则背为阳,腹为阴。言人身之脏腑中阴阳,则脏者为阴,腑者为阳。肝、心、脾、肺、肾,五脏皆为阴,胆、胃、大肠、小肠、膀胱、三焦,六腑皆为阳。

所以欲知阴中之阴,阳中之阳者,何也?为冬病在阴,夏病在阳,春病在阴,秋病在阳,皆视其所在,为施针石也。

故背为阳,阳中之阳,心也;背为阳,阳中之阴,肺也;腹为阴,阴中之阴,肾也;腹为阴,阴中之阳,肝也;腹为阴,阴中之至阴,脾也。

此皆阴阳表里,内外雌雄,相输应也。故以应天之阴阳也。

帝曰:五脏应四时,各有收受乎?

岐伯曰:有。

东方青色,入通于肝,开窍于目,藏精于肝。其病发惊骇,其味酸,其类草木,其畜鸡,其谷麦,其应四时,上为岁星,是以春气在头也。其音角,其数八,是以知病之在筋也,其臭臊。

南方赤色,入通于心,开窍于耳,藏精于心,故病在五脏。其味苦,其类火,其畜羊,其谷黍,其应四时,上为荧惑星。是以知病之在脉也。其音徵,其数七,其臭焦。

中央黄色,入通于脾,开窍于口,藏精于脾,故病在舌本。其味甘,其类土,其畜牛,其谷稷,其应四时,上为镇星。是以知病之在肉也。其音宫,其数五,其臭香。

西方白色,入通于肺,开窍于鼻,藏精于肺,故病在背。其味辛,其类金,其畜马,其谷稻,其应四时,上为太白星。是以知病之在皮毛也。其音商,其数九,其臭腥。

北方黑色,入通于肾,开窍于二阴,藏精于肾,故病在谿。其味咸,其类水,其畜彘,其谷豆,其应四时,上为辰星。是以知病之在骨也。其音羽,其数六,其臭腐。

故善为脉者,谨察五脏六腑,一逆一从,阴阳表里,雌雄之纪,藏之心意,合心于精,非其人勿教,非其真勿授,是谓得道。
\section{阴阳应象大论}
%%	5	%%
黄帝曰:阴阳者,天地之道也,万物之纲纪,变化之父母,生杀之本始,神明之府也。治病必求于本。故积阳为天,积阴为地。阴静阳躁;阳生阴长;阳杀阴藏;阳化气,阴成形。寒极生热,热极生寒;寒气生浊,热气生清;清气在下,则生飧泄;浊气在上,则生䐜胀。此阴阳反作,病之逆从也。

故清阳为天,浊阴为地。地气上为云,天气下为雨;雨出地气,云出天气。故清阳出上窍,浊阴出下窍;清阳发腠理,浊阴走五脏;清阳实四肢,浊阴归六腑。

水为阴,火为阳,阳为气,阴为味。味归形,形归气,气归精,精归化。精食气,形食味,化生精,气生形。味伤形,气伤精,精化为气,气伤于味。

阴味出下窍;阳气出上窍。味厚者为阴,薄为阴之阳;气厚者为阳,薄为阳之阴。味厚则泄,薄则通。气薄则发泄,厚则发热。壮火之气衰,少火之气壮;壮火食气,气食少火;壮火散气,少火生气。气味,辛甘发散为阳,酸苦涌泄为阴。

阴胜则阳病,阳胜则阴病。阳胜则热,阴胜则寒。重寒则热,重热则寒。寒伤形,热伤气。气伤痛,形伤肿。故先痛而后肿者,气伤形也,先肿而后痛者,形伤气也。风胜则动,热胜则肿,燥胜则干,寒胜则浮,湿胜则濡泻。

天有四时五行,以生长收藏,以生寒暑燥湿风。人有五脏化五气,以生喜怒悲忧恐。故喜怒伤气,寒暑伤形。暴怒伤阴,暴喜伤阳。厥气上行,满脉去形。喜怒不节,寒暑过度,生乃不固。故重阴必阳,重阳必阴。故曰:“冬伤于寒,春必温病;春伤于风,夏生飧泄;夏伤于暑,秋必痎疟;秋伤于湿,冬生咳嗽。”

帝曰:余闻上古圣人,论理人形,列别脏腑;端络经脉,会通六合,各从其经;气穴所发,各有处名;溪谷属骨,皆有所起;分部逆从,各有条理;四时阴阳,尽有经纪;外内之应,皆有表里,其信然乎?

岐伯对曰:东方生风,风生木,木生酸,酸生肝,肝生筋,筋生心,肝主目。其在天为玄,在人为道,在地为化。化生五味,道生智,玄生神。神在天为风,在地为木,在体为筋,在脏为肝,在色为苍,在音为角,在声为呼,在变动为握,在窍为目,在味为酸,在志为怒。怒伤肝,悲胜怒;风伤筋,燥胜风;酸伤筋,辛胜酸。

南方生热,热生火,火生苦,苦生心,心生血,血生脾,心主舌。其在天为热,在地为火,在体为脉,在脏为心,在色为赤,在音为徵,在声为笑,在变动为忧(嚘的通假字),在窍为舌,在味为苦,在志为喜。喜伤心,恐胜喜;热伤气,寒胜热;苦伤气,咸胜苦。

中央生湿,湿生土,土生甘,甘生脾,脾生肉,肉生肺,脾主口。其在天为湿,在地为土,在体为肉,在脏为脾,在色为黄,在音为宫,在声为歌,在变动为哕,在窍为口,在味为甘,在志为思。思伤脾,怒胜思;湿伤肉,风胜湿;甘伤肉,酸胜甘。

西方生燥,燥生金,金生辛,辛生肺,肺生皮毛,皮毛生肾,肺主鼻。其在天为燥,在地为金,在体为皮毛,在脏为肺,在色为白,在音为商,在声为哭,在变动为咳,在窍为鼻,在味为辛,在志为忧。忧伤肺,喜胜忧;热伤皮毛,寒胜热;辛伤皮毛,苦胜辛。

北方生寒,寒生水,水生咸,咸生肾,肾生骨髓,髓生肝,肾主耳。其在天为寒,在地为水,在体为骨,在脏为肾,在色为黑,在音为羽,在声为呻,在变动为栗,在窍为耳,在味为咸,在志为恐。恐伤肾,思胜恐;寒伤血(《太素》为骨),燥(《太素》为湿)胜寒;咸伤血,甘胜咸。

故曰:“天地者,万物之上下也;阴阳者,血气之男女也;左右者,阴阳之道路也;水火者,阴阳之征兆也;阴阳者,万物之能(胎的通假字)始也。”故曰:“阴在内,阳之守也,阳在外,阴之使也。”

帝曰:法阴阳奈何?

岐伯曰:阳盛则身热,腠理闭,喘粗为之俛仰,汗不出而热,齿干以烦冤,腹满,死,能(耐的通假字)冬不能(耐的通假字)夏。阴胜则身寒,汗出,身长清,数栗而寒,寒则厥,厥则腹满,死,能(耐的通假字)夏不能(耐的通假字)冬。此阴阳更胜之变,病之形能(态的通假字)也。

帝曰:调此二者,奈何?

岐伯曰:能知七损八益,则二者可调,不知用此,则早衰之节也。年四十,而阴气自半也,起居衰矣;年五十,体重,耳目不聪明矣;年六十,阴痿,气大衰,九窍不利,下虚上实,涕泣俱出矣。故曰:“知之则强,不知则老。”故同出而名异耳。智者察同,愚者察异,愚者不足,智者有余,有余而耳目聪明,身体轻强,老者复壮,壮者益治。是以圣人为无为之事,乐恬憺之能(态的通假字),从欲快志于虚无之守,故寿命无穷,与天地终,此圣人之治身也。

天不足西北,故西北方阴也,而人右耳目不如左明也。地不满东南,故东南方阳也,而人左手足不如右强也。

帝曰:何以然?

岐伯曰:东方阳也,阳者其精并于上,并于上则上明而下虚,故使耳目聪明而手足不便。西方阴也,阴者其精并于下,并于下则下盛而上虚,故其耳目不聪明而手足便也。故俱感于邪,其在上则右甚,在下则左甚,此天地阴阳所不能全也,故邪居之。

故天有精,地有形,天有八纪,地有五理,故能为万物之父母。清阳上天,浊阴归地,是故天地之动静,神明为之纲纪,故能以生长收藏,终而复始。惟贤人上配天以养头,下象地以养足,中傍人事以养五脏。天气通于肺,地气通于嗌,风气通于肝,雷气通于心,谷气通于脾,雨气通于肾。六经为川,肠胃为海,九窍为水注之气。以天地为之阴阳,阳之汗,以天地之雨名之;阳之气,以天地之疾风名之。暴气象雷,逆气象阳。故治不法天之纪,不用地之理,则灾害至矣。

故邪风之至,疾如风雨,故善治者治皮毛,其次治肌肤,其次治筋脉,其次治六腑,其次治五脏。治五脏者,半死半生也。故天之邪气,感则害人五脏;水谷之寒热,感则害于六腑;地之湿气,感则害皮肉筋脉。

故善用针者,从阴引阳,从阳引阴,以右治左,以左治右。以我知彼,以表知里,以观过与不及之理,见微得过,用之不殆。

善诊者,察色按脉,先别阴阳,审清浊而知部分;视喘息,听音声,而知所苦;观权衡规矩,而知病所主;按尺寸,观浮沉滑涩,而知病所生。以治无过,以诊则不失矣。

故曰:病之始起也,可刺而已,其盛,可待衰而已。故因其轻而扬之,因其重而减之,因其衰而彰之。

形不足者,温之以气;精不足者,补之以味。

其高者,因而越之;其下者,引而竭之;中满者,泻之于内;其有邪者,渍形以为汗;其在皮者,汗而发之;其慓悍者,按而收之;其实者,散而泻之。审其阴阳,以别柔刚。阳病治阴,阴病治阳,定其血气,各守其乡,血实宜决之,气虚宜掣引之。
\section{阴阳离合论}
%%	6	%%
黄帝问曰:余闻天为阳,地为阴,日为阳,月为阴。大小月三百六十日成一岁,人亦应之。今三阴三阳不应阴阳,其故何也?岐伯对曰:阴阳者,数之可十,推之可百,数之可千,推之可万,万之大不可胜数,然其要一也。天覆地载,万物方生。未出地者,命曰阴处,名曰阴中之阴;则出地者,命曰阴中之阳。阳予之正,阴为之主。故生因春,长因夏,收因秋,藏因冬。失常则天地四塞。阴阳之变,其在人者,亦数之可数。

帝曰:愿闻三阴三阳之离合也。岐伯曰:圣人南面而立,前曰广明,后曰太冲。太冲之地,名曰少阴;少阴之上,名曰太阳。太阳根起于至阴,结于命门,名曰阴中之阳。中身而上名曰广明,广明之下名曰太阴,太阴之前,名曰阳明。阳明根起于厉兑,名曰阴中之阳。厥阴之表,名曰少阳。少阳根起于窍阴,名曰阴中之少阳。是故三阳之离合也:太阳为开,阳明为阖,少阳为枢。三经者,不得相失也,搏而勿浮,名曰一阳。

帝曰:愿闻三阴?岐伯曰:外者为阳,内者为阴。然则中为阴,其冲在下,名曰太阴,太阴根起于隐白,名曰阴中之阴。太阴之后,名曰少阴,少阴根起于涌泉,名曰阴中之少阴。少阴之前,名曰厥阴,厥阴根起于大敦,阴之绝阳,名曰阴之绝阴。是故三阴之离合也,太阴为开,厥阴为阖,少阴为枢。三经者不得相失也,搏而勿沉,名曰一阴。阴阳(雩重)(雩重),积传为一周,气里形表,而为相成也。
\section{阴阳别论}
%%	7	%%
黄帝问曰:人有四经,十二从,何谓?岐伯对曰:四经应四时;十二从应十二月;十二月应十二脉。

脉有阴阳,知阳者知阴,知阴者知阳。

凡阳有五,五五二十五阳。

所谓阴者,真脏也。见则为败,败必死也。

所谓阳者,胃脘之阳也。

别于阳者,知病处也,别于阴者,知生死之期。

三阳在头,三阴在手,所谓一也。

别于阳者,知病忌时,别于阴者,知死生之期。

谨熟阴阳,无与众谋。所谓阴阳者,去者为阴,至者为阳;静者为阴,动者为阳;迟者为阴,数者为阳。

凡持真脉之藏脉者,肝至悬绝急,十八日死;心至悬绝,九日死;肺至悬绝,十二日死;肾至悬绝,七日死;脾至悬绝,四日死。

曰:二阳之病发心脾,有不得隐曲,女子不月;其传为风消,其传为息贲者,死不治。

曰:三阳为病发寒热,下为痈肿,及为痿厥,腨㾓;其传为索泽,其传为㿗疝。

曰:一阳发病,少气,善咳,善泄;其传为心掣,其传为隔。

二阳一阴发病,主惊骇、背痛、善噫、善欠,名曰风厥。

二阴一阳发病,善胀、心满善气。

三阴三阳发病,为偏枯萎易,四肢不举。

鼓一阳曰钩,鼓一阴曰毛,鼓阳胜急曰弦,鼓阳至而绝曰石,阴阳相过曰溜。

阴争于内,阳扰于外,魄汗未藏,四逆而起,起则熏肺,使人喘鸣。

阴之所生,和本曰和。

是故刚与刚,阳气破散,阴气乃消亡。

淖则刚柔不和,经气乃绝。

死阴之属,不过三日而死,生阳之属,不过四日而死。

所谓生阳死阴者,肝之心谓之生阳,心之肺谓之死阴,肺之肾谓之重阴,肾之脾谓之辟阴,死不治。

结阳者,肿四支。

结阴者,便血一升,再结二升,三结三升。

阴阳结斜,多阴少阳曰石水,少腹肿。

二阳结,谓之消。

三阳结,谓之隔。

三阴结,谓之水。

一阴一阳结,谓之喉痹。

阴搏阳别,谓之有子。

阴阳虚,肠澼死。

阳加于阴,谓之汗。

阴虚阳搏,谓之崩。

三阴俱搏,二十日夜半死;二阴俱搏,十三日夕时死;一阴俱搏,十日平旦死;三阳俱搏且鼓,三日死;三阴三阳俱搏,心腹满,发尽不得隐曲,五日死;二阳俱搏,其病温,死不治,不过十日死。
\section{灵兰秘典论}
%%	8	%%
黄帝问曰:愿闻十二脏之相使,贵贱何如?

岐伯曰:悉乎哉问也,请遂言之。心者,君主之官也,神明出焉。肺者,相傅之官,治节出焉。肝者,将军之官,谋虑出焉。胆者,中正之官,决断出焉。膻中者,臣使之官,喜乐出焉。脾胃者,仓廪之官,五味出焉。大肠者,传道(导)之官,变化出焉。小肠者,受盛之官,化物出焉。肾者,作强之官,伎巧出焉。三焦者,决渎之官,水道出焉。膀胱者,州都之官,津液藏焉,气化则能出矣。凡此十二官者,不得相失也。故主明则下安,以此养生则寿,殁世不殆,以为天下则大昌。主不明则十二官危,使道闭塞而不通,形乃大伤,以此养生则殃,以为天下者,其宗大危,戒之戒之!

至道在微,变化无穷,孰知其原!窘乎哉!消者瞿瞿,孰知其要!闵闵之当,孰者为良!恍惚之数,生于毫厘,毫厘之数,起于度量,千之万之,可以益大,推之大之,其形乃制。

黄帝曰:善哉!余闻精光之道,大圣之业,而宣明大道,非斋戒择吉日不敢受也。

黄帝乃择吉日良兆,而藏灵兰之室以传保焉。
\section{六节藏象论}
%%	9	%%
黄帝问曰:余闻天以六六之节,以成一岁,人(应为地,见注解)以九九制会,计人亦有三百六十五节,以为天地,久矣。不知其所谓也?

岐伯对曰:昭乎哉问也!请遂言之。夫六六之节,九九制会者,所以正天之度,气之数也。天度者,所以制日月之行也;气数者,所以纪化生之用也。天为阳,地为阴,日为阳,月为阴,行有分纪,周有道理,日行一度,月行十三度而有奇焉。故大小月三百六十五日而成岁,积气余而盈闰矣。立端于始,表正于中,推余于终,而天度毕矣。

帝曰:余已闻天度矣。愿闻气数,何以合之?

岐伯曰:天以六六为节,地以九九制会。天有十日,日六竟而周甲,甲六覆而终岁,三百六十日法也。夫自古通天者,生之本,本于阴阳,其气九州九窍,皆通乎天气。故其生五,其气三。三而成天,三而成地,三而成人。三而三之,合则为九,九分为九野,九野为九脏。故形脏四,神脏五,合为九脏以应之也。

帝曰:余已闻六六九九之会也,夫子言积气盈闰,愿闻何谓气?请夫子发蒙解惑焉。

岐伯曰:此上帝所秘,先师传之也。

帝曰:请遂闻之。

岐伯曰:五日谓之候,三候谓之气,六气谓之时,四时谓之岁,而各从其主治焉。五运相袭,而皆治之,终期之日,周而复始,时立气布,如环无端,候亦同法。故曰:不知年之所加,气之盛衰,虚实之所起,不可以为工矣。

帝曰:五运之始,如环无端,其太过不及如何?

岐伯曰:五气更立,各有所胜,盛虚之变,此其常也。

帝曰:平气何如?

岐伯曰,无过者也。

帝曰:太过不及奈何?

岐伯曰:在经有也。

帝曰:何谓所胜?

岐伯曰:春胜长夏,长夏胜冬,冬胜夏,夏胜秋,秋胜春,所谓得五行时之胜,各以气命其脏。

帝曰:何以知其胜?

岐伯曰:求其至也,皆归始春,未至而至,此谓太过,则薄所不胜,而乘所胜也,命曰气淫。不分邪僻内生,工不能禁。至而不至,此谓不及,则所胜妄行,而所生受病,所不胜薄之也,命曰气迫。所谓求其至者,气至之时也。谨候其时,气可与期。失时反候,五治不分,邪僻内生,工不能禁也。

帝曰:有不袭乎?

岐伯曰:苍天之气,不得无常也。气之不袭是谓非常,非常则变矣。

帝曰:非常而变奈何?

岐伯曰:变至则病,所胜则微,所不胜则甚,因而重感于邪则死矣。故非其时则微,当其时则甚也。

帝曰:善。余闻气合而有形,因变以正名。天地之运,阴阳之化,其于万物,孰少孰多,可得闻乎?

岐伯曰:悉乎哉问也!天至广不可度,地至大不可量,大神灵问,请陈其方。草生五色,五色之变,不可胜视。草生五味,五味之美,不可胜极。嗜欲不同,各有所通。天食人以五气,地食人以五味。五气入鼻,藏于心肺,上使五色修明,音声能彰。五味入口,藏于肠胃,味有所藏,以养五气,气和而生,津液相成,神乃自生。

帝曰:藏象何如?

岐伯曰:心者,生之本,神之变也,其华在面,其充在血脉,为阳中之太阳,通于夏气。肺者,气之本,魄之处也,其华在毛,其充在皮,为阳中之太(《太素》作少)阴,通于秋气。肾者,主蛰,封藏之本,精之处也,其华在发,其充在骨,为阴中之少(《太素》作太)阴,通于冬气。肝者,罢极之本,魂之居也,其华在爪,其充在筋,以生血气,其味酸,其色苍,此为阳中之少阳,通于春气。脾、胃、大肠、小肠、三焦、膀胱者,仓廪之本,营之居也,名曰器,能化糟粕,转味而入出者也,其华在唇四白,其充在肌,其味甘,其色黄,此至阴之类,通于土气。凡十一脏,取决于胆也。

故人迎一盛,病在少阳;二盛,病在太阳;三盛,病在阳明;四盛,已上为格阳。寸口一盛,病在厥阴;二盛,病在少阴;三盛,病在太阴;四盛,已上为关阴。人迎与寸口俱盛四倍以上为关格,关格之脉赢,不能极于天地之精气,则死矣。
\section{五藏生成}
%%	10	%%
心之合脉也,其荣色也,其主肾也。肺之合皮也,其荣毛也,其主心也。肝之合筋也,其荣爪也,其主肺也。脾之合肉也,其荣唇也,其主肝也。肾之合骨也,其荣发也,其主脾也。是故多食咸,则脉凝泣而变色;多食苦,则皮槁而毛拔;多食辛,则筋急而爪枯;多食酸,则肉胝(月刍)而唇揭;多食甘,则骨痛而发落,此五味之所伤也。故心欲苦,肺欲辛,肝欲酸,脾欲甘,肾欲咸,此五味之所合也。

五脏之气,故色见青如草兹者死,黄如枳实者死,黑如(火台)者死,赤如衃血者死,白如枯骨者死,此五色之见死也。青如翠羽者生,赤如鸡冠者生,黄如蟹腹者生,白如豕膏者生,黑如乌羽者生,此五色之见生也。生于心,如以缟裹朱。生于肺,如以缟裹红。生于肝,如以缟裹绀。生于脾,如以缟裹栝楼实。生于肾,如以缟裹紫。此五脏所生之外荣也。

色味当五脏,白当肺、辛,赤当心、苦,青当肝、酸,黄当脾、甘,黑当肾、咸。故白当皮,赤当脉,青当筋,黄当肉,黑当骨。

诸脉者,皆属于目;诸髓者,皆属于脑;诸筋者,皆属于节;诸血者,皆属于心;诸气者,皆属于肺,此四肢八溪之朝夕也。故人卧血归于肝,肝受血而能视,足受血而能步,掌受血而能握,指受血而能摄。卧出而风吹之,血凝于肤者为痹,凝于脉者为泣,凝于足者为厥。此三者,血行而不得反其空,故为痹厥也。人有大谷十二分,小溪三百五十四名,少十二俞,此皆卫气所留止,邪气之所客也,针石缘而去之。

诊病之始,五决为纪。欲知其始,先建其母。所谓五决者,五脉也。

是以头痛巅疾,下虚上实,过在足少阴巨阳,甚则入肾。徇蒙招尤,目冥耳聋,下实上虚,过在足少阳厥阴,甚则入肝。腹满(月真)胀,支膈胠胁、下厥上冒,过在足太阴阳明。咳嗽上气,厥在胸中,过在手阳明太阴。心烦头痛,病在膈中,过在手巨阳少阴。

夫脉之小大,滑涩浮沉,可以指别。五脏之象,可以类推。五脏相音,可以意识。五色微诊,可以目察。能合脉色,可以万全。

赤脉之至也,喘而坚。诊曰:有积气在中,时害于食名曰心痹。得之外疾,思虑而心虚,故邪从之。白脉之至也,喘而浮。上虚下实,惊,有积气在胸中,喘而虚。名曰肺痹。寒热,得之醉而使内也。青脉之至也。长而左右弹。有积气在心下,肢胠。名曰肝痹。得之寒湿,与疝同法。腰痛足清头痛。黄脉之至也,大而虚。有积气在腹中,有厥气,名曰厥疝。女子同法,得之疾使四肢,汗出当风。黑脉之至也,上坚而大。有积气在小腹与阴,名曰肾痹。得之沐浴,清水而卧。

凡相五色之奇脉,面黄目青,面黄目赤,面黄目白,面黄目黑者,皆不死也。面青目赤,面赤目白,面青目黑,面黑目白,面赤目青,皆死也。
\section{五藏别论}
%%	11	%%
黄帝问曰:余闻方士,或以脑髓为脏,或以肠胃为脏,或以为腑。敢问更相反,皆自谓是,不知其道,愿闻其说。

岐伯对曰:脑、髓、骨、脉、胆、女子胞,此六者,地气之所生也,皆藏于阴而象于地,故藏而不泻,名曰奇恒之腑。夫胃、大肠、小肠、三焦、膀胱,此五者,天气之所生也,其气象天,故泻而不藏,此受五脏浊气,名曰传化之腑,此不能久留输泻者也。魄门亦为五脏使,水谷不得久藏。所谓五脏者,藏精气而不泻也,故满而不能实;六腑者,传化物而不藏,故实而不能满也。所以然者,水谷入口,则胃实而肠虚;食下则肠实而胃虚。故曰实而不满,满而不实也。

帝曰:气口何以独为五脏主?

岐伯曰:胃者水谷之海,六腑之大源也。五味入口,藏于胃,以养五脏气。气口亦太阴也,是以五脏六腑之气味,皆出于胃,变见于气口。故五气入鼻,藏于心肺,心肺有病,而鼻为之不利也。凡治病必察其上(原文脱字,据《太素》补)下,适其脉,观其志意,与其病也。

拘于鬼神者,不可与言至德,恶于针石者,不可与言至巧。病不许治者,病必不治,治之无功矣。


\section{异法方宜论}
%%	12	%%
黄帝曰:医之治病也,一病而治各不同,皆愈,何也?

岐伯对曰:地势使然也。

故东方之域,天地之所始生也。鱼盐之地,海滨傍水,其民食鱼而嗜咸,皆安其处,美其食。鱼者使人热中,盐者胜血,故其民皆黑色疏理,其病皆为痈疡,其治宜砭石。故砭石者,亦从东方来。

西方者,金玉之域,沙石之处,天地之所收引也。其民陵居而多风,水土刚强,其民不衣而褐荐,其民华食而脂肥,故邪不能伤其形体,其病生于内,其治宜毒药。故毒药者,亦从西方来。

北方者,天地所闭藏之域也。其地高陵居,风寒冰洌。其民乐野处而乳食,脏寒生满病,其治宜灸焫。故灸焫者,亦从北方来。

南方者,天地所长养,阳之所盛处也。其地下,水土弱,雾露之所聚也。其民嗜酸而食胕,故其民皆致理而赤色,其病挛痹,其治宜微针。故九针者,亦从南方来。

中央者,其地平以湿,天地所以生万物也众。其民食杂而不劳,故其病多痿厥寒热,其治宜导引按蹻。故导引按蹻者,亦从中央出也。

故圣人杂合以治,各得其所宜。故治所以异而病皆愈者,得病之情,知治之大体也。
\section{移精变气论}
%%	13	%%
黄帝问曰:余闻古之治病,惟其移精变气,可祝由而已。今世治病,毒药治其内,针石治其外,或愈或不愈,何也?

岐伯对曰:往古人居禽兽之间,动作以避寒,阴居以避暑,内无眷慕之累,外无伸宦之形,此恬惔之世,邪不能深入也。故毒药不能治其内,针石不能治其外,故可移精祝由而已。当今之世不然,忧患缘其内,苦形伤其外,又失四时之从,逆寒暑之宜,贼风数至,虚邪朝夕,内至五脏骨髓,外伤空窍肌肤,所以小病必甚,大病必死,故祝由不能已也。

帝曰:善。余欲临病人,观死生,决嫌疑,欲知其要,如日月光,可得闻乎?

岐伯曰:色脉者,上帝之所贵也,先师之所传也。上古使僦贷季,理色脉而通神明,合之金木水火土,四时八风六合,不离其常,变化相移,以观其妙,以知其要。欲知其要,则色脉是矣。色以应日,脉以应月,常求其要,则其要也。夫色之变化,以应四时之脉,此上帝之所贵,以合于神明也。所以远死而近生,生道以长,命曰圣王。中古之治病,至而治之,汤液十日,以去八风五痹之病,十日不已,治以草苏草荄之枝,本末为助,标本已得,邪气乃服。暮世之治病也则不然,治不本四时,不知日月,不审逆从,病形已成,乃欲微针治其外,汤液治其内,粗工凶凶,以为可攻,故病未已,新病复起。

帝曰:愿闻要道。

岐伯曰:治之要极,无失色脉,用之不惑,治之大则。逆从到行,标本不得,亡神失国。去故就新,乃得真人。

帝曰:余闻其要于夫子矣,夫子言不离色脉,此余之所知也。

岐伯曰:治之极于一。

帝曰:何谓一?

岐伯曰:一者因得之。

帝曰:奈何?

岐伯曰:闭户塞牗,系之病者,数问其情,以从其意,得神者昌,失神者亡。

帝曰:善。
\section{汤液醪醴论}
%%	14	%%
黄帝问曰:为五谷汤液及醪醴奈何?

岐伯对曰:必以稻米,炊之稻薪,稻米者完,稻薪者坚。

帝曰:何以然?

岐伯曰:此得天地之和,高下之宜,故能至完:伐取得时,故能至坚也。

帝曰:上古圣人作汤液醪醴,为而不用何也?

岐伯曰:自古圣人之作汤液醪醴者,以为备耳!夫上古作汤液,故为而弗服也。中古之世,道德稍衰,邪气时至,服之万全。

帝曰:今之世不必已何也。

岐伯曰:当今之世,必齐毒药攻其中,镵石针艾治其外也。

帝曰:形弊血尽而功不立者何?

岐伯曰:神不使也。帝曰:何谓神不使?

岐伯曰:针石道也。精神不进,志意不治,故病不可愈。今精坏神去,营卫不可复收。何者?嗜欲无穷,而忧患不止,精气弛坏,营泣卫除,故神去之而病不愈也。

帝曰:夫病之始生也,极微极精,必先入结于皮肤。今良工皆称曰病成,名曰逆,则针石不能治,良药不能及也。今良工皆得其法,守其数,亲戚兄弟远近音声日闻于耳,五色日见于目,而病不愈者,亦何暇不早乎?

岐伯曰:病为本,工为标,标本不得,邪气不服,此之谓也。

帝曰:其有不从毫毛而生,五脏阳以竭也,津液充郭,其魄独居,孤精于内,气耗于外,形不可与衣相保,此四极急而动中,是气拒于内而形施于外,治之奈何?

岐伯曰:平治于权衡,去宛陈莝,微动四极,温衣缪剌其处,以复其形。开鬼门,洁净府,精以时服;五阳已布,疏涤五脏,故精自生,形自盛,骨肉相保,巨气乃平。

帝曰:善。
\section{玉版论要}
%%	15	%%
黄帝问曰:余闻《揆度》、《奇恒》,所指不同,用之奈何?

岐伯对曰:《揆度》者,度病之浅深也。《奇恒》者,言奇病也。请言道之至数,《五色》、《脉变》、《揆度》、《奇恒》,道在于一。神转不回,回则不转,乃失其机。至数之要,迫近以微,著之玉版,命曰合《玉机》。

容色见上下左右,各在其要。其色见浅者,汤液主治,十日已。其见深者,必齐(剂的通假字)主治,二十一日已。其见大深者,醪酒主治,百日已;色夭面脱,不治,百日尽已。脉短气绝死,病温虚甚死。

色见上下左右,各在其要。上为逆,下为从。女子右为逆,左为从。男子左为逆,右为从。易,重阳死,重阴死。阴阳反作,治在权衡相夺,《奇恒》事也,《揆度》事也。

搏脉痹躄,寒热之交,脉孤为消气,虚泄为夺血。孤为逆,虚为从。

行《奇恒》之法,以太阴始。行所不胜曰逆,逆则死;行所胜曰从,从则活。八风四时之胜,终而复始,逆行一过,不复可数。论要毕矣。
\section{诊要经终论}
%%	16	%%
黄帝问曰:诊要何如。

岐伯对曰:正月二月,天气始方(放的通假字),地气始发,人气在肝。三月四月,天气正方(放的通假字),地气定发,人气在脾(按《太素》为心)。五月六月,天气盛,地气高,人气在头(按《太素》为脾)。七月八月,阴气始杀,人气在肺(按《太素》为胃)。九月十月,阴气始冰,地气始闭,人气在心(按《太素》为肺)。十一月十二月,冰复,地气合,人气在肾。

故春刺散俞,及与分理,血出而止,甚者传气,间者环也。夏刺络俞,见血而止,尽气闭环,痛病必下。秋刺皮肤,循理,上下同法,神变而止。冬刺俞窍于分理,甚者直下,间者散下。春夏秋冬,各有所刺,法其所在。

春刺夏分,脉乱气微,入淫骨髓,病不能愈,令人不嗜食,又且少气。春刺秋分,筋挛逆气,环为咳嗽,病不愈,令人时惊,又且哭。春刺冬分,邪气着脏,令人胀,病不愈,又且欲言语。

夏刺春分,病不愈,令人解堕。夏刺秋分,病不愈,令人心中欲无言,惕惕如人将捕之。夏刺冬分,病不愈,令人少气,时欲怒。

秋刺春分,病不已,令人惕然,欲有所为,起而忘之。秋刺夏分,病不已,令人益嗜卧,又且善梦。秋刺冬分,病不已,令人洒洒时寒。

冬刺春分,病不已,令人欲卧不能眠,眠而有见。冬刺夏分,病不愈,气上,发为诸痹。冬刺秋分,病不已,令人善渴。

凡刺胸腹者,必避五脏。中心者环死;(刺中肝,五日死;)中脾者,五日死;中肾者七日死;中肺者五日死;中鬲者皆为伤中,其病虽愈,不过一岁必死。刺避五脏者,知逆从也。所谓从者,鬲与脾肾之处,不知者反之。刺胸腹者,必以布憿着之,乃从单布上刺。刺之不愈复刺。刺针必肃,刺肿摇针,经刺勿摇,此刺之道也。

帝曰:愿闻十二经脉之终奈何?

岐伯曰:太阳之脉,其终也,戴眼、反折、瘛瘲,其色白,绝汗乃出,出则死矣。少阳终者,耳聋,百节皆纵,目睘绝系。绝系一日半死,其死也,色先青,白乃死矣。阳明终者,口目动作,善惊妄言,色黄,其上下经盛,不仁则终矣。少阴终者,面黑齿长而垢,腹胀闭,上下不通而终矣。太阴终者,腹胀闭,不得息,善噫善呕,呕则逆,逆则面赤,不逆则上下不通,不通则面黑,皮毛焦而终矣。厥阴终者,中热嗌干,善溺心烦,甚则舌卷,卵上缩而终矣。此十二经之所败也。
\section{脉要精微论}
%%	17	%%
黄帝问曰:诊法何如?

岐伯对曰:诊法常以平旦,阴气未动,阳气未散,饮食未进,经脉未盛,络脉调匀,气血未乱,故乃可诊有过之脉。

切脉动静而视精明,察五色,观五脏有余不足,六腑强弱,形之盛衰,以此参伍,决死生之分。

夫脉者,血之府也。长则气治,短则气病,数则烦心,大则病进。上盛则气高,下盛则气胀,代则气衰,细则气少,涩则心痛,浑浑革(亟的通假字)至如涌泉,病进而色弊,绵绵其去如弦绝,死。

夫精明五色者,气之华也。赤欲如白裹朱,不欲如赭;白欲如鹅羽,不欲如盐,青欲如苍璧之泽,不欲如蓝;黄欲如罗裹雄黄,不欲如黄土;黑欲如重漆色,不欲如地苍。五色精微象见矣,其寿不久也。夫精明者,所以视万物,别白黑,审短长,以长为短,以白为黑,如是则精衰矣。

五脏者,中之守也。中盛藏满,气胜伤恐者,声如从室中言,是中气之湿也。言而微,终日乃复言者,此夺气也。衣被不敛,言语善恶,不避亲疏者,此神明之乱也。仓廪不藏者,是门户不要也。水泉不止者,是膀胱不藏也。得守者生,失守者死。

夫五脏者,身之强也。头者,精明之府,头倾视深,精神将夺矣。背者,胸中之府,背曲肩随,府将坏矣。腰者,肾之府,转摇不能,肾将惫矣。膝者,筋之府,屈伸不能,行则偻附,筋将惫矣。骨者,髓之府,不能久立,行则振掉,骨将惫矣。得强则生,失强则死。

岐伯曰:反四时者,有余为精,不足为消。应太过,不足为精;应不足,有余为消。阴阳不相应,病名曰关格。

帝曰:脉其四时动奈何?知病之所在奈何?知病之所变奈何?知病乍(作的古体字)在内奈何?知病乍(作的古体字)在外奈何?请问此五者可得闻乎?

岐伯曰:请言其与天运转大也。万物之外,六合之内,天地之变,阴阳之应,彼春之暖,为夏之暑,彼秋之忿,为冬之怒。四变之动,脉与之上下,以春应中规,夏应中矩,秋应中衡,冬应中权。是故冬至四十五日,阳气微上,阴气微下;夏至四十五日,阴气微上,阳气微下,阴阳有时,与脉为期,期而相失,知脉所分,分之有期,故知死时。微妙在脉,不可不察,察之有纪,从阴阳始,始之有经,从五行生,生之有度,四时为宜。补泻勿失,与天地如一,得一之情,以知死生。是故声合五音,色合五行,脉合阴阳。

是知阴盛则梦涉大水恐惧,阳盛则梦大火燔灼,阴阳惧盛,则梦相杀毁伤。上盛则梦飞,下盛则梦堕;甚饱则梦予,甚饥则梦取;肝气盛则梦怒,肺气盛则梦哭;短虫多则梦聚众,长虫多则梦相击毁伤。

是故,持脉有道,虚静为保。春日浮,如鱼之游在波;夏日在肤,泛泛乎万物有余;秋日下肤,蛰虫将去;冬日在骨,蛰虫周密,君子居室。故曰:“知内者按而纪之,知外者终而始之。”此六者,持脉之大法。

心脉搏坚而长,当病舌卷不能言;其软而散者,当消环(《甲乙》、《脉经》、《千金》及《太素》诸版皆为渴)自己。肺脉搏坚而长,当病唾血;其软而散者,当病灌汗,至令不复散发也。肝脉搏坚而长,色不青,当病坠若搏,因血在胁下,令人喘逆;其软而散色泽者,当病溢饮。溢饮者,渴暴多饮,而易入肌皮肠胃之外也。胃脉搏坚而长,其色赤,当病折髀;其软而散者,当病食痹。脾脉搏坚而长,其色黄,当病少气;其软而散色不泽者,当病足胻肿,若水状也。肾脉搏坚而长,其色黄而赤者,当病折腰;其软而散者,当病少血,至令不复也。

帝曰:诊得心脉而急,此为何病?病形何如?

岐伯曰:病名心疝,少腹当有形也。

帝曰:何以言之?

岐伯曰:心为牡脏,小肠为之使,故曰少腹当有形也。

帝曰:诊得胃脉病形何如?

岐伯曰:胃脉实则胀,虚则泄。

帝曰:病成而变何谓?

岐伯曰:风成为寒热,瘅成为消中,厥成为巅疾,久风为飧泄,脉风成为疠。病之变化,不可胜数。

帝曰:诸痈肿筋挛骨痛,此皆安生?

岐伯曰:此寒气之肿,八风之变也。

帝曰:治之奈何?

岐伯曰:此四时之病,以其胜治之愈也。

帝曰:有故病五脏发动,因伤脉色,各何以知其久暴至之病乎?

岐伯曰:悉乎哉问也!征其脉小,色不夺者,新病也;征其脉不夺,其色夺者,此久病也。征其脉与五色俱夺者,此久病也;征其脉与五色俱不夺者,新病也。肝与肾脉并至,其色苍赤,当病毁伤不见血,已见血,湿若中水也。

尺内两旁,则季胁也,尺外以候肾,尺里以候腹。中附上,左外以候肝,内以候鬲。右外以候胃,内以候脾。上附上,右外以候肺,内以候胸中,左外以候心,内以候膻中。前以候前,后以候后。上竟上者,胸喉中事也。下竟下者,少腹腰股膝胫足中事也。

粗大者阴不足,阳有余,为热中也。来疾去徐,上实下虚,为厥巅疾。来徐去疾,上虚下实,为恶风也。故中恶风者,阳气受也。有脉俱沉细数者,少阴厥也;沉细数散者,寒热也;浮而散者为眴仆。诸浮不躁者,皆在阳,则为热;其有躁者在手,诸细而沉者,皆在阴,则为骨痛;其有静者在足。数动一代者,病在阳之脉也,泄及便脓血。诸过者切之,涩者阳气有余也,滑者阴气有余也。阳气有余为身热无汗,阴气有余为多汗身寒,阴阳有余则无汗而寒。

推而外之,内而不外,有心腹积也。推而内之,外而不内,身有热也。推而上之,上而不下,腰足清也。推而下之,下而不上,头项痛也。按之至骨,脉气少者,腰脊痛而身有痹也。
\section{平人气象论}
%%	18	%%
黄帝问曰:平人何如?

岐伯对曰:人一呼脉再动,一吸脉亦再动,呼吸定息,脉五动,闰以太息,命曰平人。平人者,不病也。常以不病调病人,医不病,故为病人平息以调之为法。

人一呼脉一动,一吸脉一动,曰少气。人一呼脉三动,一吸脉三动而躁,尺热曰病温,尺不热脉滑曰病风,脉涩曰痹。人一呼脉四动以上曰死,脉绝不至曰死,乍疏乍数曰死。

平人之常气禀于胃,胃者平人之常气也,人无胃气曰逆,逆者死。

春胃微弦曰平,弦多胃少曰肝病,但弦无胃曰死,胃而有毛曰秋病,毛甚曰今病。脏真散于肝,肝藏筋膜之气也。

夏胃微钩曰平,钩多胃少曰心病,但钩无胃曰死,胃而有石曰冬病,石甚曰今病。脏真通于心,心藏血脉之气也。

长夏胃微软弱曰平,弱多胃少曰脾病,但代无胃曰死,软弱有石曰冬病,弱甚曰今病。脏真濡于脾,脾藏肌肉之气也。

秋胃微毛曰平,毛多胃少曰肺病,但毛无胃曰死。毛而有弦曰春病,弦甚曰今病。脏真高于肺,以行荣卫阴阳也。

冬胃微石曰平,石多胃少曰肾病,但石无胃曰死,石而有钩曰夏病,钩甚曰今病。脏真下于肾,肾藏骨髓之气也。

胃之大络,名曰虚里,贯鬲络肺,出于左乳下,其动应衣,脉宗气也。盛喘数绝者,则病在中;结而横,有积矣;绝不至曰死。乳之下其动应衣,宗气泄也。

欲知寸口太过与不及,寸口之脉中手短者,曰头痛;寸口脉中手长者,曰足胫痛;寸口脉中手促上击者,曰肩背痛;寸口脉沉而坚者,曰病在中;寸口脉浮而盛者,曰病在外;寸口脉沉而弱,曰寒热及疝瘕少腹痛;寸口脉沉而横,曰胁下有积,腹中有横积痛;寸口脉沉而喘,曰寒热。

脉盛滑坚者,曰病在外。脉小实而坚者,病在内。脉小弱以涩,谓之久病。脉滑浮而疾者,谓之新病。脉急者,曰疝瘕少腹痛。脉滑曰风。脉涩曰痹。缓而滑曰热中。盛而紧曰胀。

脉从阴阳,病易已;脉逆阴阳,病难已。脉得四时之顺,曰病无他;脉反四时及不间脏,曰难已。

臂多青脉,曰脱血。尺脉缓涩,谓之解㑊;安卧脉盛,谓之脱血;尺涩脉滑,谓之多汗;尺寒脉细,谓之后泄;脉尺粗常热者,谓之热中。

肝见庚辛死,心见壬癸死,脾见甲乙死,肺见丙丁死,肾见戊己死。是谓真脏见,皆死。

颈脉动喘疾咳,曰水。目裹微肿如卧蚕起之状,曰水。溺黄赤安卧者,黄疸。已食如饥者,胃疸。面肿曰风。足胫肿曰水。目黄者曰黄疸。妇人手少阴脉动甚者,姙子也。

脉有逆从四时,未有脏形,春夏而脉瘦,秋冬而脉浮大,命曰逆四时也。风热而脉静,泄而脱血脉实,病在中脉虚,病在外脉涩坚者,皆难治,命曰反四时也。

人以水谷为本,故人绝水谷则死,脉无胃气亦死。所谓无胃气者,但得真脏脉不得胃气也。所谓脉不得胃气者,肝不弦,肾不石也。

太阳脉至,洪大以长;少阳脉至,乍数乍疏,乍短乍长;阳明脉至,浮大而短。

夫平心脉来,累累如连珠,如循琅玕,曰心平。夏以胃气为本。病心脉来,喘喘连属,其中微曲,曰心病。死心脉来,前曲后居,如操带钩,曰心死。

平肺脉来,厌厌聂聂,如落榆荚,曰肺平。秋以胃气为本。病肺脉来,不上不下,如循鸡羽,曰肺病。死肺脉来,如物之浮,如风吹毛,曰肺死。

平肝脉来,软弱招招,如揭长竿末梢,曰肝平。春以胃气为本。病肝脉来,盈实而滑,如循长竿,曰肝病。死肝脉来,急益劲,如新张弓弦,曰肝死。

平脾脉来,和柔相离,如鸡践地,曰脾平。长夏以胃气为本。病脾脉来,实而盈数,如鸡举足,曰脾病。死脾脉来,锐坚如乌之喙,如鸟之距,如屋之漏,如水之流,曰脾死。

平肾脉来,喘喘累累,如钩,按之而坚,曰肾平。冬以胃气为本。病肾脉来,如引葛,按之益坚,曰肾病。死肾脉来,发如夺索,辟辟如弹石,曰肾死。
\section{玉机真藏论}
%%	19	%%
黄帝问曰:春脉如弦,何如而弦?

岐伯对曰:春脉者,肝也,东方木也,万物之所以始生也,故其气来,软弱轻虚而滑,端直以长,故曰弦,反此者病。

帝曰:何如而反?

岐伯曰:其气来实而强,此谓太过,病在外。其气来不实而微,此谓不及,病在中。

帝曰:春脉太过与不及,其病皆何如?

岐伯曰:太过则令人善忘,忽忽眩冒而巅疾;其不及则令人胸痛引背,下则两胁胠满。

帝曰:善。夏脉如钩,何如而钩?

岐伯曰:夏脉者,心也,南方火也,万物之所以盛长也,故其气来盛去衰,故曰钩,反此者病。

帝曰:何如而反?

岐伯曰:其气来盛去亦盛,此谓太过,病在外。其气来不盛,去反盛,此谓不及,病在中。

帝曰:夏脉太过与不及,其病皆何如?

岐伯曰:太过则令人身热而肤痛,为浸淫;其不及则令人烦心,上见咳唾,下为气泄。

帝曰:善。秋脉如浮,何如而浮?

岐伯曰:秋脉者,肺也,西方金也,万物之所以收成也,故其气来轻虚以浮,来急去散,故曰浮,反此者病。

帝曰:何如而反?

岐伯曰:其气来,毛而中央坚,两旁虚,此谓太过,病在外。其气来,毛而微,此谓不及,病在中。

帝曰:秋脉太过与不及,其病皆何如?

岐伯曰:太过则令人逆气而背痛。愠愠然,其不及则令人喘,呼吸少气而咳,上气见血,下闻病音。

帝曰:善。冬脉如营,何如而营?

岐伯曰:冬脉者,肾也,北方水也,万物之所以合藏也,故其气来沉以搏,故曰营,反此者病。

帝曰:何如而反?

岐伯曰:其气来如弹石者,此谓太过,病在外。其去如数者,此谓不及,病在中。

帝曰:冬脉太过与不及,其病皆何如?

岐伯曰:太过则令人解㑊,脊脉痛,而少气不欲言;其不及则令人心悬,如病饥,䏚中清,脊中痛,少腹满,小便变。

帝曰:善。

帝曰:四时之序,逆从之变异也,然脾脉独,何主?

岐伯曰:脾脉者土也,孤脏,以灌四旁者也。

帝曰:然则脾善恶可得见之乎?

岐伯曰:善者不可得见,恶者可见。

帝曰:恶者何如可见?

岐伯曰:其来如水之流者,此谓太过,病在外。如乌之喙者,此谓不及,病在中。

帝曰:夫子言脾为孤藏,中央土,以灌四旁,其太过与不及,其病皆何如?

岐伯曰:太过则令人四肢不举,其不及则令人九窍不通,名曰重强。

帝瞿然而起,再拜而稽首曰:善。吾得脉之大要,天下至数,《五色》、《脉变》、《揆度》、《奇恒》,道在于一,神转不回,回则不转,乃失其机,至数之要,迫近以微,著之玉版,藏之藏府,每旦读之,名曰《玉机》。

五脏受气于其所生,传之于其所胜,气舍于其所生,死于其所不胜。病之且死,必先传行,至其所不胜,病乃死,此言气之逆行也,故死。肝受气于心,传之于脾,气舍于肾,至肺而死。心受气于脾,传之于肺,气舍于肝,至肾而死。脾受气于肺,传之于肾,气舍于心,至肝而死。肺受气于肾,传之于肝,气舍于脾,至心而死。肾受气于肝,传之于心,气舍于肺,至脾而死。此皆逆死也,一日一夜,五分之,此所以占死生之早暮也。

黄帝曰:五脏相通,移皆有次。五脏有病,则各传其所胜。不治,法三月,若六月,若三日,若六日,传五脏而当死,是顺传所胜之次。故曰:“别于阳者,知病从来;别于阴者,知死生之期。”言知至其所困而死。

是故风者,百病之长也。今风寒客于人,使人毫毛毕直,皮肤闭而为热。当是之时,可汗而发也。或痹不仁肿痛,当是之时,可汤熨及火灸刺而去之。弗治,病入舍于肺,名曰肺痹,发咳上气。弗治,肺即传而行之肝,病名曰肝痹,一名曰厥,胁痛出食,当是之时,可按若刺耳。弗治,肝传之脾,病名曰脾风,发瘅,腹中热,烦心,出黄,当此之时,可按、可药、可浴。弗治,脾传之肾,病名曰疝瘕,少腹冤热而痛,出白,一名曰蛊,当此之时,可按、可药。弗治,肾传之心,病筋脉相引而急,病名曰瘈,当此之时,可灸、可药。弗治,满十日,法当死。肾因传之心,心即复反传而行之肺,发寒热,法当三岁死,此病之次也。

然其卒发者,不必治于传,或其传化有不以次,不以次入者,忧恐悲喜怒,令不得以其次,故令人有大病矣。因而喜大虚则肾气乘矣,怒则肝气乘矣,悲则肺气乘矣,恐则脾气乘矣,忧则心气乘矣,此其道也。故病有五,五五二十五变,及其传化。传,乘之名也。

大骨枯槁,大肉陷下,胸中气满,喘息不便,其气动形,期六月死,真脏脉见,乃予之期日。大骨枯槁,大肉陷下,胸中气满,喘息不便,内痛引肩项,期一月死,真脏见,乃予之期日。大骨枯槁,大肉陷下,胸中气满,喘息不便,内痛引肩项,身热脱肉破䐃,真脏见,十日之内死。大骨枯槁,大肉陷下,肩髓内消,动作益衰,真脏来见,期一岁死,见其真脏,乃予之期日。大骨枯槁,大肉陷下,胸中气满,腹内痛,心中不便,肩项身热,破䐃脱肉,目眶陷,真脏见,目不见人,立死;其见人者,至其所不胜之时则死。

急虚身中卒至,五脏绝闭,脉道不通,气不往来,譬于堕溺,不可为期。其脉绝不来,若人一息五六至,其形肉不脱,真脏虽不见,犹死也。

真肝脉至,中外急,如循刀刃,责责然如按琴瑟弦,色青白不泽,毛折乃死。真心脉至,坚而搏,如循薏苡子,累累然,色赤黑不泽,毛折乃死。真肺脉至,大而虚,如以毛羽中人肤,色白赤不泽,毛折乃死。真肾脉至,搏而绝,如指弹石,辟辟然,色黑黄不泽,毛折乃死。真脾脉至,弱而乍数乍疏,色黄青不泽,毛折乃死。诸真脏脉见者,皆死不治也。

黄帝曰:见真脏曰死,何也?

岐伯曰:五脏者,皆禀气于胃,胃者五脏之本也。藏气者,不能自致于手太阴,必因于胃气,乃至于手太阴也。故五脏各以其时,自为而至于手太阴也。故邪气胜者,精气衰也。故病甚者,胃气不能与之俱至于手太阴,故真脏之气独见。独见者,病胜脏也,故曰死。

帝曰:善。

黄帝曰:凡治病察其形气色泽,脉之盛衰,病之新故,乃治之,无后其时。形气相得,谓之可治;色泽以浮,谓之易已;脉从四时,谓之可治;脉弱以滑,是有胃气,命曰易治,取之以时。形气相失,谓之难治;色夭不泽,谓之难已;脉实以坚,谓之益甚;脉逆四时,为不可治。必察四难,而明告之。

所谓逆四时者,春得肺脉,夏得肾脉,秋得心脉,冬得脾脉,其至皆悬绝沉涩者,命曰逆四时。未有藏形,于春夏而脉沉涩,秋冬而脉浮大,名曰逆四时也。

病热脉静;泄而脉大;脱血而脉实;病在中,脉实坚;病在外,脉不实坚者,皆难治。

黄帝曰:余闻虚实以决死生,愿闻其情。

岐伯曰:五实死,五虚死。

帝曰:愿闻五实五虚。

岐伯曰:脉盛,皮热,腹胀,前后不通,闷瞀,此谓五实。脉细,皮寒,气少,泄利前后,饮食不入,此谓五虚。

帝曰:其时有生者何也?

岐伯曰:浆粥入胃,泄注止,则虚者活;身汗得后利,则实者活。此其候也。
\section{三部九候论}
%%	20	%%
黄帝问曰:余闻九针于夫子,众多博大,不可胜数。余愿闻要道,以属子孙,传之后世,着之骨髓,藏之肝肺,歃血而受,不敢妄泄,令合天道,必有终始。上应天光星辰历纪,下副四时五行,贵贱更互,冬阴夏阳,以人应之奈何?愿闻其方!

岐伯对曰:妙乎哉问也!此天地之至数。

帝曰:愿闻天地之至数,合于人形血气,通决死生,为之奈何?

岐伯曰:天地之至数,始于一,终于九焉。一者天,二者地,三者人。因而三之,三三者九,以应九野。故人有三部,部有三候,以决死生,以处百病,以调虚实,而除邪疾。

帝曰:何谓三部?

岐伯曰:有下部,有中部,有上部,部各有三候。三候者,有天、有地、有人也。必指而导之,乃以为真。上部天,两额之动脉;上部地,两颊之动脉;上部人,耳前之动脉。中部天,手太阴也;中部地,手阳明也;中部人,手少阴也。下部天,足厥阴也;下部地,足少阴也;下部人,足太阴也。故下部之天以候肝,地以候肾,人以候脾胃之气。

帝曰:中部之候奈何?

岐伯曰:亦有天,亦有地,亦有人。天以候肺,地以候胸中之气,人以候心。

帝曰:上部以何候之?

岐伯曰:亦有天,亦有地,亦有人。天以候头角之气,地以候口齿之气,人以候耳目之气。三部者,各有天,各有地,各有人。三而成天,三而成地,三而成人。三而三之,合则为九,九分为九野,九野为九脏。故神脏五,形脏四,合为九脏。五脏已败,其色必夭,夭必死矣。

帝曰:以候奈何?

岐伯曰:必先度其形之肥瘦,以调其气之虚实,实则泻之,虚则补之。必先去其血脉而后调之,无问其病,以平为期。

帝曰:决死生奈何?

岐伯曰:形盛脉细,少气不足以息者危。形瘦脉大,胸中多气者死。形气相得者生。参伍不调者病。三部九候皆相失者死。上下左右之脉相应如参舂者病甚。上下左右相失不可数者死。中部之候虽独调,与众脏相失者死。中部之候相减者死。目内陷者死。

帝曰:何以知病之所在?

岐伯曰:察九候,独小者病,独大者病,独疾者病,独迟者病,独热者病,独寒者病,独陷下者病。以左手足上,上去踝五寸按之,庶右手足当踝而弹之,其应过五寸以上,蠕蠕然者不病;其应疾,中手浑浑然者病;中手徐徐然者病;其应上不能至五寸,弹之不应者死。是以脱肉身不去者死。中部乍疏乍数者死。其脉代而钩者,病在络脉。九候之相应也,上下若一,不得相失。一候后则病,二候后则病甚,三候后则病危。所谓后者,应不俱也。察其腑脏,以知死生之期,必先知经脉,然后知病脉,真脏脉见者,胜死。足太阳气绝者,其足不可屈伸,死必戴眼。

帝曰:冬阴夏阳奈何?

岐伯曰:九候之脉,皆沉细悬绝者为阴,主冬,故以夜半死;盛躁喘数者为阳,主夏,故以日中死。是故寒热病者,以平旦死。热中及热病者,以日中死。病风者,以日夕死。病水者,以夜半死。其脉乍疏乍数乍迟乍疾者,日乘四季死。形肉已脱,九候虽调犹死。七诊虽见,九候皆从者不死。所言不死者,风气之病,及经月之病,似七诊之病而非也,故言不死。若有七诊之病,其脉候亦败者死矣,必发哕噫。必审问其所始病,与今之所方病,而后各切循其脉,视其经络浮沉,以上下逆从循之。其脉疾者不病,其脉迟者病,脉不往来者死,皮肤着者死。

帝曰:其可治者奈何?

岐伯曰:经病者治其经。孙络病者治其孙络血。血病身有痛者,治其经络。其病者在奇邪,奇邪之脉则缪刺之。留瘦不移,节而刺之。上实下虚,切而从之,索其结络脉,刺出其血,以见通之。瞳子高者,太阳不足;戴眼者,太阳已绝,此决死生之要,不可不察也。
\section{经脉别论}
%%	21	%%
黄帝问曰:人之居处、动静、勇怯,脉亦为之变乎?

岐伯对曰:凡人之惊恐恚劳动静,皆为变也。是以夜行则喘出于肾,淫气病肺。有所堕恐,喘出于肝,淫气害脾。有所惊恐,喘出于肺,淫气伤心。度水跌仆,喘出于肾与骨,当是之时,勇者气行则已;怯者则着而为病也。故曰:诊病之道,观人勇怯、骨肉、皮肤,能知其情,以为诊法也。

故饮食饱甚,汗出于胃。惊而夺精,汗出于心。持重远行,汗出于肾。疾走恐惧,汗出于肝。摇体劳苦,汗出于脾。故春秋冬夏,四时阴阳,生病起于过用,此为常也。

食气入胃,散精于肝,淫气于筋。食气入胃,浊气归心,淫精于脉。脉气流经,经气归于肺,肺朝百脉,输精于皮毛。毛脉合精,行气于腑,腑精神明,留于四脏,气归于权衡。权衡以平,气口成寸,以决死生。

饮入于胃,遊溢精气,上输于脾,脾气散精,上归于肺,通调水道,下输膀胱,水精四布,五经并行,合于四时五脏阴阳,揆度以为常也。

太阳脏独至,厥喘虚气逆,是阴不足阳有余也。表里当俱泻,取之下俞。

阳明脏独至,是阳气重并也。当泻阳补阴,取之下俞。

少阳脏独至,是厥气也。蹻前卒大,取之下俞。

少阳独至者,一阳之过也。

太阴脏搏者,用心省真,五脉气少,胃气不平,三阴也。宜治其下俞,补阳泻阴。

一阴独啸,少阴厥也,阳并于上,四脉争张,气归于肾。宜治其经络,泻阳补阴。

二阴至,厥阴之治也。真虚㾓心,厥气留薄,发为白汗,调食和药,治在下俞。

帝曰:太阳脏何象?

岐伯曰:象三阳而浮也。

帝曰:少阳脏何象?

岐伯曰:象一阳也。一阳脏者,滑而不实也。

帝曰:阳明脏何象?

岐伯曰:象大浮也。太阴脏搏,言伏鼓也。二阴搏至,肾沉不浮也。
\section{藏气法时论}
%%	22	%%
黄帝问曰:合人形以法四时五行而治,何如而从?何如而逆?得失之意,愿闻其事!

岐伯对曰:五行者,金木水火土也。更贵更贱,以知死生,以决成败,而定五脏之气,间甚之时,死生之期也。

帝曰:愿卒闻之。

岐伯曰:肝主春,足厥阴、少阳主治。其日甲乙。肝苦急,急食甘以缓之。心主夏,手少阴、太阳主治。其日丙丁。心苦缓,急食酸以收之。脾主长夏,足太阴、阳明主治。其日戊己。脾苦湿,急食苦以燥之。肺主秋,手太阴、阳明主治。其日庚辛。肺苦气上逆,急食苦以泄之。肾主冬,足少阴、太阳主治。其日壬癸。肾苦燥,急食辛以润之,开腠理,致津液,通气也。

病在肝,愈于夏,夏不愈,甚于秋,秋不死,持于冬,起于春,禁当风。肝病者,愈在丙丁,丙丁不愈,加于庚辛,庚辛不死,持于壬癸,起于甲乙。肝病者,平旦慧,下晡甚,夜半静。肝欲散,急食辛以散之,用辛补之,酸泻之。

病在心,愈在长夏,长夏不愈,甚于冬,冬不死,持于春,起于夏,禁温食、热衣。心病者,愈在戊己,戊己不愈,加于壬癸,壬癸不死,持于甲乙,起于丙丁。心病者,日中慧,夜半甚,平旦静。心欲软,急食咸以软之,用咸补之,甘泻之。

病在脾,愈在秋,秋不愈,甚于春,春不死,持于夏,起于长夏,禁温食饱食、湿地濡衣。脾病者,愈在庚辛,庚辛不愈,加于甲乙,甲乙不死,持于丙丁,起于戊己。脾病者,日昳慧,日出甚,下晡静。脾欲缓,急食甘以缓之,用苦泻之,甘补之。

病在肺,愈在冬,冬不愈,甚于夏,夏不死,持于长夏,起于秋,禁寒饮食、寒衣。肺病者,愈在壬癸,壬癸不愈,加于丙丁,丙丁不死,持于戊己,起于庚辛。肺病者,下晡慧,日中甚,夜半静。肺欲收,急食酸以收之,用酸补之,辛泻之。

病在肾,愈在春,春不愈,甚于长夏,长夏不死,持于秋,起于冬,禁犯焠㶼热食、温炙衣。肾病者,愈在甲乙,甲乙不愈,甚于戊己,戊己不死,持于庚辛,起于壬癸。肾病者,夜半慧,四季甚,下晡静。肾欲坚,急食苦以坚之,用苦补之,咸泻之。

夫邪气之客于身也,以胜相加,至其所生而愈,至其所不胜而甚,至于所生而持,自得其位而起。必先定五脏之脉,乃可言间甚之时,死生之期也。

肝病者,两胁下痛引少腹,令人善怒。虚则目䀮䀮无所见,耳无所闻,善恐,如人将捕之。取其经,厥阴与少阳。气逆则头痛,耳聋不聪,颊肿,取血者。

心病者,胸中痛,胁支满,胁下痛,膺背肩甲间痛,两臂内痛。虚则胸腹大,胁下与腰相引而痛。取其经,少阴、太阳,舌下血者。其变病,刺郄中血者。

脾病者,身重,善饥(别本作肌)肉痿,足不收行,善瘛,脚下痛。虚则腹满,肠鸣飧泄,食不化。取其经,太阴、阳明、少阴血者。

肺病者,喘咳逆气,肩背痛,汗出,尻阴股膝髀腨胻足皆痛。虚则少气,不能报息,耳聋嗌干。取其经,太阴、足太阳之外,厥阴内血者。

肾病者,腹大,胫肿,喘咳,身重,寝汗出憎风。虚则胸中痛,大腹小腹痛,清厥意不乐。取其经,少阴、太阳血者。

肝色青,宜食甘,粳米、牛肉、枣、葵皆甘。心色赤,宜食酸,小豆、犬肉、李、韭皆酸。肺色白,宜食苦,麦、羊肉、杏、薤皆苦。脾色黄,宜食咸,大豆、豕肉、栗、藿皆咸。肾色黑,宜食辛,黄黍、鸡肉、桃、葱皆辛。辛散,酸收,甘缓,苦坚,咸软。

毒药攻邪,五谷为养,五果为助,五畜为益,五菜为充。气味合而服之,以补精益气。此五者有辛酸甘苦咸,各有所利,或散或收,或缓或急,或坚或软。四时五脏,病随五味所宜也。
\section{宣明五气}
%%	23	%%
五味所入:酸入肝,辛入肺,苦入心,咸入肾,甘入脾。是谓五入。

五气所病:心为噫。肺为咳。肝为语。脾为吞。肾为欠、为嚏。胃为气逆、为哕、为恐。大肠、小肠为泄。下焦溢为水。膀胱不利为癃,不约为遗溺。胆为怒。是谓五病。

五精所并:精气并于心则喜,并于肺则悲,并于肝则忧,并于脾则畏,并于肾则恐,是谓五并。虚而相并者也。

五脏所恶:心恶热,肺恶寒,肝恶风,脾恶湿,肾恶燥。是谓五恶。

五脏化液:心为汗,肺为涕,肝为泪,脾为涎,肾为唾。是谓五液。

五味所禁:辛走气,气病无多食辛。咸走血,血病无多食咸。苦走骨,骨病无多食苦。甘走肉,肉病无多食甘。酸走筋,筋病无多食酸。是谓五禁,无令多食。

五病所发:阴病发于骨,阳病发于血,阴病发于肉,阳病发于冬,阴病发于夏。是谓五发。

五邪所乱:邪入于阳则狂,邪入于阴则痹,搏阳则为巅疾,搏阴则为瘖,阳入之阴则静,阴出之阳则怒。是谓五乱。

五邪所见:春得秋脉,夏得冬脉,长夏得春脉,秋得夏脉,冬得长夏脉,名曰阴出之阳,病善怒,不治。是谓五邪,皆同命,死不治。

五脏所藏:心藏神,肺藏魄,肝藏魂,脾藏意,肾藏志。是谓五脏所藏。

五脏所主:心主脉,肺主皮,肝主筋,脾主肉,肾主骨。是谓五主。

五劳所伤:久视伤血,久卧伤气,久坐伤肉,久立伤骨,久行伤筋。是谓五劳所伤。

五脉应象:肝脉弦,心脉钩,脾脉代,肺脉毛,肾脉石。是谓五脏之脉。
\section{血气形志}
%%	24	%%
夫人之常数,太阳常多血少气,少阳常少血多气,阳明常多气多血,少阴常少血多气,厥阴常多血少气,太阴常多气少血,此天之常数。

足太阳与少阴为表里,少阳与厥阴为表里,阳明与太阴为表里,是为足阴阳也。手太阳与少阴为表里,少阳与心主为表里,阳明与太阴为表里,是为手之阴阳也。

今知手足阴阳所苦,凡治病必先去其血,乃去其所苦,伺之所欲,然后泻有余,补不足。

欲知背俞,先度其两乳间,中折之,更以他草度去半已,即以两隅相拄也。乃举以度其背,令其一隅居上,齐脊大椎,两隅在下,当其下隅者,肺之俞也。复下一度,心之俞也。复下一度,左角肝之俞也,右角脾之俞也。复下一度,肾之俞也。是谓五脏之俞,灸刺之度也。

形乐志苦,病生于脉,治之以灸刺。形乐志乐,病生于肉,治之以针石。形苦志乐,病生于筋,治之以熨引。形苦志苦,病生于咽嗌,治之以百药。形数惊恐,经络不通,病生于不仁,治之以按摩醪药。是谓五形志也。

刺阳明,出血气。刺太阳,出血恶气。刺少阳,出气恶血。刺太阴,出气恶血。刺少阴,出气恶血。刺厥阴,出血恶气也。
\section{宝命全形论}
%%	25	%%
黄帝问曰:天覆地载,万物悉备,莫贵于人。人以天地之气生,四时之法成。君王众庶,尽欲全形,形之疾病,莫知其情,留淫日深,着于骨髓,心私虑之。余欲针除其疾病,为之奈何?

岐伯对曰:夫盐之味咸者,其气令器津泄;弦绝者,其音嘶败;木敷(按《太素》为陈)者,其叶发(按《太素》为落)。病深者,其声哕。人有此三者,是谓坏腑,毒药无治,短针无取,此皆绝皮伤肉,血气争黑。

帝曰:余念其痛,心之乱惑,反甚其病,不可更代,百姓闻之,以为残贼,为之奈何?

岐伯曰:夫人生于地,悬命于天,天地合气,命之曰人。人能应四时者,天地为之父母;知万物者,谓之天子。天有阴阳,人有十二节;天有寒暑,人有虚实。能经天地阴阳之化者,不失四时;知十二节之理者,圣智不能欺也。能存八动之变,五胜更立;能达虚实之数者,独出独入,呿吟至微,秋毫在目。

帝曰:人生有形,不离阴阳。天地合气,别为九野,分为四时,月有小大,日有短长,万物并至,不可胜量,虚实呿吟,敢问其方?

岐伯曰:木得金而伐,火得水而灭,土得木而达,金得火而缺,水得土而绝。万物尽然,不可胜竭。故针(箴的通假字)有悬布天下者五,黔首共余食,莫知之也。一曰治神,二曰知养身,三曰知毒药为真,四曰制砭石小大,五曰知腑脏血气之诊。五法俱立,各有所先。今末世之刺也,虚者实之,满者泄之,此皆众工所共知也。若夫法天则地,随应而动,和之者若响,随之者若影,道无鬼神,独来独往。

帝曰:愿闻其道。

岐伯曰:凡刺之真,必先治神,五脏已定,九候已备,后乃存针,众脉不见,众凶弗闻,外内相得,无以形先,可玩往来,乃施于人。人有虚实,五虚勿近,五实勿远,至其当发,间不容瞚。手动若务,针耀而匀,静意视义,观适之变,是谓冥冥,莫知其形。见其乌乌,见其稷稷,从见其飞,不知其谁。伏如横弩,起如发机。

帝曰:何如而虚?何如而实?

岐伯曰:刺虚者须其实,刺实者须其虚。经气已至,慎守勿失。深浅在志,远近若一。如临深渊,手如握虎,神无营于众物。
\section{八正神明论}
%%	26	%%
黄帝问曰:用针之服,必有法则焉,今何法何则?

岐伯对曰:法天则地,合以天光。

帝曰:愿卒闻之。

岐伯曰:凡刺之法,必候日月星辰,四时八正之气,气定乃刺之。

是故天温日明,则人血淖液而卫气浮,故血易泻,气易行;天寒日阴,则人血凝泣(涩的通假字)而卫气沉。月始生则血气始精,卫气始行;月郭满则血气实,肌肉坚;月郭空则肌肉减,经络虚,卫气去,形独居。是以因天时而调血气也。是以天寒无刺,天温无疑,月生无泻,月满无补,月郭空无治。是谓得时而调之。因天之序,盛虚之时,移光定位,正立而待之。

故曰:月生而泻,是谓藏(全元起本作“减”)虚。月满而补,血气扬溢,络有留血,命曰重实。月郭空而治,是谓乱经。阴阳相错,真邪不别,沉以留止,外虚内乱,淫邪乃起。

帝曰:星辰八正何候?

岐伯曰:星辰者,所以制日月之行也。八正者,所以候八风之虚邪,以时至者也。四时者,所以分春秋冬夏之气所在,以时调之也。八正之虚邪而避之勿犯也。以身之虚而逢天之虚,两虚相感,其气至骨。入则伤五脏,工候救之,弗能伤也。故曰:“天忌不可不知也。”

帝曰:善。其法星辰者,余闻之矣,愿闻法往古者。

岐伯曰:法往古者,先知《针经》也。验于来今者,先知日之寒温,月之虚盛,以候气之浮沉,而调之于身,观其立有验也。观其冥冥者,言形气荣卫之不形于外,而工独知之。以日之寒温,月之虚盛,四时气之浮沉,参伍相合而调之,工常先见之,然而不形于外,故曰观于冥冥焉!通于无穷者,可以传于后世也。是故工之所以异也。然而不形见于外,故俱不能见也。视之无形,尝之无味,故谓冥冥,若神髣髴。

虚邪者,八正之虚邪气也;正邪者,身形若用力汗出,腠理开,逢虚风,其中人也微,故莫知其情,莫见其形。上工救其萌芽,必先见三部九候之气,尽调不败而救之,故曰上工。下工救其已成,救其已败。救其已成者,言不知三部九候之相失,因病而败之也。知其所在者,知诊三部九候之病脉处而治之,故曰守其门户焉。莫知其情,而见邪形也。

帝曰:余闻补泻,未得其意。

岐伯曰:泻必用方。方者,以气方盛也,以月方满也,以日方温也,以身方定也。以息方吸而内(纳的通假字)针,乃复候其方吸而转针,乃复候其方呼而徐引针,故曰泻必用方,其气而行焉。补必用员。员者,行也,行者,移也。刺必中其荣,复以吸排针也。故员与方,非针也。故养神者,必知形之肥瘦,荣卫血气之盛衰。血气者,人之神,不可不谨养。

帝曰:妙乎哉论也!合人形于阴阳四时,虚实之应,冥冥之期,其非夫子,孰能通之。然夫子数言形与神,何谓形?何谓神?愿卒闻之。

岐伯曰:请言形。形乎形,目冥冥,问其所病,索之于经,慧然在前。按之不得,不知其情,故曰形。

帝曰:何谓神?

岐伯曰:请言神。神乎神,耳不闻,目明心开而志先,慧然独悟,口弗能言,俱视独见,适若昏,昭然独明,若风吹云,故曰神。三部九候为之原,九针之论不必存也。
\section{离合真邪}
%%	27	%%
黄帝问曰:余闻九针九篇,夫子乃因而九之,九九八十一篇,余尽通其意矣。经言气之盛衰,左右倾移,以上调下,以左调右,有余不足,补泻于荣(为荥字之误)、输,余知之矣。此皆营卫之倾移,虚实之所生,非邪气从外入于经也。余愿闻邪气之在经也,其病人何如?取之奈何?

岐伯对曰:夫圣人之起度数,必应于天地。故天有宿度,地有经水,人有经脉。天地温和,则经水安静;天寒地冻,则经水凝泣(涩的通假字,下同);天暑地热,则经水沸溢;卒风暴起,则经水波涌而陇起。

夫邪之入于脉也,寒则血凝泣,暑则气淖泽,虚邪因而入客,亦如经水之得风也,经之动脉,其至也,亦时陇起,其行于脉中,循循然。其至寸口中手也,时大时小,大则邪至,小则平。其行无常处,在阴与阳,不可为度,从而察之,三部九候。卒然逢之,早遏其路。吸则内(纳的通假字,下同)针,无令气忤,静以久留,无令邪布。吸则转针,以得气为故。候呼引针,呼尽乃去,大气皆出,故命曰泻。

帝曰:不足者补之奈何?

岐伯曰:必先扪而循之,切而散之,推而按之,弹而怒之,抓而下之,通而取之,外引其门,以闭其神。呼尽内针,静以久留,以气至为故,如待所贵,不知日暮。其气以至,适而自护,候吸引针,气不得出,各在其处,推阖其门,令神气存,大气留止,故命曰补。

帝曰:候气奈何?

岐伯曰:夫邪去络入于经也,舍于血脉之中,其寒温未相得,如涌波之起也,时来时去,故不常在。故曰,方其来也,必按而止之,止而取之,无逢其冲而泻之。

真气者,经气也,经气太虚,故曰:“其来不可逢。”此之谓也。故曰:“候邪不审,大气已过,泻之则真气脱,脱则不复,邪气复至,而病益蓄。”故曰:“其往不可追。”此之谓也。不可挂以发者,待邪之至时而发针泻矣。若先若后者,血气已尽(全元起本为虚),其病不可下。故曰:“知其可取如发机,不知其取如扣椎(应为楗之误)。”故曰:“知机道者不可挂以发,不知机者扣之不发。”此之谓也。

帝曰:补泻奈何?

岐伯曰:此攻邪也。疾出以去盛血,而复其真气。此邪新客,溶溶未有定处也。推之则前,引之则止,逆而刺之,温血也。刺出其血,其病立已。

帝曰:善。然真邪以合,波陇不起,候之奈何?

岐伯曰:审扪循三部九候之盛虚而调之。察其左右上下,相失及相减者,审其病脏以期之。不知三部者,阴阳不别,天地不分。地以候地,天以候天,人以候人,调之中府,以定三部。故曰:“刺不知三部九候病脉之处,虽有大过且至,工不能禁也。”诛罚无过,命曰大惑,反乱大经,真不可复。用实为虚,以邪为真,用针无义,反为气贼。夺人正气,以从为逆,荣卫散乱,真气已失,邪独内着,绝人长命,予人夭殃。不知三部九候,故不能久长。因不知合之四时五行,因加相胜,释邪攻正,绝人长命。邪之新客来也,未有定处,推之则前,引之则止,逢而泻之,其病立已。
\section{通评虚实论}
%%	28	%%
黄帝问曰:何谓虚实?

岐伯对曰:邪气盛则实,精气夺则虚。

帝曰:虚实何如?

岐伯曰:气虚者,肺虚也。气逆者,足寒也。非其时则生,当其时则死。余脏皆如此。

帝曰:何谓重实?

岐伯曰:所谓重实者,言大热病,气热脉满,是谓重实。

帝曰:经络俱实何如?何以治之?

岐伯曰:经络皆实,是寸脉急而尺缓也,皆当治之。故曰:“滑则从,涩则逆也。”夫虚实者,皆从其物类始,故五脏骨肉,滑利可以长久也。

帝曰:络气不足,经气有余,何如?

岐伯曰:络气不足,经气有余者,脉口热而尺寒也。秋冬为逆,春夏为从,治主病者。

帝曰:经虚络满何如?

岐伯曰:经虚络满者,尺热满,脉口寒涩也。此春夏死,秋冬生也。

帝曰:治此者奈何?

岐伯曰:络满经虚,灸阴刺阳;经满络虚,刺阴灸阳。

帝曰:何谓重虚?

岐伯曰:脉气(《甲乙经》为虚)、上(《甲乙经》为气)虚、尺虚,是为重虚。

帝曰:何以治之?

岐伯曰:所谓气虚者,言无常也;尺虚者,行步恇然。脉虚者,不象阴也。如此者,滑则生,涩则死也。

帝曰:寒气暴上,脉满而实,何如?

岐伯曰:实而滑则生,实而逆则死。

帝曰:脉实满,手足寒,头热,何如?

岐伯曰:春秋则生,冬夏则死。脉浮而涩,涩而身有热者死。

帝曰:其形尽满何如?

岐伯曰:其形尽满者,脉急大坚,尺涩而不应也,如是者,故从则生,逆则死。

帝曰:何谓从则生,逆则死?

岐伯曰:所谓从者,手足温也;所谓逆者,手足寒也。

帝曰:乳子而病热,脉悬小者,何如?

岐伯曰:手足温则生,寒则死。

帝曰:乳子中风热,喘鸣肩息者,脉何如?

岐伯曰:喘鸣肩息者,脉实大也,缓则生,急则死。

帝曰:肠澼便血,何如?

岐伯曰:身热则死,寒则生。

帝曰:肠澼下白沫,何如?

岐伯曰:脉沉则生,脉浮则死。

帝曰:肠澼下脓血,何如?

岐伯曰:脉悬绝则死,滑大则生。

帝曰:肠澼之属,身不热,脉不悬绝,何如?

岐伯曰:滑大者曰生,悬涩者曰死,以脏期之。

帝曰:癫疾何如?

岐伯曰:脉搏大滑,久自已;脉小坚急,死不治。

帝曰:癫疾之脉,虚实何如?

岐伯曰:虚则可治,实则死。

帝曰:消瘅虚实何如?

岐伯曰:脉实大,病久可治;脉悬小坚,病久不可治。

帝曰:形度、骨度、脉度、筋度,何以知其度也?(读注:此处有问无答,且与上下文不搭,明清医家已经认为此处是错简而形成的缀文。)

帝曰:春亟治经络,夏亟治经俞,秋亟治六腑府,冬则闭塞。闭塞者,用药而少针石也。所谓少针石者,非痈疽之谓也。痈疽不得顷时回,痈不知所,按之不应手,乍来乍已,刺手太阴傍三痏,与缨脉各二。掖痈大热,刺足少阳五,刺而热不止,刺手心主三,刺手太阴经络者,大骨之会各三。暴痈筋软,随分而痛,魄汗不尽,胞气不足,治在经俞。

腹暴满,按之不下,取手太阳经络者,胃之募也。少阴俞去脊椎三寸傍五,用圆利针。

霍乱,刺俞旁五,足阳明及上傍三。

刺痫惊脉五,针手太阴各五,刺经太阳五,刺手少阴经络傍者一,足阳明一,上踝五寸刺三针。

凡治消瘅、仆击、偏枯、痿厥、气满发逆,肥贵人,则高粱之疾也。隔塞闭绝,上下不通,则暴忧之病也。暴厥而聋,偏塞闭不通,内气暴薄也。不从内外中风之病,故瘦留着也。跖跛,寒风湿之病也。

黄帝曰:黄疸、暴痛、癫疾、厥狂,久逆之所生也。五脏不平,六腑闭塞之所生也。头痛耳鸣,九窍不利,肠胃之所生也。
\section{太阴阳明论}
%%	29	%%
黄帝问曰:太阴阳明为表里,脾胃脉也,生病而异者,何也?

岐伯对曰:阴阳异位,更虚更实,更逆更从,或从内,或从外,所从不同,故病异名也。

帝曰:愿闻其异状也?

岐伯曰:阳者天气也,主外;阴者地气也,主内。故阳道实,阴道虚。故犯贼风虚邪者,阳受之;食饮不节,起居不时者,阴受之。阳受之则入六腑,阴受之则入五脏。入六腑则身热不时卧,上为喘呼。入五脏则䐜满闭塞,下为飧泄,久为肠澼。故喉主天气,咽主地气。故阳受风气,阴受湿气。故阴气从足上行至头,而下行循臂至指端。阳气从手上行至头,而下行至足。故曰:“阳病者上行极而下,阴病者下行极而上。”故伤于风者,上先受之;伤于湿者,下先受之。

帝曰:脾病而四肢不用,何也?

岐伯曰:四肢皆禀气于胃,而不得至经,必因于脾,乃得禀也。今脾病不能为胃行其津液,四肢不得禀水谷气,气日以衰,脉道不利,筋骨肌肉,皆无气以生,故不用焉。

帝曰:脾不主时,何也?

岐伯曰:脾者土也,治中央,常以四时长四脏,各十八日寄治,不得独主于时也。脾脏者,常着胃土之精也。土者,生万物而法天地,故上下至头足,不得主时也。

帝曰:脾与胃以膜相连耳,而能为之行其津液,何也?

岐伯曰:足太阴者三阴也,其脉贯胃属脾络嗌,故太阴为之行气于三阴。阳明者表也,五脏六腑之海也,亦为之行气于三阳。脏腑各因其经而受气于阳明,故为胃行其津液。四肢不得禀水谷气,日以益衰,阴道不利,筋骨肌肉,无气以生,故不用焉。
\section{阳明脉解}
%%	30	%%
黄帝问曰:足阳明之脉病,恶人与火,闻木音则惕然而惊,钟鼓不为动。闻木音而惊,何也?愿闻其故。

岐伯对曰:阳明者,胃脉也,胃者土也。故闻木音而惊者,土恶木也。

帝曰:善。其恶火何也?

岐伯曰:阳明主肉,其脉血气盛,邪客之则热,热甚则恶火。

帝曰:其恶人何也?

岐伯曰:阳明厥则喘而惋,惋则恶人。

帝曰:或喘而死者,或喘而生者,何也?

岐伯曰:厥逆连脏则死,连经则生。

帝曰:善。病甚则弃衣而走,登高而歌,或至不食数日,踰垣上屋,所上之处,皆非其素所能也,病反能者,何也?

岐伯曰:四肢者,诸阳之本也,阳盛则四肢实,实则能登高也。

帝曰:其弃衣而走者,何也?

岐伯曰:热盛于身,故弃衣欲走也。

帝曰:其妄言骂詈,不避亲疏而歌者,何也?

岐伯曰:阳盛则使人妄言骂詈,不避亲疏而不欲食,不欲食,故妄走也。
\section{热论}
%%	31	%%
黄帝问曰:今夫热病者,皆伤寒之类也,或愈或死,其死皆以六七日之间,其愈皆以十日以上者,何也?不知其解,愿闻其故。

岐伯对曰:巨阳者,诸阳之属也。其脉连于风府,故为诸阳主气也。人之伤于寒也,则为病热,热虽甚不死,其两感于寒而病者,必不免于死。

帝曰:愿闻其状。

岐伯曰:伤寒一日,巨阳受之,故头项痛,腰脊强。二日,阳明受之,阳明主肉,其脉侠鼻,络于目,故身热目疼而鼻干,不得卧也。三日,少阳受之,少阳主胆(《甲乙经》《太素》等并作骨),其脉循胁络于耳,故胸胁痛而耳聋。三阳经络皆受其病,而未入于脏(《太素》作腑)者,故可汗而已。

四日,太阴受之,太阴脉布胃中,络于嗌,故腹满而嗌干。五日,少阴受之,少阴脉贯肾,络于肺,系舌本,故口燥舌干而渴。六日,厥阴受之,厥阴脉循阴器而络于肝,故烦满而囊缩。三阴三阳五脏六腑皆受病,荣卫不行,五脏不通则死矣。

其不两感于寒者:七日,巨阳病衰,头痛少愈;八日,阳明病衰,身热少愈;九日,少阳病衰,耳聋微闻,十日,太阴病衰,腹减如故,则思饮食;十一日,少阴病衰,渴止不满,舌干已而嚏;十二日,厥阴病衰,囊纵,少腹微下,大气皆去,病日已矣。

帝曰:治之奈何?

岐伯曰:治之各通其脏脉,病日衰已矣。其未满三日者,可汗而已;其满三日者,可泄而已。

帝曰:热病已愈,时有所遗者,何也?

岐伯曰:诸遗者,热甚而强食之,故有所遗也。若此者,皆病已衰而热有所藏,因其谷气相薄,两热相合,故有所遗也。

帝曰:善。治遗奈何?

岐伯曰:视其虚实,调其逆从,可使必已矣。

帝曰:病热当何禁之?

岐伯曰:病热少愈,食肉则复,多食则遗,此其禁也。

帝曰:其病两感于寒者,其脉应与其病形何如?

岐伯曰:两感于寒者:病一日则巨阳与少阴俱病,则头痛口干而烦满;二日则阳明与太阴俱病,则腹满身热,不欲食,谵言;三日则少阳与厥阴俱病,则耳聋、囊缩而厥、水浆不入、不知人,六日死。

帝曰:五脏已伤,六腑不通,荣卫不行,如是之后,三日乃死,何也?

岐伯曰:阳明者,十二经脉之长也,其血气盛,故不知人。三日,其气乃尽,故死矣。

凡病伤寒而成温者,先夏至日者为病温,后夏至日者为病暑。暑当与汗皆出,勿止。


\section{刺热}
%%	32	%%
肝热病者,小便先黄,腹痛,多卧,身热。热争则狂言及惊,胁满痛,手足躁,不得安卧。庚辛甚,甲乙大汗,气逆则庚辛死。刺足厥阴、少阳。其逆则头痛员员,脉引冲头也。

心热病者,先不乐,数日乃热。热争则卒心痛,烦闷,善呕,头痛,面赤,无汗。壬癸甚,丙丁大汗,气逆则壬癸死。刺手少阴、太阳。

脾热病者,先头重,颊痛,烦心,颜青,欲呕,身热。热争则腰痛,不可用俛仰,腹满泄,两颔痛。甲乙甚,戊己大汗,气逆则甲乙死。刺足太阴、阳明。

肺热病者,先淅然厥,起毫毛,恶风寒,舌上黄,身热。热争则喘咳,痛走胸膺背,不得大息,头痛不堪,汗出而寒。丙丁甚,庚辛大汗,气逆则丙丁死。刺手太阴、阳明,出血如大豆,立已。

肾热病者,先腰痛,胻酸,苦渴数饮,身热。热争则项痛而强,胻寒且酸,足下热,不欲言,其逆则项痛员员澹澹然。戊己甚,壬癸大汗,气逆则戊己死。刺足少阴、太阳。

诸汗者,至其所胜日,汗出也。

肝热病者,左颊先赤;心热病者,颜先赤;脾热病者,鼻先赤;肺热病者,右颊先赤;肾热病者,颐先赤。病虽未发,见赤色者刺之,名曰治未病。

热病从部所起者,至期而已。其刺之反者,三周而已。重逆则死。诸当汗者,至其所胜日,汗大出也。

诸治热病,以饮之寒水,乃刺之,必寒衣之,居止寒处。身寒而止也。

热病先胸胁痛,手足躁,刺足少阳,补足太阴。病甚者为五十九刺。热病始手臂痛者,刺手阳明太阴而汗出止。热病始于头首者,刺项太阳而汗出止。热病始于足胫者,刺足阳明而汗出止。热病先身重骨痛,耳聋好瞑,刺足少阴,病甚为五十九刺。热病先眩冒而热,胸胁满,刺足少阴、少阳。

太阳之脉,色荣颧骨,热病也。荣未交,曰今且得汗,待时而已。与厥阴脉争见者,死期不过三日,其热病内连肾,少阳之脉色也。少阳之脉,色荣颊前,热病也。荣未交,曰今且得汗,待时而已。与少阴脉争见者,死期不过三日。

热病气穴:三椎下间,主胸中热;四椎下间,主鬲中热;五椎下间,主肝热;六椎下间,主脾热;七椎下间,主肾热;荣在骶也,项上三椎陷者中也。

颊下逆颧为大瘕。下牙车为腹满。颧后为胁痛。颊上者,鬲上也。
\section{评热病论}
%%	33	%%
黄帝问曰:有病温者,汗出辄复热,而脉躁疾不为汗衰,狂言不能食,病名为何?

岐伯对曰:病名阴阳交,交者死也。

帝曰:愿闻其说。

岐伯曰:人所以汗出者,皆生于谷,谷生于精,今邪气交争于骨肉而得汗者,是邪却而精胜也。精胜则当能食而不复热。复热者邪气也,汗者精气也,今汗出而辄复热者,是邪胜也。不能食者,精无俾也。病而留者,其寿可立而倾也。且夫《热论》曰:“汗出而脉尚躁盛者死。”今脉不与汗相应,此不胜其病也,其死明矣。狂言者是失志,失志者死。今见三死,不见一生,虽愈必死也。

帝曰:有病身热汗出烦满,烦满不为汗解,此为何病?

岐伯曰:汗出而身热者,风也;汗出而烦满不解者,厥也。病名曰风厥。

帝曰:愿卒闻之。

岐伯曰:巨阳主气,故先受邪,少阴与其为表里也,得热则上从之,从之则厥也。

帝曰:治之奈何?

岐伯曰:表里刺之,饮之服汤。

帝曰:劳风为病何如?

岐伯曰:劳风法在肺下,其为病也,使人强上,冥视,唾出若涕,恶风而振寒,此为劳风之病。

帝曰:治之奈何?

岐伯曰:以救俛仰。巨阳引,精者三日,中年者五日,不精者七日。咳出青黄涕,其状如脓,大如弹丸,从口中若鼻中出,不出则伤肺,伤肺则死也。

帝曰:有病肾风者,面、胕痝然壅,害于言,可刺不?

岐伯曰:虚不当刺,不当刺而刺,后五日其气必至。

帝曰:其至何如?

岐伯曰:至必少气时热,时热从胸背上至头,汗出,手热,口干苦渴,小便黄,目下肿,腹中鸣,身重难以行,月事不来,烦而不能食,不能正偃,正偃则咳,病名曰风水,论在《刺法》中。

帝曰:愿闻其说。

岐伯曰:邪之所凑,其气必虚。阴虚者,阳必凑之,故少气时热而汗出也。小便黄者,少腹中有热也。不能正偃者,胃中不和也。正偃则咳甚,上迫肺也。诸有水气者,微肿先见于目下也。

帝曰:何以言?

岐伯曰:水者阴也,目下亦阴也,腹者至阴之所居,故水在腹者,必使目下肿也。真气上逆,故口苦舌干,卧不得正偃,正偃则咳出清水也。诸水病者,故不得卧,卧则惊,惊则咳甚也。腹中鸣者,病本于胃也。薄脾则烦,不能食。食不下者,胃脘隔也。身重难以行者,胃脉在足也。月事不来者,胞脉闭也。胞脉者,属心而络于胞中。今气上迫肺,心气不得下通,故月事不来也。

帝曰:善。
\section{逆调论}
%%	34	%%
黄帝问曰:人身非常温也,非常热也,为之热而烦满者,何也?

伯对曰:阴气少而阳气胜,故热而烦满也。

帝曰:人身非衣寒也,中非有寒气也,寒从中生者何?

岐伯曰:是人多痹气也。阳气少,阴气多,故身寒如从水中出。

帝曰:人有四肢热,逢风寒如炙如火者,何也?

岐伯曰:是人者,阴气虚,阳气盛。四肢者阳也,两阳相得而阴气虚少,少水不能灭盛火,而阳独治。独治者,不能生长也,独胜而止耳。逢风而如炙如火者,是人当肉烁也。

帝曰:人有身寒,汤火不能热,厚衣不能温,然不冻栗,是为何病?

岐伯曰:是人者,素肾气胜,以水为事,太阳气衰,肾脂枯不长,一水不能胜两火。肾者水也,而生于骨,肾不生,则髓不能满,故寒甚至骨也。所以不能冻栗者,肝一阳也,心二阳也,肾孤脏也,一水不能胜二火,故不能冻栗,病名曰骨痹,是人当挛节也。

帝曰:人之肉苛者,虽近衣絮,犹尚苛也,是谓何疾?

岐伯曰:荣气虚,卫气实也。荣气虚则不仁,卫气虚则不用,荣卫俱虚,则不仁且不用,肉如故(《太素》作苛)也。人身与志不相有,曰死。

帝曰:人有逆气不得卧而息有音者,有不得卧而息无音者,有起居如故而息有音者,有得卧行而喘者,有不得卧不能行而喘者,有不得卧卧而喘者,皆何脏使然?愿闻其故。

岐伯曰:不得卧而息有音者,是阳明之逆也。足三阳者下行,今逆而上行,故息有音也。阳明者,胃脉也,胃者,六府之海,其气亦下行,阳明逆,不得从其道,故不得卧也。《下经》曰:“胃不和,则卧不安。”此之谓也。夫起居如故而息有音者,此肺之络脉逆也。络脉不得随经上下,故留经而不行。络脉之病人也微,故起居如故而息有音也。夫不得卧,卧则喘者,是水气之客也。夫水者,循津液而流也。肾者水脏,主津液,主卧与喘也。

帝曰:善。
\section{疟论}
%%	35	%%
黄帝问曰:夫痎疟皆生于风,其蓄作有时者何也?

岐伯对曰:疟之始发也,先起于毫毛,伸欠乃作,寒栗鼓颔,腰脊俱痛;寒去则内外皆热,头痛如破,渴欲冷饮。

帝曰:何气使然?愿闻其道。

岐伯曰:阴阳上下交争,虚实更作,阴阳相移也。阳并于阴,则阴实而阳虚,阳明虚则寒栗,鼓颔也;巨阳虚则腰背头项痛;三阳俱虚则阴气胜,阴气胜则骨寒而痛;寒生于内,故中外皆寒;阳盛则外热,阴虚则内热,外内皆热则喘而渴,故欲冷饮也。此皆得之夏伤于暑,热气盛,藏于皮肤之内,肠胃之外,此荣气之所舍也。此令人汗空疏,腠理开,因得秋气,汗出遇风,及得之以浴水,气舍于皮肤之内,与卫气并居。卫气者,昼日行于阳,夜行于阴,此气得阳而外出,得阴而内薄,内外相薄,是以日作。

帝曰:疟疾间日而发作者为何也?

岐伯曰:其气之舍深,内薄于阴,阳气独发,阴邪内着,阴与阳争不得出,是以间日而作也。

帝曰:善。其作日晏与其日早者,何气使然?

岐伯曰:邪气客于风府,循膂而下,卫气一日一夜大会于风府,其明日,日下一节,故其作也晏。此先客于脊背也,每至于风府则腠理开,腠理开则邪气入,邪气入则病作,以此日作稍益晏也。其出于风府,日下一节,二十五日下至骶骨,二十六日入于脊内,注于伏膂之脉,其气上行,九日出于缺盆之中,其气日高,故作日益早也。其间日发者,由邪气内薄于五脏,横连募原也。其道远,其气深,其行迟,不能与卫气俱行,不得皆出,故间日乃作也。

帝曰:夫子言卫气每至于风府,腠理乃发,发则邪气入,入则病作。今卫气日下一节,其气之发也,不当风府,其日作者奈何?

岐伯曰:此邪气客于头项,循膂而下者也。故虚实不同,邪中异所,则不得当其风府也。故邪中于头项者,气至头项而病;中于背者,气至背而病;中于腰脊者,气至腰脊而病;中于手足者,气至手足而病。卫气之所在,与邪气相合,则病作。故风无常府,卫气之所发,必开其腠理,邪气之所合,则其府也。

帝曰:善。夫风之与疟也,相似同类,而风独常在,疟得有时而休者,何也?

岐伯曰:风气留其处,故常在。疟气随经络,沉以内薄,故卫气应乃作。

帝曰:疟先寒而后热者何也?

岐伯曰:夏伤于大暑,其汗大出,腠理开发,因遇夏气凄沧之水寒,藏于腠理皮肤之中,秋伤于风,则病成矣。夫寒者,阴气也;风者,阳气也。先伤于寒而后伤于风,故先寒而后热也。病以时作,名曰寒疟。

帝曰:先热而后寒者何也?

岐伯曰:此先伤于风而后伤于寒,故先热而后寒也。亦以时作,名曰温疟。其但热而不寒者,阴气先绝,阳气独发,则少气烦冤,手足热而欲呕,名曰瘅疟。

帝曰:夫经言有余者泻之,不足者补之,今热为有余,寒为不足,夫疟者之寒,汤火不能温也,及其热,冰水不能寒也,此皆有余不足之类,当此之时,良工不能止,必须其自衰乃刺之,其故何也?愿闻其说。

岐伯曰:经言无刺熇熇之热,无刺浑浑之脉,无刺漉漉之汗,故为其病逆,未可治也。夫疟之始发也,阳气并于阴,当是之时,阳虚而阴盛,外无气,故先寒栗也。阴气逆极则复出之阳,阳与阴复并于外,则阴虚而阳实,故先热而渴。

夫疟气者,并于阳则阳胜,并于阴则阴胜,阴胜则寒,阳胜则热。疟者,风寒之气不常也,病极则复至。病之发也,如火之热,如风雨不可当也。故经言曰:“方其盛时必毁,因其衰也,事必大昌。”此之谓也。夫疟之未发也,阴未并阳,阳未并阴,因而调之,真气得安,邪气乃亡。故工不能治其已发,为其气逆也。

帝曰:善。攻之奈何?早晏何如?

岐伯曰:疟之且发也,阴阳之且移也,必从四末始也。阳已伤,阴从之,故先其时坚束其处,令邪气不得入,阴气不得出,审候见之,在孙络盛坚而血者皆取之,此真往而未得并者也。

帝曰:疟不发,其应何如?

岐伯曰:疟气者,必更盛更虚,当气之所在也。病在阳则热而脉躁,在阴则寒而脉静,极则阴阳俱衰,卫气相离,故病得休,卫气集则复病也。

帝曰:时有间二日或至数日发,或渴或不渴,其故何也?

岐伯曰:其间日者,邪气与卫气客于六腑,而有时相失,不能相得,故休数日乃作也。疟者,阴阳更胜也,或甚或不甚,故或渴或不渴。

帝曰:论言夏伤于暑,秋必病疟,今疟不必应者何者?

岐伯曰:此应四时者也。其病异形者,反四时也。其以秋病者寒甚,以冬病者寒不甚,以春病者恶风,以夏病者多汗。

帝曰:夫病温疟与寒疟,而皆安舍,舍于何脏?

岐伯曰:温疟者,得之冬中于风,寒气藏于骨髓之中,至春则阳气大发,邪气不能自出,因遇大暑,脑髓烁,肌肉消,腠理发泄,或有所用力,邪气与汗皆出。此病藏于肾,其气先从内出之于外也。如是者,阴虚而阳盛,阳盛则热矣。衰则气复反入,入则阳虚,阳虚则寒矣。故先热而后寒,名曰温疟。

帝曰:瘅疟何如?

岐伯曰:瘅疟者,肺素有热,气盛于身,厥逆上冲,中气实而不外泄,因有所用力,腠理开,风寒舍于皮肤之内,分肉之间而发,发则阳气盛,阳气盛而不衰则病矣。其气不及于阴,故但热而不寒,气内藏于心,而外舍于分肉之间,令人消烁脱肉,故命曰瘅疟。

帝曰:善。
\section{刺疟}
%%	36	%%
足太阳之疟,令人腰痛头重,寒从背起,先寒后热,熇熇暍暍然,热止汗出,难已,刺郄中出血。足少阳之疟,令人身体解㑊,寒不甚,热不甚,恶见人,见人心惕惕然,热多汗出甚,刺足少阳。足阳明之疟,令人先寒洒浙,洒浙寒甚,久乃热,热去汗出,喜见日月光火气,乃快然,刺足阳明跗上。足太阴之疟,令人不乐,好太息,不嗜食,多寒热,汗出;病至则善呕,呕已乃衰,即取之。足少阴之疟,令人呕吐甚,多寒热,热多寒少,欲闭户牖而处,其病难已。足厥阴之疟,令人腰痛,少腹满,小便不利,如癃状,非癃也。数便,意恐惧,气不足,腹中悒悒,刺足厥阴。

肺疟者,令人心寒,寒甚热,热间善惊,如有所见者,刺手太阴、阳明。心疟者,令人烦心甚,欲得清水,反寒多,不甚热,刺手少阴。肝疟者,令人色苍苍然,太息,其状若死者,刺足厥阴见血。脾疟者,令人寒,腹中痛,热则肠中鸣,鸣已汗出,刺足太阴。肾疟者,令人洒洒然,腰脊痛宛转,大便难,目眴眴然,手足寒,刺足太阳、少阴。胃疟者,令人且病也,善饥而不能食,食而支满腹大,刺足阳明、太阴横脉出血。

疟发身方热,刺跗上动脉,开其空,出其血,立寒。疟方欲寒,刺手阳明、太阴,足阳明、太阴。疟脉满大急,刺背俞,用中针,傍五胠俞各一,适肥瘦,出其血也。疟脉小实急,灸陉少阴,刺指井。疟脉满大急,刺背俞,用五胠俞、背俞各一,适行至于血也。疟脉缓大虚,便宜用药,不宜用针。凡冶疟,先发如食顷,乃可以治;过之,则失时也。诸疟而脉不见,刺十指间出血,血去必已。先视身之赤如小豆者,尽取之。

十二疟者,其发各不同时,察其病形,以知其何脉之病也。先其发时,如食顷而刺之,一刺则衰,二刺则知,三刺则已。不已,刺舌下两脉出血;不已,刺郄中盛经出血,又刺项已下侠脊者,必已。舌下两脉者廉泉也。

刺疟者,必先问其病之所先发者,先刺之。先头痛及重者,先刺头上及两额、两眉间出血。先项背痛者,先刺之。先腰脊痛者,先刺郄中出血。先手臂痛者,先刺手少阴、阳明十指间。先足胫酸痛者,先刺足阳明十指间出血。

风疟,疟发则汗出,恶风,刺三阳经背俞之血者。胻酸痛甚,按之不可,名曰胕髓病,以馋针,针绝骨,出血立已。身体小痛,刺至阴。诸阴之井,无出血,间日一刺。疟不渴,间日而作,刺足太阳;渴而间日作,刺足少阳。温疟汗不出,为五十九刺。
\section{气厥论}
%%	37	%%
黄帝问曰:五脏六腑,寒热相移者何?

岐伯对曰:肾移寒于肝(《太素》、《甲乙经》作脾),痈肿、少气。脾移寒于肝,痈肿、筋挛。肝移寒于心,狂,隔中。心移寒于肺,肺消。肺消者,饮一溲二,死不治。肺移寒于肾,为涌水。涌水者,按腹不坚,水气客于大肠,疾行则鸣濯濯,如囊裹浆,水之病也。

脾移热于肝,则为惊衄。肝移热于心,则死。心移热于肺,传为鬲消。肺移热于肾,传为柔痓。肾移热于脾,传为虚,肠澼,死不可治。

胞移热于膀胱,则癃,溺血。膀胱移热于小肠,鬲肠不便,上为口糜。小肠移热于大肠,为虙瘕,为沉。大肠移热于胃,善食而瘦,又谓之食亦。胃移热于胆,亦曰食亦。胆移热于脑,则辛頞鼻渊,鼻渊者,浊涕下不止也,传为衄蔑、瞑目。

故得之气厥也。
\section{咳论}
%%	38	%%
黄帝问曰:肺之令人咳,何也?

岐伯对曰:五脏六腑,皆令人咳,非独肺也。

帝曰:愿闻其状。

岐伯曰:皮毛者,肺之合也。皮毛先受邪气,邪气以从其合也。其寒饮食入胃,从肺脉上至于肺则肺寒,肺寒则外内合邪,因而客之,则为肺咳。五脏各以其时受病,非其时各传以与之。人与天地相参,故五脏各以治时,感于寒则受病,微则为咳,甚则为泄、为痛。乘秋则肺先受邪,乘春则肝先受之,乘夏则心先受之,乘至阴则脾先受之,乘冬则肾先受之。

帝曰:何以异之?

岐伯曰:肺咳之状,咳而喘息有音,甚则唾血。心咳之状,咳则心痛,喉中介介如梗状,甚则咽肿喉痹。肝咳之状,咳则两胁下痛,甚则不可以转,转则两胠下满。脾咳之状,咳则右胁下痛,阴阴引肩背,甚则不可以动,动则咳剧。肾咳之状,咳则腰背相引而痛,甚则咳涎。

帝曰:六腑之咳奈何?安所受病?

岐伯曰:五脏之久咳,乃移于六腑。脾咳不已,则胃受之。胃咳之状,咳而呕,呕甚则长虫出。肝咳不已,则胆受之。胆咳之状,咳呕胆汁。肺咳不已,则大肠受之。大肠咳状,咳而遗失(《甲乙经》作矢)。心咳不已,则小肠受之。小肠咳状,咳而失气,气与咳俱失。肾咳不已,则膀胱受之。膀胱咳状,咳而遗溺,久咳不已,则三焦受之。三焦咳状,咳而腹满,不欲食饮。此皆聚于胃,关于肺,使人多涕唾,而面浮肿气逆也。

帝曰:治之奈何?

岐伯曰:治脏者治其俞,治腑者治其合,浮肿者治其经。

帝曰:善。


\section{举痛论}
%%	39	%%
黄帝问曰:余闻善言天者,必有验于人;善言古者,必有合于今;善言人者,必有厌于己。如此则道不惑而要数极,所谓明也。今余问于夫子,令言而可知,视而可见,扪而可得,令验于己而发蒙解惑,可得而闻乎?

岐伯再拜稽首对曰:何道之问也?

帝曰:愿闻人之五脏卒痛,何气使然?

岐伯对曰:经脉流行不止,环周不休。寒气入经而稽迟,泣(涩的通假字,下同)而不行,客于脉外则血少,客于脉中则气不通,故卒然而痛。

帝曰:其痛或卒然而止者,或痛甚不休者,或痛甚不可按者,或按之而痛止者,或按之无益者,或喘动应手者,或心与背相引而痛者,或胁肋与少腹相引而痛者,或腹痛引阴股者,或痛宿昔而成积者,或卒然痛死不知人,有少间复生者,或痛而呕者,或腹痛而后泄者,或痛而闭不通者。凡此诸痛,各不同形,别之奈何?

岐伯曰:寒气客于脉外则脉寒,脉寒则缩蜷,缩蜷则脉绌急,绌急则外引小络,故卒然而痛。得炅则痛立止,因重中于寒,则痛久矣。寒气客于经脉之中,与炅气相薄则脉满,满则痛而不可按也。寒气稽留,炅气从上,则脉充大而血气乱,故痛甚不可按也。寒气客于肠胃之间,膜原之下,血不得散,小络急引故痛,按之则血气散,故按之痛止。寒气客于侠脊之脉,则深按之不能及,故按之无益也。寒气客于冲脉,冲脉起于关元,随腹直上,寒气客则脉不通,脉不通则气因之,故喘动应手矣。寒气客于背俞之脉则脉泣,脉泣则血虚,血虚则痛。其俞注于心,故相引而痛。按之则热气至,热气至则痛止矣。寒气客于厥阴之脉,厥阴之脉者,络阴器,系于肝,寒气客于脉中,则血泣脉急,故胁肋与少腹相引痛矣。厥气客于阴股,寒气上及少腹,血泣在下相引,故腹痛引阴股。寒气客于小肠膜原之间,络血之中,血泣不得注于大经,血气稽留不得行,故宿昔而成积矣。寒气客于五脏,厥逆上泄,阴气竭,阳气未入,故卒然痛死不知人,气复反则生矣。寒气客于肠胃,厥逆上出,故痛而呕也。寒气客于小肠,小肠不得成聚,故后泄腹痛矣。热气留于小肠,肠中痛、瘅热、焦渴,则坚干不得出,故痛而闭不通矣。

帝曰:所谓言而可知者也,视而可见,奈何?

岐伯曰:五脏六腑,固尽有部,视其五色,黄赤为热,白为寒,青黑为痛,此所谓视而可见者也。

帝曰:扪而可得奈何?

岐伯曰:视其主病之脉,坚而血及陷下者,皆可扪而得也。

帝曰:善。余知百病生于气也,怒则气上,喜则气缓,悲则气消,恐则气下,寒则气收,炅则气泄,惊则气乱,劳则气耗,思则气结。九气不同,何病之生?

岐伯曰:怒则气逆,甚则呕血及飧泄,故气上矣。喜则气和志达,荣卫通利,故气缓矣。悲则心系急,肺布叶举,而上焦不通,荣卫不散,热气在中,故气消矣。恐则精却,却则上焦闭,闭则气还,还则下焦胀,故气不(当为下之误)行矣。寒则腠理闭,气不行,故气收矣。炅则腠理开,荣卫通,汗大泄,故气泄。惊则心无所倚,神无所归,虑无所定,故气乱矣。劳则喘息汗出,外内皆越,故气耗矣。思则心有所存,神有所归,正气留而不行,故气结矣。
\section{腹中论}
%%	40	%%
黄帝问曰:有病心腹满,旦食则不能暮食,此为何病?

岐伯对曰:名为鼓胀。

帝曰:治之奈何?

岐伯曰:治之以鸡矢醴,一剂知,二剂已。

帝曰:其时有复发者何也?

岐伯曰:此饮食不节,故时有病也。虽然其病且已,时故当病,气聚于腹也。

帝曰:有病胸胁支满者,妨于食,病至则先闻腥臊臭,出清液,先唾血,四肢清,目眩,时时前后血,病名为何?何以得之?

岐伯曰:病名血枯,此得之年少时,有所大脱血,若醉入房中,气竭肝伤,故月事衰少不来也。

帝曰:治之奈何?复以何术?

岐伯曰:以四乌鲗骨,一藘茹,二物并合之,丸以雀卵,大如小豆,以五丸为后饭,饮以鲍鱼汁,利肠中及伤肝也。

帝曰:病有少腹盛,上下左右皆有根,此为何病?可治不?

岐伯曰:病名曰伏梁。

帝曰:伏梁何因而得之?

岐伯曰:裹大脓血,居肠胃之外,不可治,治之每切按之致死。

帝曰:何以然?

岐伯曰:此下则因阴,必下脓血,上则迫胃脘,生(为出之误)鬲,侠(《太素》作使)胃脘内痈。此久病也,难治。居脐上为逆,居脐下为从,勿动亟夺。论在《刺法》中。

帝曰:人有身体髀股胻皆肿,环脐而痛,是为何病?

岐伯曰:病名伏梁,此风根也。其气溢于大肠而着于肓,肓之原在脐下,故环脐而痛也。不可动之,动之为水溺涩之病。

帝曰:夫子数言热中消中,不可服高梁、芳草、石药。石药发瘨,芳草发狂。夫热中消中者,皆富贵人也,今禁高梁,是不合其心,禁芳草石药,是病不愈,愿闻其说。

岐伯曰:夫芳草之气美,石药之气悍,二者其气急疾坚劲,故非缓心和人,不可以服此二者。

帝曰:不可以服此二者,何以然?

岐伯曰:夫热气慓悍,药气亦然,二者相遇,恐内伤脾。脾属土而恶木,服此药者,至甲乙日更论。

帝曰:善。有病膺肿、颈痛、胸满、腹胀,此为何病?何以得之?

岐伯曰:名厥逆。

帝曰:治之奈何?

岐伯曰:灸之则瘖,石之则狂,须其气并,乃可治也。

帝曰:何以然?

岐伯曰:阳气重上,有余于上,灸之则阳气入阴,入则瘖;石之则阳气虚,虚则狂。须其气并而治之,可使全也。

帝曰:善。何以知怀子之且生也?

岐伯曰:身有病而无邪脉也。

帝曰:病热而有所痛者,何也?

岐伯曰:病热者,阳脉也。以三阳之动也,人迎一盛少阳,二盛太阳,三盛阳明,入阴也。夫阳入于阴,故病在头与腹,乃䐜胀而头痛也。

帝曰:善。
\section{刺腰痛}
%%	41	%%
足太阳脉令人腰痛,引项脊尻背如重状,刺其郄中。太阳正经出血,春无见血。

少阳令人腰痛,如以针刺其皮中,循循然不可以俛仰,不可以顾。刺少阳成骨之端出血,成骨在膝外廉之骨独起者,夏无见血。

阳明令人腰痛,不可以顾,顾如有见者,善悲。刺阳明于胻前三痏,上下和之出血,秋无见血。

足少阴令人腰痛,痛引脊内廉。刺少阴于内踝上二痏,春无见血,出血太多,不可复也。

厥阴之脉令人腰痛,腰中如张弓弩弦。刺厥阴之脉,在腨踵鱼腹之外,循之累累然,乃刺之。其病令人善言,默默然不慧,刺之三痏。

解脉令人腰痛,痛引肩,目䀮䀮然,时遗溲。刺解脉,在膝筋肉分间郄外廉之横脉出血,血变而止。解脉令人腰痛如引带,常如折腰状,善恐。刺解脉,在郄中结络如黍米,刺之血射以黑,见赤血而已。

同阴之脉令人腰痛,痛如小锤(《太素》作针)居其中,怫然肿。刺同阴之脉,在外踝上绝骨之端,为三痏。

阳维之脉令人腰痛,痛上怫然肿。刺阳维之脉,脉与太阳合腨下间,去地一尺所。

衡络之脉令人腰痛,不可以俛仰,仰则恐仆,得之举重伤腰,衡络绝,恶血归之。刺之在郄阳筋之间,上郄数寸,衡居,为二痏,出血。

会阴之脉令人腰痛,痛上漯漯然汗出,汗干令人欲饮,饮已欲走。刺直阳之脉上三痏,在蹻上郄下五寸横居,视其盛者出血。

飞阳之脉令人腰痛,痛上怫怫然,甚则悲以恐。刺飞阳之脉,在内踝上五寸,少阴之前与阴维之会。

昌阳之脉令人腰痛,痛引膺,目䀮䀮然,甚则反折,舌卷不能言。刺内筋为二痏,在内踝上大筋前太阴后,上踝二寸所。

散脉令人腰痛而热,热甚生烦,腰下如有横木居其中,甚则遗溲。刺散脉在膝前骨肉分间,络外廉,束脉,为三痏。

肉里之脉令人腰痛,不可以咳,咳则筋缩急。刺肉里之脉为二痏,在太阳之外,少阳绝骨之后。

腰痛,侠脊,而痛至头几几然,目䀮䀮欲僵仆,刺足太阳郄中出血。腰痛,上寒,刺足太阳、阳明;上热,刺足厥阴;不可以俛仰,刺足少阳;中热而喘,刺足少阴,刺郄中出血。腰痛,上寒,不可顾,刺足阳明;上热,刺足太阴;中热而喘,刺足少阴。

大便难,刺足少阴。少腹满,刺足厥阴。如折,不可以俛仰,不可举,刺足太阳。引脊内廉,刺足少阴。

腰痛引少腹,控䏚,不可以仰。刺腰尻交者,两踝胂上。以月生死为痏数,发针立已。左取右,右取左。
\section{风论}
%%	42	%%
黄帝问曰:风之伤人也,或为寒热,或为热中,或为寒中,或为疠风,或为偏枯,或为风也。其病各异,其名不同。或内至五脏六腑,不知其解,愿闻其说。

岐伯对曰:风气藏于皮肤之间,内不得通,外不得泄。风者,善行而数变,腠理开则洒然寒,闭则热而闷。其寒也则衰食饮,其热也则消肌肉,故使人怢栗而不能食,名曰寒热。

风气与阳明入胃,循脉而上至目内眦,其人肥,则风气不得外泄,则为热中而目黄;人瘦,则外泄而寒,则为寒中而泣出。

风气与太阳俱入,行诸脉俞,散于分肉之间,与卫气相干,其道不利,故使肌肉愤䐜而有疡;卫气有所凝而不行,故其肉有不仁也。

疠者有荣气热胕,其气不清,故使其鼻柱坏而色败,皮肤疡溃。风寒客于脉而不去,名曰疠风,或名曰寒热。

以春甲乙伤于风者为肝风,以夏丙丁伤于风者为心风,以季夏戊己伤于邪者为脾风,以秋庚辛中于邪者为肺风,以冬壬癸中于邪者为肾风。

风中五脏六腑之俞,亦为脏府之风,各入其门户所中,则为偏风。风气循风府而上,则为脑风。风入系头,则为目风,眼寒。饮酒中风,则为漏风。入房,汗出中风,则为内风。新沐中风,则为首风。久风入中,则为肠风,飧泄。外在腠理,则为泄风。

故风者,百病之长也,至其变化,乃为他病也,无常方,然致有风气也。

帝曰:五脏风之形状不同者何?愿闻其诊,及其病能。

岐伯曰:肺风之状,多汗恶风,色皏然白,时咳短气,昼日则差,暮则甚,诊在眉上,其色白。心风之状,多汗恶风,焦绝善怒吓,赤色,病甚则言不可快,诊在口,其色赤。肝风之状,多汗恶风,善悲,色微苍,嗌干,善怒,时憎女子,诊在目下,其色青。脾风之状,多汗恶风,身体怠堕,四肢不欲动,色薄微黄,不嗜食,诊在鼻上,其色黄。肾风之状,多汗恶风,面痝然浮肿,脊痛不能正立,其色炲,隐曲不利,诊在肌上,其色黑。

胃风之状,颈多汗,恶风,食饮不下,鬲塞不通,腹善满,失衣则䐜胀,食寒则泄,诊形瘦而腹大。

首风之状,头面多汗,恶风,当先风一日则病甚,头痛不可以出内,至其风日,则病少愈。漏风之状,或多汗,常不可单衣,食则汗出,甚则身汗,喘息,恶风,衣常濡,口干,善渴,不能劳事。泄风之状,多汗,汗出泄衣上,口中干,上渍其风,不能劳事,身体尽痛,则寒。

帝曰:善。
\section{痹论}
%%	43	%%
黄帝问曰:痹之安生?

岐伯对曰:风寒湿三气杂至,合而为痹也。其风气胜者为行痹,寒气胜者为痛痹,湿气胜者为着痹也。

帝曰:其有五者何也?

岐伯曰:以冬遇此者为骨痹,以春遇此者为筋痹,以夏遇此者为脉痹,以至阴遇此者为肌痹,以秋遇此者为皮痹。

帝曰:内舍五脏六腑,何气使然?

岐伯曰:五脏皆有合,病久而不去者,内舍于其合也。故骨痹不已,复感于邪,内舍于肾;筋痹不已,复感于邪,内舍于肝;脉痹不已,复感于邪,内舍于心;肌痹不已,复感于邪,内舍于脾;皮痹不已,复感于邪,内舍于肺。所谓痹者,各以其时,重感于风寒湿之气也。

凡痹之客五脏者:肺痹者,烦满喘而呕;心痹者,脉不通,烦则心下鼓,暴上气而喘,嗌干善噫,厥气上则恐;肝痹者,夜卧则惊,多饮,数小便,上为引如怀;肾痹者,善胀,尻以代踵,脊以代头;脾痹者,四肢解堕,发咳呕汁,上为大塞。

肠痹者,数饮而出不得,中气喘争,时发飧泄。胞痹者,少腹膀胱按之内痛,若沃以汤,涩于小便,上为清涕。

阴气者,静则神藏,躁则消亡。饮食自倍,肠胃乃伤。

淫气喘息,痹聚在肺;淫气忧思,痹聚在心;淫气遗溺,痹聚在肾;淫气乏竭,痹聚在肝;淫气肌绝,痹聚在脾。诸痹不已,亦益内也。其风气胜者,其人易已也。

帝曰:痹,其时有死者,或疼久者,或易已者,其故何也?

岐伯曰:其入脏者死,其留连筋骨间者疼久,其留皮肤间者易已。

帝曰:其客于六腑者何也?

岐伯曰:此亦其食饮居处,为其病本也。六腑亦各有俞,风寒湿气中其俞,而食饮应之,循俞而入,各舍其腑也。

帝曰:以针治之奈何?

岐伯曰:五脏有俞,六腑有合,循脉之分,各有所发,各随其过,则病瘳也。

帝曰:荣卫之气,亦令人痹乎?

岐伯曰:荣者,水谷之精气也,和调于五脏,洒陈于六腑,乃能入于脉也。故循脉上下,贯五脏,络六腑也。卫者,水谷之悍气也,其气慓疾滑利,不能入于脉也。故循皮肤之中,分肉之间,熏于肓膜,散于胸腹。逆其气则病,从其气则愈,不与风寒湿气合,故不为痹。

帝曰:善。痹,或痛,或不痛,或不仁,或寒,或热,或燥,或湿,其故何也?

岐伯曰:痛者寒气多也,有寒故痛也。其不痛不仁者,病久入深,荣卫之行涩,经络时疏,故不通;皮肤不营,故为不仁。其寒者,阳气少,阴气多,与病相益,故寒也。其热者,阳气多,阴气少,病气胜,阳遭阴,故为痹热。其多汗而濡者,此其逢湿甚也。阳气少,阴气盛,两气相感,故汗出而濡也。

帝曰:夫痹之为病,不痛何也?

岐伯曰:痹在于骨则重,在于脉则血凝而不流,在于筋则屈不伸,在于肉则不仁,在于皮则寒。故具此五者,则不痛也。凡痹之类,逢寒则虫(《甲乙经》作急),逢热则纵。

帝曰:善。
\section{痿论}
%%	44	%%
黄帝问曰:五脏使人痿,何也?

岐伯对曰:肺主身之皮毛,心主身之血脉,肝主身之筋膜,脾主身之肌肉,肾主身之骨髓。故肺热叶焦,则皮毛虚弱,急薄,着则生痿躄也。心气热,则下脉厥而上,上则下脉虚,虚则生脉痿,枢折挈,胫纵,而不任地也。肝气热,则胆泄口苦,筋膜干,筋膜干则筋急而挛,发为筋痿。脾气热,则胃干而渴,肌肉不仁,发为肉痿。肾气热,则腰脊不举,骨枯而髓减,发为骨痿。

帝曰:何以得之?

岐伯曰:肺者,脏之长也,为心之盖也,有所失亡,所求不得,则发肺鸣,鸣则肺热叶焦。故曰:“五脏因肺热叶焦,发为痿躄”。此之谓也。

悲哀太甚则胞络绝,胞络绝则阳气内动,发则心下崩,数溲血也。故《本病》曰:“大经空虚,发为肌痹,传为脉痿。”

思想无穷,所愿不得,意淫于外,入房太甚,宗筋弛纵,发为筋痿,及为白淫。故《下经》曰:“筋痿者,生于肝,使内也。”

有渐于湿,以水为事,若有所留,居处相湿,肌肉濡渍,痹而不仁,发为肉痿。故《下经》曰:“肉痿者,得之湿地也。”

有所远行劳倦,逢大热而渴,渴则阳气内伐,内伐则热舍于肾。肾者,水脏也,今水不胜火,则骨枯而髓虚,故足不任身,发为骨痿。故《下经》曰:“骨痿者,生于大热也。”

帝曰:何以别之?

岐伯曰:肺热者,色白而毛败;心热者,色赤而络脉溢;肝热者,色苍而爪枯;脾热者,色黄而肉蠕动;肾热者,色黑而齿槁。

帝曰:如夫子言可矣。论言:“治痿者,独取阳明。”何也?

岐伯曰:阳明者,五脏六腑之海,主润宗筋,宗筋主束骨而利机关也。冲脉者,经脉之海也,主渗灌溪谷,与阳明合于宗筋。阴阳摠宗筋之会,会于气街,而阳明为之长,皆属于带脉,而络于督脉。故阳明虚,则宗筋纵,带脉不引,故足痿不用也。

帝曰:治之奈何?

岐伯曰:各补其荥而通其俞,调其虚实,和其逆顺,筋脉骨肉,各以其时受月,则病已矣。

帝曰:善。
\section{厥论}
%%	45	%%
黄帝问曰:厥之寒热者何也?

岐伯对曰:阳气衰于下则为寒厥,阴气衰于下则为热厥。

帝曰:热厥之为热也,必起于足下者,何也?

岐伯曰:阳气起(《甲乙经》作走)于足五趾之表,阴脉者,集于足下而聚于足心,故阳气胜则足下热也。

帝曰:寒厥之为寒也,必从五趾而上于膝者,何也?

岐伯曰:阴气起于五趾之里,集于膝下而聚于膝上,故阴气胜则从五趾至膝上寒,其寒也,不从外,皆从内也。

帝曰:寒厥何失而然也?

岐伯曰:前阴者,宗(《甲乙经》作众)筋之所聚,太阴、阳明之所合也。春夏则阳气多而阴气少,秋冬则阴气胜而阳气衰。此人者质壮,以秋冬夺于所用,下气上争,不能复,精气溢下,邪气因从之而上也。气因于中,阳气衰,不能渗营其经络,阳气日损,阴气独在,故手足为之寒也。

帝曰:热厥何如而然也?

岐伯曰:酒入于胃,则络脉满而经脉虚,脾主为胃行其津液者也。阴气虚则阳气入,阳气入则胃不和,胃不和则精气竭,精气竭则不营其四肢也。此人必数醉,若饱以入房,气聚于脾中不得散,酒气与谷气相薄,热盛于中,故热遍于身,内热而溺赤也。夫酒气盛而慓悍,肾气有衰,阳气独胜,故手足为之热也。

帝曰:厥,或令人腹满,或令人暴不知人,或至半日远至一日,乃知人者,何也?

岐伯曰:阴气盛于上则下虚,下虚则腹胀满;阳气盛于上,则下气重上而邪气逆,逆则阳气乱,阳气乱则不知人也。

帝曰:善。愿闻六经脉之厥状病能也。

岐伯曰:巨阳之厥,则肿首头重,足不能行,发为眴仆。阳明之厥,则癫疾欲走呼,腹满不得卧,面赤而热,妄见而妄言。少阳之厥,则暴聋颊肿而热,胁痛,䯒不可以运。太阴之厥,则腹满䐜胀,后不利,不欲食,食则呕,不得卧。少阴之厥,则口干溺赤,腹满心痛。厥阴之厥,则少腹肿痛,腹胀,泾溲不利,好卧屈膝,阴缩肿,䯒内热。盛则泻之,虚则补之,不盛不虚,以经取之。

太阴厥逆,䯒急挛,心痛引腹,治主病者。少阴厥逆,虚满呕变,下泄清,治主病者。厥阴厥逆,挛腰痛,虚满,前闭,谵言,治主病者。三阴俱逆,不得前后,使人手足寒,三日死。

太阳厥逆,僵仆,呕血,善衄,治主病者。少阳厥逆,机关不利,机关不利者,腰不可以行,项不可以顾,发肠痈不可治,惊者死。阳明厥逆,喘咳,身热,善惊,衄,呕血。

手太阴厥逆,虚满而咳,善呕沫,治主病者。手心主、少阴厥逆,心痛引喉,身热死,不可治。手太阳厥逆,耳聋,泣出,项不可以顾,腰不可以俛仰,治主病者。手阳明、少阳厥逆,发喉痹,嗌肿痓,治主病者。
\section{病能论}
%%	46	%%
黄帝问曰:人病胃脘痈者,诊当何如?

岐伯对曰:诊此者当候胃脉,其脉当沉细,沉细者气逆,逆者,人迎甚盛,甚盛则热。人迎者,胃脉也,逆而盛,则热聚于胃口而不行,故胃脘为痈也。

帝曰:善。人有卧而有所不安者,何也?

岐伯曰:藏有所伤,及精有所之寄则安,故人不能悬其病也。

帝曰:人之不得偃卧者,何也?

岐伯曰:肺者脏之盖也,肺气盛则脉大,脉大则不得偃卧,论在《奇恒阴阳》中。

帝曰:有病厥者,诊右脉沉而紧,左脉浮而迟,不然病主安在?

岐伯曰:冬诊之,右脉固当沉紧,此应四时。左脉浮而迟,此逆四时。在左当主病在肾,颇关在肺,当腰痛也。

帝曰:何以言之?

岐伯曰:少阴脉贯肾络肺,今得肺脉,肾为之病,故肾为腰痛之病也。

帝曰:善。有病颈痈者,或石治之,或针灸治之,而皆已,其真安在?

岐伯曰:此同名异等者也。夫痈气之息者,宜以针开除去之。夫气盛血聚者,宜石而泻之。此所谓同病异治也。

帝曰:有病怒狂者,此病安生?

岐伯曰:生于阳也。

帝曰:阳何以使人狂?

岐伯曰:阳气者,因暴折而难决,故善怒也,病名曰阳厥。

帝曰:何以知之?

岐伯曰:阳明者常动,巨阳少阳不动,不动而动,大疾,此其候也。

帝曰:治之奈何?

岐伯曰:夺(《甲乙经》、《太素》作衰)其食即已。夫食入于阴,长气于阳,故夺其食即已。使之服以生铁洛为饮(《甲乙经》“铁洛”作“铁落”,“为饮”作为“后饭”)。夫生铁洛者,下气疾也。

帝曰:善。有病身热解堕,汗出如浴,恶风少气,此为何病?

岐伯曰:病名曰酒风。

帝曰:治之奈何?

岐伯曰:以泽泻、术各十分,麋衔五分,合,以三指撮,为后饭。

所谓深之细者,其中手如针也,摩之切之。聚者坚也,博者大也。《上经》者,言气之通天也。《下经》者,言病之变化也。《金匮》者,决死生也。《揆度》者,切度之也。《奇恒》者,言奇病也。所谓奇者,使奇病不得以四时死也。恒者,得以四时死也。所谓揆者,方切求之也,言切求其脉理也。度者,得其病处,以四时度之也。
\section{奇病论}
%%	47	%%
黄帝问曰:人有重身,九月而瘖,此为何也?

岐伯对曰:胞之络脉绝也。

帝曰:何以言之?

岐伯曰:胞络者,系于肾,少阴之脉,贯肾、系舌本,故不能言。

曰:治之奈何?

岐伯曰:无治也,当十月复。《刺法》曰:“无损不足,益有余,以成其疹,然后调之(《甲乙经》、《太素》无此四字,为全元起注文,误书于此,当删去)。”所谓,无损不足者,身羸瘦,无用镵石也;无益其有余者,腹中有形而泄之,泄之则精出而病独擅中,故曰疹成也。

帝曰:病胁下满气逆,二三岁不已,是为何病?

岐伯曰:病名曰息积。此不妨于食,不可灸刺。积为导引服药,药不能独治也。

帝曰:人有身体髀股胻皆肿,环齐而痛,是为何病?

岐伯曰:病名曰伏梁,此风根也。其气溢于大肠而着于肓,肓之原在齐下,故环齐而痛也,不可动之,动之为水溺涩之病也。

帝曰:人有尺脉数甚,筋急而见,此为何病?

岐伯曰:此所谓疹筋。是人腹必急,白色黑色见,则病甚。

帝曰:人有病头痛,以数岁不已,此安得之?名为何病?

岐伯曰:当有所犯大寒,内至骨髓,髓者以脑为主,脑逆故令头痛、齿亦痛,病名曰厥逆。

帝曰:善。

帝曰:有病口甘者,病名为何?何以得之?

岐伯曰:此五气之溢也,名曰脾瘅。夫五味入口,藏于胃,脾为之行其精气,津液在脾,故令人口甘也。此肥美之所发也,此人必数食甘美而多肥也。肥者令人内热,甘者令人中满,故其气上溢,转为消渴。治之以兰,除陈气也。

帝曰:有病,口苦取阳陵泉(按全元起本及《太素》无“口苦取阳陵泉”六字)口苦者,病名为何?何以得之?

岐伯曰:病名曰胆瘅。夫肝者中之将也,取决于胆,咽为之使。此人者数谋虑不决,故胆虚,气上溢,而口为之苦。治之以胆募、俞,治在《阴阳十二官相使》中。

帝曰:有癃者,一日数十溲,此不足也。身热如炭,颈膺如格,人迎躁盛,喘息气逆,此有余也。太阴脉微细如发者,此不足也。其病安在?名为何病?

岐伯曰:病在太阴,其盛在胃,颇在肺,病名曰厥,死不治。此所谓得五有余二不足也。

帝曰:何谓五有余二不足?

岐伯曰:所谓五有余者,五病气之有余也;二不足者,亦病气之不足也。今外得五有余,内得二不足,此其身不表不里,亦正死明矣。

帝曰:人生而有病巅疾者,病名曰何?安所得之?

岐伯曰:病名为胎病。此得之在母腹中,时其母有所大惊,气上而不下,精气并居,故令子发为巅疾也。

帝曰:有病痝然如有水状,切其脉大紧,身无痛者,形不瘦,不能食,食少,名为何病?

岐伯曰:病生在肾,名为肾风。肾风而不能食,善惊,惊已心气痿者死。

帝曰:善。
\section{大奇论}
%%	48	%%
肝满、肾满、肺满皆实,即为肿。肺之雍,喘而两胠满;肝雍,两胠满,卧则惊,不得小便;肾雍,脚(《甲乙经》作胠)下至小腹满,胫有大小,髀胻大跛,易偏枯。

心脉满大,癎、瘛、筋挛。肝脉小急,癎、瘛、筋挛。肝脉骛暴,有所惊骇,脉不至若瘖,不治自己。肾脉小急,肝脉小急,心脉小急,不鼓皆为瘕。

肾肝并沉为石水,并浮为风水,并虚为死,并小弦欲惊。

肾脉大急沉,肝脉大急沉,皆为疝。心脉搏滑急为心疝,肺脉沉搏为肺疝。

三阳急为瘕,三阴急为疝。二阴急为癎厥,二阳急为惊。

脾脉外鼓,沉为肠澼,久自已。肝脉小缓为肠澼,易治。肾脉小搏沉为肠澼,下血,血温、身热者死。心肝澼亦下血,二脏同病者,可治;其脉小沉濇为肠澼,其身热者死,热见七日死。

胃脉沉鼓涩,胃外鼓大,心脉小坚急,皆鬲偏枯,男子发左,女子发右,不瘖舌转可治,三十日起;其从者瘖,三岁起;年不满二十者,三岁死。

脉至而搏,血衂,身热者死;脉来悬钩浮为常脉。脉至如喘,名曰暴厥。暴厥者,不知与人言,脉至如数,使人暴惊,三四日自已;脉至浮合,浮合如数,一息十至以上,是经气予不足也,微见九、十日死。

脉至如火薪然,是心精之予夺也,草干而死。脉至如散叶,是肝气予虚也,木叶落而死。脉至如省客,省客者脉塞而鼓,是肾气予不足也,悬去枣华而死。

脉至如丸泥,是胃精予不足也,榆荚落而死。脉至如横格,是胆气予不足也,禾熟而死。脉至如弦缕,是胞精予不足也,病善言,下霜而死;不言,可治。

脉至如交漆,交漆者左右傍至也,微见三十日死。脉至如涌泉,浮鼓肌中,太阳气予不足也,少气味,韭英而死。脉至如颓土之状,按之不得,是肌气予不足也,五色先见黑,白垒发死。

脉至如悬雍,悬雍者,浮揣切之益大,是十二俞之予不足也,水凝而死。脉至如偃刀,偃刀者,浮之小急,按之坚大急,五脏菀热,寒热独并于肾也,如此其人不得坐,立春而死。脉至如丸滑,不直手,不直手者,按之不可得也,是大肠气予不足也,枣叶生而死。脉至如华者,令人善恐,不欲坐卧,行立常听,是小肠气予不足也,季秋而死。
\section{脉解}
%%	49	%%
 太阳所谓肿腰脽痛者,正月太阳寅,寅,太阳也。正月阳气出在上而阴气盛,阳未得自次也,故肿腰脽痛也。病偏虚为跛者,正月阳气冻解,地气而出也。所谓偏虚者,冬寒颇有不足者,故偏虚为跛也。所谓强上引背者,阳气大上而争,故强上也。所谓耳鸣者,阳气万物盛上而跃,故耳鸣也。所谓甚则狂巅疾者,阳尽在上而阴气从下,下虚上实,故狂巅疾也。所谓浮为聋者,皆在气也。所谓入中为瘖者,阳盛已衰,故为瘖也。内夺而厥,则为瘖俳,此肾虚也。少阴不至者,厥也。

少阳所谓心胁痛者,言少阳盛也。盛者,心之所表也。九月阳气尽而阴气盛,故心胁痛也。所谓不可反侧者,阴气藏物也,物藏则不动,故不可反侧也。所谓甚则跃者,九月万物尽衰,草木毕落而堕,则气去阳而之阴,气盛而阳之下长,故谓跃。

阳明所谓洒洒振寒者,阳明者午也,五月盛阳之阴也,阳盛而阴气加之,故洒洒振寒也。所谓胫肿而股不收者,是五月盛阳之阴也,阳者衰于五月,而一阴气上,与阳始争,故胫肿而股不收也。所谓上喘而为水者,阴气下而复上,上则邪客于脏腑间,故为水也。所谓胷痛少气者,水气在脏腑也;水者阴炁也,阴炁在中,故胷痛少炁也。所谓甚则厥,恶人与火,闻木音则惕然而惊者,阳气与阴气相薄,水火相恶,故惕然而惊也。所谓欲独闭户牖而处者,阴阳相薄也,阳尽而阴盛,故欲独闭户牖而居。所谓病至则欲乘高而歌,弃衣而走者,阴阳复争而外并于阳,故使之弃衣而走也。所谓客孙脉则头痛、鼻鼽、腹肿者,阳明并于上,上者则其孙络太阴也,故头痛、鼻鼽、腹肿也。

太阴所谓病胀者,太阴子也,十一月万物气皆藏于中,故曰病胀。所谓上走心为噫者,阴盛而上走于阳明,阳明络属心,故曰上走心为噫也。所谓食则呕者,物盛满而上溢,故呕也。所谓得后与气则快然如衰者,十二月阴气下衰而阳气且出,故曰得后与气则快然如衰也。

少阴所谓腰痛者,少阴者,申也,十月(《太素》为七月)万物阳气皆伤,故腰痛也。所谓呕欬上气喘者,阴气在下,阳气在上,诸阳气浮,无所依从,故呕欬上气喘也。所谓色色不能久立久坐,起则目䀮䀮无所见者,万物阴阳不定未有主也,秋气始至,微霜始下,而方杀万物,阴阳内夺,故目䀮䀮无所见也。所谓少气善怒者,阳气不治,阳气不治,则阳气不得出,肝气当治而未得,故善怒,善怒者,名曰煎厥。所谓恐如人将捕之者,秋气万物未有毕去,阴气少,阳气入,阴阳相薄,故恐也。所谓恶闻食臭者,胃无气,故恶闻食臭也。所谓面黑如地色者,秋气内夺,故变于色也。所谓欬则有血者,阳脉伤也,阳气未盛于上而脉满,满则欬,故血见于鼻也。

厥阴所谓㿗疝,妇人少腹肿者,厥阴者辰也,三月阳中之阴,邪在中,故曰㿗疝少腹肿也。所谓腰脊痛不可以俛仰者,三月一振,荣华万物,一俛而不仰也。所谓㿗癃疝膺胀者,曰阴亦盛而脉胀不通,故曰㿗癃疝也。所谓甚则嗌干热中者,阴阳相薄而热,故嗌干也。
\section{刺要论}
%%	50	%%
黄帝问曰:愿闻刺要。

岐伯对曰:病有浮沉,刺有浅深,各至其理,无过其道,过之则内伤,不及则生外壅,壅则邪从之。浅深不得,反为大贼,内动五脏,后生大病。故曰:“病有在毫毛腠理者,有在皮肤者,有在肌肉者,有在脉者,有在筋者,有在骨者,有在髓者。”是故刺毫毛腠理无伤皮,皮伤则内动肺,肺动则秋病温疟,泝泝然寒栗。刺皮无伤肉,肉伤则内动脾,脾动则七十二日四季之月,病腹胀,烦不嗜食。刺肉无伤脉,脉伤则内动心,心动则夏病心痛。刺脉无伤筋,筋伤则内动肝,肝动则春病热而筋弛。刺筋无伤骨,骨伤则内动肾,肾动则冬病胀腰痛。刺骨无伤髓,髓伤则销铄胻酸,体解㑊然不去也。
\section{刺齐论}
%%	51	%%
黄帝问曰:愿闻刺浅深之分。

岐伯对曰:刺骨者无伤筋,刺筋者无伤肉,刺肉者无伤脉,刺脉者无伤皮,刺皮者无伤肉,刺肉者无伤筋,刺筋者无伤骨。

帝曰:余未知其所谓,愿闻其解。

岐伯曰:刺骨无伤筋者,针至筋而去,不及骨也。刺筋无伤肉者,至肉而去,不及筋也。刺肉无伤脉者,至脉而去,不及肉也。刺脉无伤皮者,至皮而去,不及脉也。所谓刺皮无伤肉者,病在皮中,针入皮中,无伤肉也。刺肉无伤筋者,过肉中筋也。刺筋无伤骨者,过筋中骨也。此之谓反也。
\section{刺禁论}
%%	52	%%
黄帝问曰:愿闻禁数。

岐伯对曰:脏有要害,不可不察。肝生于左,肺藏于右,心部于表,肾治于里,脾为之使,胃为之市。鬲肓之上,中有父母,七节之旁,中有小心。从之有福,逆之有咎。

刺中心,一日死,其动为噫。刺中肝,五日死,其动为语。刺中肾,六日死,其动为嚏。刺中肺,三日死,其动为欬。刺中脾,十日死,其动为吞。刺中胆,一日半死,其动为呕。

刺跗上中大脉,血出不止死。刺面中溜脉,不幸为盲。刺头中脑户,入脑,立死。刺舌下中脉太过,血出不止为瘖。刺足下布络中脉,血不出为肿。刺郄中大脉,令人仆,脱色。刺气街中脉,血不出为肿鼠仆。刺脊间中髓,为伛。刺乳上中乳房,为肿,根蚀。刺缺盆中内陷,气泄,令人喘欬逆。刺手鱼腹内陷,为肿。

无刺大醉,令人气乱。无刺大怒,令人气逆。无刺大劳人,无刺新饱人,无刺大饥人,无刺大渴人,无刺大惊人。

刺阴股中大脉,血出不止,死。刺客主人内陷中脉,为内漏,为聋。刺膝髌出液,为跛。刺臂太阴脉,出血多,立死。刺足少阴脉,重虚出血,为舌难以言。刺膺中陷中肺,为喘逆仰息。刺肘中内陷,气归之,为不屈伸。刺阴股下三寸内陷,令人遗溺。刺腋下胁间内陷,令人欬。刺少腹中膀胱,溺出,令人少腹满。刺腨肠内陷,为肿。刺匡上陷骨中脉,为漏为盲。刺关节中液出,不得屈伸。
\section{刺志论}
%%	53	%%
黄帝问曰:愿闻虚实之要。

岐伯对曰:气实形实,气虚形虚,此其常也,反此者病。谷盛气盛,谷虚气虚,此其常也,反此者病。脉实血实,脉虚血虚,此其常也,反此者病。

帝曰:如何而反?

岐伯曰:气盛身寒,此谓反也。气虚身热,此谓反也。(读注:此二句,各本多只有后一句。《甲乙经》二句并作:“气盛身寒,气虚身热。”明•马莳增补为二句。)谷入多而气少,此谓反也。谷不入而气多,此谓反也。脉盛血少,此谓反也。脉少血多,此谓反也。

气盛身寒,得之伤寒。气虚身热,得之伤暑。谷入多而气少者,得之有所脱血,湿居下也。谷入少而气多者,邪在胃及与肺也。脉小血多者,饮中热也。脉大血少者,脉有风气,水浆不入。此之谓也。

夫实者,气入也;虚者,气出也。气实者,热也;气虚者,寒也。入实者,左手开针空也;入虚者,左手闭针空也。
\section{针解}
%%	54	%%
黄帝问曰:愿闻九针之解,虚实之道。

岐伯对曰:刺虚则实之者,针下热也,气实乃热也。满而泄之者,针下寒也,气虚乃寒也。菀陈则除之者,出恶血也。邪盛则虚之者,出针勿按。

徐而疾则实者,徐出针而疾按之。疾而徐则虚者,疾出针而徐按之。言实与虚者,寒温气多少也。若无若有者,疾不可知也。察后与先者,知病先后也。为虚与实者,工勿失其法。若得若失者,离其法也。

虚实之要,九针最妙者,为其各有所宜也。补泻之时者(读注:《甲乙经》作:“补泻之时,以针为之者,”),与气开阖相合也。九针之名各不同形者,针穷其所当补泻也。

刺实须其虚者,留针,阴气隆至,乃去针也。刺虚须其实者,阳气隆至,针下热乃去针也。经气已至,慎守勿失者,勿变更也。深浅在志者,知病之内外也。近远如一者,浅深其候等也。如临深渊者,不敢堕也。手如握虎者,欲其壮也。神无营于众物者,静志观病人,无左右视也。义无邪下者,欲端以正也。必正其神者,欲瞻病人目,制其神,令气易行也。

所谓三里者,下膝三寸也。所谓跗上者,举膝分易见也。巨虚者,蹻足胻独陷者。下廉者,陷下者也。

帝曰:余闻九针,上应天地四时阴阳,愿闻其方,令可传于后世,以为常也。

岐伯曰:夫一天、二地、三人、四时、五音、六律、七星、八风、九野,身形亦应之,针各有所宜,故曰九针。人皮应天,人肉应地,人脉应人,人筋应时,人声应音,人阴阳合气应律,人齿面目应星,人出入气应风,人九窍三百六十五络应野。

故一针皮,二针肉,三针脉,四针筋,五针骨,六针调阴阳,七针益精,八针除风,九针通九窍,除三百六十五节气,此之谓各有所主也。人心意应八风,人气应天,人发齿耳目五声应五音六律,人阴阳脉血气应地,人肝目应之九。
\section{长刺节论}
%%	55	%%
刺家不诊,听病者言。

在头,头疾痛,为藏针之(按全元起本云:“为针之。”无“藏”字),刺至骨病已,上无伤骨肉及皮,皮者道也。

阴刺,入一旁四处(按《甲乙经》,阳刺者,正内一,傍内四;阴刺者,左右卒刺之。此阴刺疑是阳刺也),治寒热。深专者,刺大脏,迫脏刺背,背俞也。刺之迫脏,脏会,腹中寒热去而止。与刺之要,发针而浅出血。

治腐(按全元起本及《甲乙经》“腐”作“痈”)肿者,刺腐上,视痈小大深浅刺。刺大者多血,小者深之,必端内针为故止。

病在少腹有积,刺皮𩩻以下,至少腹而止。刺侠脊两旁四椎间,刺两髂髎、季胁肋间,导腹中气热下已。

病在少腹,腹痛,不得大小便,病名曰疝,得之寒。刺少腹两股间,刺腰髁骨间,刺而多之,尽炅病已。

病在筋,筋挛节痛,不可以行,名曰筋痹。刺筋上为故,刺分肉间,不可中骨也。病起筋炅,病已止。

病在肌肤,肌肤尽痛,名曰肌痹,伤于寒湿。刺大分小分,多发针而深之,以热为故,无伤筋骨,伤筋骨,痈发若变。诸分尽热,病已止。

病在骨,骨重不可举,骨髓酸痛,寒气至,名曰骨痹。深者刺,无伤脉肉为故。其道大分小分,骨热病已止。

病在诸阳脉,且寒且热,诸分且寒且热,名曰狂。刺之虚脉,视分尽热,病已止。

病初发,岁一发;不治,月一发;不治,月四五发,名曰癫病。刺诸分诸脉。其无寒者,以针调之,病已止。

病风且寒且热,炅汗出,一日数过,先刺诸分理络脉。汗出且寒且热,三日一刺,百日而已。

病大风,骨节重,须眉堕,名曰大风,刺肌肉为故,汗出百日;刺骨髓,汗出百日。凡二百日,须眉生而止针。
\section{皮部论}
%%	56	%%
黄帝问曰:余闻皮有分部,脉有经纪,筋有结络,骨有度量,其所生病各异,别其分部,左右上下,阴阳所在,病之始终,愿闻其道。

岐伯对曰:欲知皮部以经脉为纪者,诸经皆然。

阳明之阳,名曰害蜚。上下同法,视其部中有浮络者,皆阳明之络也。其色多青则痛,多黑则痹,黄赤则热,多白则寒,五色皆见,则寒热也。络盛则入客于经,阳主外,阴主内。

少阳之阳,名曰枢持。上下同法,视其部中有浮络者,皆少阳之络也。络盛则入客于经,故在阳者主内,在阴者主出,以渗于内。诸经皆然。

太阳之阳,名曰关枢。上下同法,视其部中有浮络者,皆太阳之络也。络盛则入客于经。

少阴之阴,名曰枢儒。上下同法,视其部中有浮络者,皆少阴之络也。络盛则入客于经。其入经也,从阳部注于经;其出者,从阴内注于骨。

心主之阴,名曰害肩。上下同法,视其部中有浮络者,皆心主之络也。络盛则入客于经。

太阴之阴,名曰关蛰。上下同法,视其部中有浮络者,皆太阴之络也。络盛则入客于经。

凡十二经络脉者,皮之部也。是故百病之始生也,必先于皮毛。邪中之则腠理开,开则入客于络脉,留而不去,传入于经,留而不去,传入于腑,廪于肠胃。邪之始入于皮也,泝然起毫毛,开腠理,其入于络也,则络脉盛色变;其入客于经也,则感虚乃陷下。其留于筋骨之间,寒多则筋挛骨痛;热多则筋弛骨消,肉烁䐃破,毛直而败。

帝曰:夫子言皮之十二部,其生病皆何如?

岐伯曰:皮者,脉之部也。邪客于皮则腠理开,开则邪入客于络脉,络脉满则注于经脉,经脉满则入舍于腑脏也。故皮者有分部,不与而生大病也。

帝曰:善。
\section{经络论}
%%	57	%%
黄帝问曰:夫络脉之见也,其五色各异,青黄赤白黑不同,其故何也?

岐伯对曰:经有常色,而络无常变也。

帝曰:经之常色何如?

岐伯曰:心赤,肺白,肝青,脾黄,肾黑,皆亦应其经脉之色也。

帝曰:络之阴阳,亦应其经乎?

岐伯曰:阴络之色应其经,阳络之色变无常,随四时而行也。寒多则凝泣,凝泣则青黑;热多则淖泽,淖泽则黄赤。此皆常色,谓之无病(读注:清•张志聪认为:此八字当在“随四时而行也”之下,误脱在此)。五色具见者,谓之寒热。

帝曰:善。
\section{气穴论}
%%	58	%%
黄帝问曰:余闻气穴三百六十五以应一岁,未知其所,愿卒闻之。

岐伯稽首再拜对曰:窘乎哉问也!其非圣帝,孰能穷其道焉!因请溢意,尽言其处。

帝捧手逡巡而却曰:夫子之开余道也,目未见其处,耳末闻其数,而目以明,耳以聪矣。

岐伯曰:此所谓:“圣人易语,良马易御”也。

帝曰:余非圣人之易语也,世言真数开人意,今余所访问者真数,发蒙解惑,未足以论也。然余愿闻夫子溢志尽言其处,令解其意,请藏之金匮,不敢复出。

岐伯再拜而起曰:臣请言之。背与心相控而痛,所治天突与十椎及上纪。上纪者胃脘也,下纪者关元也。背胷邪系阴阳左右,如此其病前后痛濇,胷胁痛而不得息,不得卧,上气短气,偏痛,脉满起,斜出尻脉,络胷胁,支心贯鬲,上肩加天突,斜下肩,交十椎下。

脏俞五十穴。腑俞七十二穴。热俞五十九穴。水俞五十七穴。头上五行,行五,五五二十五穴。中𦛗两旁各五,凡十穴。大椎上两旁各一,凡二穴。目瞳子、浮白二穴。两髀厌分中二穴。犊鼻二穴。耳中多所闻二穴。眉本二穴。完骨二穴。项中央一穴。枕骨二穴。上关二穴。大迎二穴。下关二穴。天柱二穴。巨虚上下廉四穴。曲牙二穴。天突一穴。天府二穴。天牖二穴。扶突二穴。天窻二穴。肩解二穴。关元一穴。委阳二穴。肩贞二穴。瘖门一穴。齐一穴。胷俞十二穴。背俞二穴。膺俞十二穴。分肉二穴。踝上横二穴。阴阳蹻四穴。水俞在诸分,热俞在气穴,寒热俞在两骸厌中二穴。大禁二十五,在天府下五寸。凡三百六十五穴,针之所由行也。

黄帝曰:余已知气穴之处,游针之居,愿闻孙络溪谷,亦有所应乎?

岐伯曰:孙络三百六十五穴会,亦以应一岁,以溢奇邪,以通荣卫。荣卫稽留,卫散荣溢,气竭血着,外为发热,内为少气。疾泻无怠,以通荣卫。见而泻之,无问所会。

帝曰:善。愿闻溪谷之会也。

岐伯曰:肉之大会为谷,肉之小会为溪,肉分之间,溪谷之会,以行荣卫,以会大气。邪溢气壅,脉热肉败,荣卫不行,必将为脓。内销骨髓,外破大腘(别本作䐃)。留于节凑,必将为败。积寒留舍,荣卫不居。卷肉缩筋(全元起本作“寒肉缩筋”),肋肘不得伸,内为骨痹,外为不仁,命曰不足,大寒留于溪谷也。溪谷三百六十五穴会,亦应一岁。其小痹淫溢,循脉往来,微针所及,与法相同。

帝乃辟左右而起,再拜曰:今日发蒙解惑,藏之金匮,不敢复出。乃藏之金兰之室,署曰:“气穴所在”。

岐伯曰:孙络之脉别经者,其血盛而当泻者,亦三百六十五脉,并注于络,传注十二脉络,非独十四脉络也。内解泻于中者,十脉。
\section{气府论}
%%	59	%%
足太阳脉气所发者,七十八穴:两眉头各一。入发至项三寸半(此句按别本云:“入发至顶三寸。”当为是),旁五,相去三寸,其浮气在皮中者凡五行,行五,五五二十五。项中大筋两旁各一。风府两旁各一。侠背以下至尻尾二十一节十五间各一,五脏之俞各五,六腑之俞各六。委中以下至足小指旁各六俞。

足少阳脉气所发者,六十二穴:两角上各二。直目上发际内各五。耳前角上各一。耳前角下各一。锐发下各一。客主人各一。耳后陷中各一。下关各一。耳下牙车之后各一。缺盆各一。腋下三寸,胁下至胠,八间各一。髀枢中旁各一。膝以下至足小指次指各六俞。

足阳明脉气所发者,六十八穴:额颅发际旁各三。面鼽(读注:当作頄,本篇下同)骨空各一。大迎之骨空各一。人迎各一。缺盆外骨空各一。膺中骨间各一。侠鸠尾之外,当乳下三寸,侠胃脘各五;侠齐广三寸各三;下齐二寸侠之各三。气街动脉各一。伏菟上各一。三里以下至足中指各八俞,分之所在穴空。

手太阳脉气所发者,三十六穴:目内眦各一。目外各一。鼽骨下各一。耳郭上各一。耳中各一。巨骨穴各一。曲腋上骨穴各一。柱骨上陷者各一。上天窻四寸各一。肩解各一。肩解下三寸各一。肘以下至手小指本各六俞。

手阳明脉气所发者,二十二穴:鼻空外廉项上各二。大迎骨空各一。柱骨之会各一。髃骨之会各一。肘以下至手大指次指本各六俞。

手少阳脉气所发者,三十二穴:鼽骨下各一。眉后各一。角上各一。下完骨后各一。项中足太阳之前各一。侠扶突各一。肩贞各一。肩贞下三寸分间各一。肘以下至手小指次指本各六俞。

督脉气所发者,二十八穴:项中央二。发际后中八。面中三。大椎以下至尻尾及旁十五穴。至骶下凡二十一节,脊椎法也。

任脉之气所发者,二十八穴:喉中央二。膺中骨陷中各一。鸠尾下三寸,胃脘五寸,胃脘以下至横骨六寸半一,腹脉法也。下阴别一。目下各一。下唇一。龂交一。

冲脉气所发者,二十二穴:侠鸠尾外各半寸,至齐寸一;侠脐下旁各五分,至横骨寸一,腹脉法也。

足少阴舌下。厥阴毛中急脉各一。手少阴各一。

阴阳蹻各一。

手足诸鱼际脉气所发者。

凡三百六十五穴也。
\section{骨空论}
%%	60	%%
黄帝问曰:余闻风者百病之始也,以针治之奈何?

岐伯对曰:风从外入,令人振寒,汗出头痛,身重恶寒,治在风府,调其阴阳,不足则补,有余则泻。

大风,颈项痛,刺风府,风府在上椎。大风汗出,灸譩嘻,譩嘻在背下侠脊旁三寸所,厌之令病者呼譩嘻,譩嘻应手。从风憎风,刺眉头。失枕在肩上横骨间。折使揄臂,齐肘正,灸脊中。䏚络季胁引少腹而痛胀,刺譩嘻。腰痛不可以转摇,急引阴卵,刺八髎与痛上,八髎在腰尻分间。

鼠瘘寒热,还刺寒府,寒府在附膝外解营。取膝上外者,使之拜;取足心者,使之跪。

任脉者,起于中极之下,以上毛际,循腹里,上关元,至咽喉,上颐,循面,入目。冲脉者,起于气街,并少阴之经,侠齐上行,至胷中而散。任脉为病,男子内结、七疝;女子带下、瘕聚。冲脉为病,逆气里急。

督脉为病,脊强反折。督脉者,起于少腹以下骨中央。女子入系廷孔,其孔溺孔之端也,其络循阴器,合篡间,绕篡后,别绕臀,至少阴,与巨阳中络者,合少阴上股内后廉,贯脊属肾。与太阳起于目内眦,上额交巅上,入络脑,还出别下项,循肩膊内,侠脊抵腰中,入循膂络肾。其男子循茎下至篡,与女子等。其少腹直上者,贯齐中央,上贯心,入喉,上颐环唇,上系两目之下中央。此生病,从少腹上冲心而痛,不得前后,为冲疝。其女子不孕,癃,痔,遗溺,嗌干。督脉生病治督脉,治在骨上,甚者在齐下营。

其上气有音者,治其喉中央,在缺盆中者。其病上冲喉者,治其渐,渐者,上侠颐也。

蹇膝伸不屈,治其楗。坐而膝痛,治其机。立而暑解,治其骸关。膝痛,痛及拇指,治其腘。坐而膝痛如物隐者,治其关。膝痛不可屈伸,治其背内。连胻若折,治阳明中俞髎,若别,治巨阳少阴荣。淫泺胫酸,不能久立,治少阳之维,在外上五寸。

辅骨上横骨下为楗,侠髋为机,膝解为骸关,侠膝之骨为连骸,骸下为辅,辅上为腘,腘上为关,头横骨为枕。

水俞五十七穴者,尻上五行,行五。伏菟上两行,行五,左右各一行,行五。踝上各一行,行六穴。

髓空,在脑后三分,在颅际锐骨之下;一在龂基下;一在项后中,复骨下;一在脊骨上空,在风府上。脊骨下空,在尻骨下空。数髓空,在面侠鼻,或骨空在口下,当两肩。两髆骨空,在髆中之阳。臂骨空在臂阳,去踝四寸,两骨空之间。股骨上空,在股阳,出上膝四寸。胻骨空,在辅骨之上端。股际骨空,在毛中动下。尻骨空,在髀骨之后,相去四寸。扁骨有渗理凑,无髓孔,易髓无空。

灸寒热之法:先灸项大椎,以年为壮数;次灸橛骨,以年为壮数。视背俞陷者灸之,举臂肩上陷者灸之,两季胁之间灸之,外踝上绝骨之端灸之,足小指次指间灸之,腨下陷脉灸之,外踝后灸之。缺盆骨上切之坚痛如筋者灸之,膺中陷骨间灸之,掌束骨下灸之,齐下关元三寸灸之,毛际动脉灸之,膝下三寸分间灸之,足阳明跗上动脉灸之,巅上一灸之。犬所囓之处灸之,三壮,即以犬伤病法灸之。凡当灸二十九处。伤食灸之不已者,必视其经之过于阳者,数刺其俞而药之。
\section{水热穴论}
%%	61	%%
黄帝问曰:少阴何以主肾?肾何以主水?

岐伯对曰:肾者,至阴也。至阴者,盛水也。肺者,太阴也。少阴者,冬脉也。故其本在肾,其末在肺,皆积水也。

帝曰:肾何以能聚水而生病?

岐伯曰:肾者,胃之关也,关门不利,故聚水而从其类也。上下溢于皮肤,故为胕肿。胕肿者,聚水而生病也。

帝曰:诸水皆生于肾乎?

岐伯曰:肾者牝脏也。地气上者,属于肾,而生水液也,故曰至阴。勇而劳甚,则肾汗出,肾汗出逢于风,内不得入于脏腑,外不得越于皮肤,客于玄府,行于皮里,传为胕肿,本之于肾,名曰风水。所谓玄府者,汗空也。

帝曰:水俞五十七处者,是何主也?

岐伯曰:肾俞五十七穴,积阴之所聚也,水所从出入也。尻上五行,行五者,此肾俞。故水病下为胕肿,大腹,上为喘呼不得卧者,标本俱病。故肺为喘呼,肾为水肿,肺为逆不得卧,分为相输俱受者,水气之所留也。伏菟上各二行,行五者,此肾之街也。三阴之所交结于脚也。踝上各一行,行六者,此肾脉之下行也,名曰太冲。凡五十七穴者,皆脏之阴络,水之所客也。

帝曰:春取络脉分肉,何也?

岐伯曰:春者木始治,肝气始生,肝气急,其风疾,经脉常深,其气少,不能深入,故取络脉分肉间。

帝曰:夏取盛经分腠,何也?

岐伯曰:夏者火始治,心气始长,脉瘦气弱,阳气留溢,热熏分腠,内至于经,故取盛经分腠。绝肤而病去者,邪居浅也。所谓盛经者,阳脉也。

帝曰:秋取经俞,何也?

岐伯曰:秋者金始治,肺将收杀,金将胜火,阳气在合,阴气初胜,湿气及体,阴气未盛,未能深入,故取俞以泻阴邪,取合以虚阳邪,阳气始衰,故取于合。

帝曰:冬取井荥,何也?

岐伯曰:冬者水始治,肾方闭,阳经衰少,阴气坚盛,巨阳伏沉,阳脉乃去,故取井以下阴逆,取荥以实阳气,故曰:“冬取井荥,春不鼽衄。”此之谓也。

帝曰:夫子言治热病五十九俞,余论其意,未能领别其处,愿闻其处,因闻其意。

岐伯曰:头上五行,行五者,以越诸阳之热逆也。大杼、膺俞、缺盆、背俞,此八者,以泻胷中之热也。气街、三里、巨虚上下廉,此八者,以泻胃中之热也。云门、髃骨、委中、髓空,此八者,以泻四肢之热也。五脏俞旁五,此十者,以泻五脏之热也。凡此五十九穴者,皆热之左右也。

帝曰:人伤于寒而传为热,何也?

岐伯曰:夫寒盛则生热也。
\section{调经论}
%%	62	%%
黄帝问曰:余闻刺法言有余泻之,不足补之。何谓有余,何谓不足?

岐伯对曰:有余有五,不足亦有五。帝欲何问?

帝曰:愿尽闻之。

岐伯曰:神有余有不足,气有余有不足,血有余有不足,形有余有不足,志有余有不足。凡此十者,其气不等也。

帝曰:人有精气、津液,四肢、九窍、五脏十六部,三百六十五节,乃生百病。百病之生,皆有虚实。今夫子乃言有余有五,不足亦有五,何以生之乎?

岐伯曰:皆生于五脏也。夫心藏神,肺藏气,肝藏血,脾藏肉,肾藏志,而此成形。志意通,内连骨髓,而成身形五脏。五脏之道,皆出于经隧,以行血气,血气不和,百病乃变化而生,是故守经隧焉。

帝曰:神有余不足何如?

岐伯曰:神有余则笑不休,神不足则悲。血气未并,五脏安定,邪客于形,洒淅起于毫毛,未入于经络也,故命曰神之微。

帝曰:补泻奈何?

岐伯曰:神有余则泻其小络之血,出血勿之深斥,无中其大经,神气乃平。神不足者,视其虚络,按而致之,刺而利之,无出其血,无泄其气,以通其经,神气乃平。

帝曰:刺微奈何?

岐伯曰:按摩勿释,着针勿斥,移气于不足,神气乃得复。

帝曰:善。气有余不足,奈何?

岐伯曰:气有余则喘欬上气,不足则息利少气。血气未并,五脏安定,皮肤微病,命曰白气微泄。

帝曰:补泻奈何?

岐伯曰:气有余则泻其经隧,无伤其经,无出其血,无泄其气;不足则补其经隧,无出其气。

帝曰:刺微奈何?

岐伯曰:按摩勿释,出针视之,曰我将深之,适人必革,精气自伏,邪气散乱,无所休息,气泄腠理,真气乃相得。

帝曰:善。血有余不足,奈何?

岐伯曰:血有余则怒,不足则恐。血气未并,五脏安定,孙络水溢,则经有留血。

帝曰:补泻奈何?

岐伯曰:血有余则泻其盛经,出其血;不足则视其虚经,内针其脉中,久留而视,脉大疾出其针,无令血泄。

帝曰:刺留血奈何?

岐伯曰:视其血络,刺出其血,无令恶血得入于经,以成其疾。

帝曰:善。形有余不足奈何?

岐伯曰:形有余则腹胀,泾溲不利,不足则四肢不用。血气未并,五脏安定,肌肉蠕动,命曰微风。

帝曰:补泻奈何?

岐伯曰:形有余则泻其阳经,不足则补其阳络。

帝曰:刺微奈何?

岐伯曰:取分肉间,无中其经,无伤其络,卫气得复,邪气乃索。

帝曰:善。志有余不足,奈何?

岐伯曰:志有余则腹胀飧泄,不足则厥。血气未并,五脏安定,骨节有动。

帝曰:补泻奈何?

岐伯曰:志有余则泻然筋血者,不足则补其复溜。

帝曰:刺未并奈何?

岐伯曰:即取之,无中其经,邪所乃能立虚。

帝曰:善。余已闻虚实之形,不知其何以生?

岐伯曰:气血以并,阴阳相倾,气乱于卫,血逆于经,血气离居,一实一虚。血并于阴,气并于阳,故为惊狂。血并于阳,气并于阴,乃为炅中。血并于上,气并于下,心烦惋善怒。血并于下,气并于上,乱而喜忘。

帝曰:血并于阴,气并于阳,如是血气离居,何者为实?何者为虚?

岐伯曰:血气者,喜温而恶寒,寒则泣不能流,温则消而去之。是故气之所并为血虚,血之所并为气虚。

帝曰:人之所有者血与气耳,今夫子乃言血并为虚,气并为虚,是无实乎?

岐伯曰:有者为实,无者为虚。故气并则无血,血并则无气。今血气相失,故为虚焉。络之与孙脉俱输于经,血与气并则为实焉。血之与气并走于上,则为大厥,厥则暴死。气复反则生,不反则死。

帝曰:实者何道从来?虚者何道从去?虚实之要,愿闻其故。

岐伯曰:夫阴与阳皆有俞会,阳注于阴,阴满之外,阴阳匀平,以充其形,九候若一,命曰平人。夫邪之生也,或生于阴,或生于阳。其生于阳者,得之风雨寒暑;其生于阴者,得之饮食居处,阴阳喜怒。

帝曰:风雨之伤人奈何?

岐伯曰:风雨之伤人也,先客于皮肤,传入于孙脉,孙脉满则传入于络脉,络脉满则输于大经脉。血气与邪并客于分腠之间,其脉坚大,故曰实。实者,外坚充满,不可按之,按之则痛。

帝曰:寒湿之伤人奈何?

岐伯曰:寒湿之中人也,皮肤不收,肌肉坚紧,营血泣,卫气去,故曰虚。虚者,聂辟气不足,按之则气足以温之,故快然而不痛。

帝曰:善。阴之生实奈何?

岐伯曰:喜怒不节,则阴气上逆,上逆则下虚,下虚则阳气走之,故曰实矣。

帝曰:阴之生虚奈何?

岐伯曰:喜则气下,悲则气消,消则脉虚空。因寒饮食,寒气熏满,则血泣气去,故曰虚矣。

帝曰:经言阳虚则外寒,阴虚则内热;阳盛则外热,阴盛则内寒,余已闻之矣,不知其所由然也。

岐伯曰:阳受气于上焦,以温皮肤分肉之间。今寒气在外,则上焦不通,上焦不通则寒气独留于外,故寒栗。

帝曰:阴虚生内热奈何?

岐伯曰:有所劳倦,形气衰少,谷气不盛,上焦不行,下脘不通,胃气热,热气熏胷中,故内热。

帝曰:阳盛生外热奈何?

岐伯曰:上焦不通利,则皮肤致密,腠理闭塞,玄府不通,卫气不得泄越,故外热。

帝曰:阴盛生内寒奈何?

岐伯曰:厥气上逆,寒气积于胷中而不泻,不泻则温气去,寒独留,则血凝泣,凝则脉不通,其脉盛大以濇,故中寒。

帝曰:阴与阳并,血气以并,病形以成,刺之奈何?

岐伯曰:刺此者,取之经隧,取血于营,取气于卫,用形哉,因四时多少高下。

帝曰:血气以并,病形以成,阴阳相倾,补泻奈何?

岐伯曰:泻实者,气盛乃内针,针与气俱内,以开其门,如利其户;针与气俱出,精气不伤,邪气乃下,外门不闭,以出其疾,摇大其道,如利其路,是谓大泻,必切而出,大气乃屈。

帝曰:补虚奈何?

岐伯曰:持针勿置,以定其意,候呼内针,气出针入,针空四塞,精无从去,方实而疾出针,气入针出,热不得还,闭塞其门,邪气布散,精气乃得存,动气候时,近气不失,远气乃来,是谓追之。

帝曰:夫子言虚实者有十,生于五脏,五脏五脉耳。夫十二经脉皆生其病,今夫子独言五脏。夫十二经脉者,皆络三百六十五节,节有病必被经脉,经脉之病皆有虚实,何以合之?

岐伯曰:五脏者,故得六腑与为表里,经络支节,各生虚实,其病所居,随而调之。病在脉,调之血;病在血,调之络;病在气,调之卫;病在肉,调之分肉;病在筋,调之筋;病在骨,调之骨。燔针劫刺其下及与急者。病在骨,焠针药熨;病不知所痛,两蹻为上;身形有痛,九候莫病,则缪刺之;痛在于左而右脉病者,巨刺之。必谨察其九候,针道备矣。
\section{缪刺论}
%%	63	%%
黄帝问曰:余闻缪刺,未得其意。何谓缪刺?

岐伯对曰:夫邪之客于形也,必先舍于皮毛;留而不去,入舍于孙脉;留而不去,入舍于络脉;留而不去,入舍于经脉;内连五脏,散于肠胃,阴阳俱感,五脏乃伤。此邪之从皮毛而入,极于五脏之次也,如此则治其经焉。今邪客于皮毛,入舍于孙络,留而不去,闭塞不通,不得入于经,流溢于大络,而生奇病也。夫邪客大络者,左注右,右注左,上下左右与经相干,而布于四末,其病无常处,不入于经俞,命曰缪刺。

帝曰:愿闻缪刺,以左取右,以右取左,奈何?其与巨刺,何以别之?

岐伯曰:邪客于经,左盛则右病,右盛则左病;亦有移易者,左病未已,而右脉先病,如此者必巨刺之,必中其经,非络脉也。故络病者,其痛与经脉缪处,故命曰缪刺。

帝曰:愿闻缪刺奈何?取之何如?

岐伯曰:邪客于足少阴之络,令人卒心痛、暴胀、胷胁支满,无积者,刺然骨之前出血,如食顷而已。不已,左取右,右取左。病新发者,取五日已。

邪客于手少阳之络,令人喉痹、舌卷、口干心烦、臂外廉痛、手不及头,刺手中指次指爪甲上,去端如韭叶,各一痏,壮者立已,老者有顷已。左取右,右取左。此新病,数日已。

邪客于足厥阴之络,令人卒疝暴痛,刺足大指爪甲上与肉交者各一痏,男子立已,女子有顷已。左取右,右取左。

邪客于足太阳之络,令人头项肩痛,刺足小指爪甲上与肉交者各一痏,立已。不已,刺外踝下三痏。左取右,右取左,如食顷已。

邪客于手阳明之络,令人气满胷中、喘息而支胠、胷中热,刺手大指次指爪甲上,去端如韭叶,各一痏。左取右,右取左,如食顷已。

邪客于臂掌之间,不可得屈,刺其踝后,先以指按之痛,乃刺之,以月死生为数,月生一日一痏,二日二痏,十五日十五痏,十六日十四痏。

邪客于足阳蹻之脉,令人目痛从内眦始,刺外踝之下半寸所各二痏。左刺右,右刺左,如行十里顷而已。

人有所堕坠,恶血留内、腹中满胀、不得前后,先饮利药。此上伤厥阴之脉,下伤少阴之络。刺足内踝之下,然骨之前血脉,出血,刺足跗上动脉。不已,刺三毛上各一痏,见血立已。左刺右,右刺左。善悲惊不乐,刺如右方。

邪客于手阳明之络,令人耳聋、时不闻音,刺手大指次指爪甲上,去端如韭叶,各一痏,立闻。不已,刺中指爪甲上与肉交者,立闻。其不时闻者,不可刺也。耳中生风者,亦刺之如此数。左刺右,右刺左。

凡痹往来,行无常处者,在分肉间痛而刺之,以月死生为数。用针者,随气盛衰,以为痏数,针过其日数则脱气,不及日数则气不泻。左刺右,右刺左,病已止。不已,复刺之如法。月生一日一痏,二日二痏,渐多之;十五日十五痏,十六日十四痏,渐少之。

邪客于足阳明之经,令人鼽衄、上齿寒,刺足中指次指爪甲上与肉交者各一痏。左刺右,右刺左。

邪客于足少阳之络,令人胁痛不得息、欬而汗出,刺足小指次指爪甲上与肉交者各一痏,不得息立已,汗出立止。欬者温衣饮食,一日已。左刺右,右刺左,病立已。不已,复刺如法。

邪客于足少阴之络,令人嗌痛不可内食、无故善怒、气上走贲上,刺足下中央之脉各三痏,凡六刺立已。左刺右,右刺左。嗌中肿不能内,唾时不能出唾者,刺然骨之前,出血立已。左刺右,右刺左。

邪客于足太阴之络,令人腰痛、引少腹控䏚、不可以仰息,刺腰尻之解,两胂之上,是腰俞。以月死生为痏数,发针立已。左刺右,右刺左。

邪客于足太阳之络,令人拘挛、背急、引胁而痛,刺之从项始,数脊椎侠脊,疾按之应手如痛,刺之旁三痏,立已。

邪客于足少阳之络,令人留于枢中痛、髀不可举,刺枢中以毫针,寒则久留针,以月死生为数,立已。

治诸经刺之,所过者不病,则缪刺之。耳聋,刺手阳明。不已,刺其通脉出耳前者。齿龋,刺手阳明。不已,刺其脉入齿中者,立已。

邪客于五脏之间,其病也,脉引而痛、时来时止。视其病,缪刺之于手足爪甲上,视其脉,出其血,间日一刺;一刺不已,五刺已。

缪传引上齿,齿唇寒痛,视其手背脉血者,去之,足阳明中指爪甲上一痏,手大指次指爪甲上各一痏,立已。左取右,右取左。

邪客于手足少阴、太阴;足阳明之络,此五络皆会于耳中,上络左角,五络俱竭,令人身脉皆动,而形无知也,其状若尸,或曰尸厥。刺其足大指内侧爪甲上,去端如韭叶;后刺足心;后刺足中指爪甲上各一痏;后刺手大指内侧,去端如韭叶;后刺手心主少阴锐骨之端,各一痏,立已。不已,以竹管吹其两耳,鬄其左角之发方一寸,燔治,饮以美酒一杯;不能饮者灌之,立已。

凡刺之数:先视其经脉,切而从之,审其虚实而调之;不调者,经刺之;有痛而经不病者,缪刺之;因视其皮部有血络者,尽取之。此缪刺之数也。
\section{四时刺逆从论}
%%	64	%%
厥阴有余病阴痹,不足病生热痹;滑则病狐疝风;濇则病少腹积气。少阴有余病皮痹、隐轸,不足病肺痹;滑则病肺风疝;濇则病积、溲血。太阴有余病肉痹、寒中,不足病脾痹;滑则病脾风疝;濇则病积、心腹时满。阳明有余病脉痹、身时热,不足病心痹;滑则病心风疝;濇则病积、时善惊。太阳有余病骨痹、身重,不足病肾痹;滑则病肾风疝;濇则病积、时善巅疾。少阳有余病筋痹、胁满,不足病肝痹;滑则病肝风疝;濇则病积、时筋急目痛。是故春气在经脉,夏气在孙络,长夏气在肌肉,秋气在皮肤,冬气在骨髓中。

帝曰:余愿闻其故。

岐伯曰:春者,天气始开,地气始泄,冻解冰释,水行经通,故人气在脉。夏者,经满气溢,入孙络受血,皮肤充实。长夏者,经络皆盛,内溢肌中。秋者,天气始收,腠理闭塞,皮肤引急。冬者,盖藏,血气在中,内着骨髓,通于五脏。是故邪气者,常随四时之气血而入客也。至其变化不可为度,然必从其经气,僻除其邪,除其邪则乱气不生。

帝曰:逆四时而生乱气奈何?

岐伯曰:春刺络脉,血气外溢,令人少气;春刺肌肉,血气环逆,令人上气;春刺筋骨,血气内着,令人腹胀。夏刺经脉,血气乃竭,令人解㑊;夏刺肌肉,血气内却,令人善恐;夏刺筋骨,血气上逆,令人善怒。秋刺经脉,血气上逆,令人善忘;秋刺络脉,气不外行,令人卧不欲动;秋刺筋骨,血气内散,令人寒栗。冬刺经脉,血气皆脱,令人目不明;冬刺络脉,内气外泄,留为大痹;冬刺肌肉,阳气竭绝,令人善忘。凡此四时刺者,大逆之病,不可不从也。反之,则生乱气相淫病焉。故刺不知四时之经,病之所生,以从为逆,正气内乱,与精相薄,必审九候,正气不乱,精气不转。

帝曰:善。刺五脏,中心一日死,其动为噫;中肝五日死,其动为语;中肺三日死,其动为欬;中肾六日死,其动为嚏欠;中脾十日死,其动为吞。刺伤人五脏必死,其动则依其脏之所变候,知其死也。
\section{标本病传论}
%%	65	%%
黄帝问曰:病有标本,刺有逆从,奈何?

岐伯对曰:凡刺之方,必别阴阳,前后相应,逆从得施,标本相移。故曰:“有其在标而求之于标,有其在本而求之于本,有其在本而求之于标,有其在标而求之于本”。故治有取标而得者,有取本而得者,有逆取而得者,有从取而得者。故知逆与从,正行无问,知标本者,万举万当,不知标本,是谓妄行。

夫阴阳逆从,标本之为道也,小而大,言一而知百病之害。少而多,浅而博,可以言一而知百也。以浅而知深,察近而知远。言标与本,易而勿及。治反为逆,治得为从。

先病而后逆者治其本,先逆而后病者治其本;先寒而后生病者治其本,先病而后生寒者治其本;先热而后生病者治其本,先热而后生中满者治其标;先病而后泄者治其本,先泄而后生他病者治其本。必且调之,乃治其他病。先病而后生中满者治其标,先中满而后烦心者治其本。人有客气,有同气。小大不利治其标,小大利治其本。病发而有余,本而标之,先治其本,后治其标;病发而不足,标而本之,先治其标,后治其本。谨察间甚,以意调之;间者并行,甚者独行。先小大不利而后生病者,治其本。

夫病传者,心病先心痛,一日先欬,三日胁支痛,五日闭塞不通,身痛体重,三日不已,死,冬夜半,夏日中。

肺病喘欬,三日而胁支满痛,一日身重体痛,五日而胀,十日不已,死,冬日入,夏日出。

肝病头目眩,胁支满,三日体重身痛,五日而胀,三日腰脊少腹痛、胫酸。三日不已,死,冬日入,夏早食。

脾病身痛体重,一日而胀,二日少腹腰脊痛、胫酸,三日背𦛗筋痛,小便闭。十日不已,死,冬人定,夏晏食。

肾病少腹腰脊痛、胻酸,三日背𦛗筋痛、小便闭,三日腹胀,三日两胁支痛,三日不已,死,冬大晨,夏晏晡。

胃病胀满,五日少腹腰脊痛、胻酸,三日背𦛗筋痛、小便闭,五日身体重,六日不已,死,冬夜半后,夏日昳。

膀胱病小便闭,五日少腹胀、腰脊痛、胻酸,一日腹胀,一日身体痛,二日不已,死,冬鸡鸣,夏下晡。

诸病以次是相传如是者,皆有死期,不可刺。间一脏止,及至三四脏者,乃可刺也。
\section{天元纪大论}
%%	66	%%
黄帝问曰:天有五行御五位,以生寒暑燥湿风;人有五脏化五气,以生喜怒忧思恐。论言五运相袭,而皆治之,终期之日,周而复始,予已知之矣。愿闻其与三阴三阳之候,奈何合之?

鬼臾区稽首再拜对曰:昭乎哉问也!夫五运阴阳者,天地之道也,万物之纲纪,变化之父母,生杀之本始,神明之府也,可不通乎!故物生谓之化,物极谓之变,阴阳不测谓之神,神用无方谓之圣。夫变化之为用也,在天为玄,在人为道,在地为化。化生五味,道生智,玄生神。神在天为风,在地为木;在天为热,在地为火;在天为湿,在地为土;在天为燥,在地为金;在天为寒,在地为水。故在天为气,在地成形,形气相感,而化生万物矣。然天地者,万物之上下也;左右者,阴阳之道路也;水火者,阴阳之征兆也;金木者,生成之终始也。气有多少,形有盛衰,上下相召而损益彰矣。

帝曰:愿闻五运之主时也何如?

鬼臾区曰:五气运行,各终期日,非独主时也。

帝曰:请闻其所谓也。

鬼臾区曰:臣积考《太始天元册》文曰:“太虚寥廓,肇基化元,万物资始,五运终天,布气真灵,总统坤元,九星悬朗,七曜周旋,曰阴曰阳,曰柔曰刚,幽显既位,寒暑弛张,生生化化,品物咸章。”臣斯十世,此之谓也。

帝曰:善。何谓气有多少,形有盛衰?

鬼臾区曰:阴阳之气,各有多少,故曰三阴三阳也。形有盛衰,谓五行之治,各有太过不及也。故其始也,有余而往,不足随之;不足而往,有余从之,知迎知随,气可与期。应天为天符,承岁为岁直,三合为治。

帝曰:上下相召,奈何?

鬼臾区曰:寒暑燥湿风火,天之阴阳也,三阴三阳上奉之。木火土金水火,地之阴阳也,生长化收藏下应之。天以阳生阴长,地以阳杀阴藏。天有阴阳,地亦有阴阳。木火土金水火,地之阴阳也,生长化收藏,故阳中有阴,阴中有阳。所以欲知天地之阴阳者,应天之气动而不息,故五岁而右迁;应地之气静而守位,故六期而环会。动静相召,上下相临,阴阳相错,而变由生也。

帝曰:上下周纪,其有数乎?

鬼臾区曰:天以六为节,地以五为制。周天气者,六期为一备。终地纪者,五岁为一周。君火以明,相火以位。五六相合,而七百二十气为一纪,凡三十岁;千四百四十气,凡六十岁而为一周。不及太过,斯皆见矣。

帝曰:夫子之言,上终天气,下毕地纪,可谓悉矣。余愿闻而藏之,上以治民,下以治身,使百姓昭著,上下和亲,德泽下流,子孙无忧,传之后世,无有终时,可得闻乎?

鬼臾区曰:至数之机,迫迮以微,其来可见,其往可追,敬之者昌,慢之者亡,无道行私,必得天殃,谨奉天道,请言真要。

帝曰:善言始者必会于终,善言近者必知其远。是则至数极而道不惑,所谓明矣。愿夫子推而次之,令有条理,简而不匮,久而不绝,易用难忘,为之纲纪。至数之要,愿尽闻之。

鬼臾区曰:昭乎哉问!明乎哉道!如鼓之应桴,响之应声也。臣闻之,甲己之岁,土运统之;乙庚之岁,金运统之;丙辛之岁,水运统之;丁壬之岁,木运统之;戊癸之岁,火运统之。

帝曰:其于三阴三阳合之奈何?

鬼臾区曰:子午之岁,上见少阴;丑未之岁,上见太阴;寅申之岁,上见少阳;卯酉之岁,上见阳明;辰戌之岁,上见太阳;巳亥之岁,上见厥阴。少阴所谓标也,厥阴所谓终也。厥阴之上,风气主之;少阴之上,热气主之;太阴之上,湿气主之;少阳之上,相火主之;阳明之上,燥气主之;太阳之上,寒气主之。所谓本也,是谓六元。

帝曰:光乎哉道!明乎哉论!请著之玉版,藏之金匮,署曰《天元纪》。
\section{五运行大论}
%%	67	%%
黄帝坐明堂,始正天纲,临观八极,考建五常。请天师而问之曰:论言天地之动静,神明为之纪;阴阳之升降,寒暑彰其兆。余闻五运之数于夫子,夫子之所言,正五气之各主岁耳,首甲定运,余因论之。鬼臾区曰:“土主甲己,金主乙庚,水主丙辛,木主丁壬,火主戊癸。子午之上,少阴主之;丑未之上,太阴主之;寅申之上,少阳主之;卯酉之上,阳明主之;辰戌之上,太阳主之;巳亥之上,厥阴主之。”不合阴阳,其故何也?

岐伯曰:是明道也。此天地之阴阳也。夫数之可数者,人中之阴阳也。然所合,数之可得者也。夫阴阳者,数之可十,推之可百,数之可千,推之可万。天地阴阳者,不以数推,以象之谓也。

帝曰:愿闻其所始也。

岐伯曰:昭乎哉问也!臣览《太始天元册》文,丹天之气经于牛女戊分,黅天之气经于心尾己分,苍天之气经于危室柳鬼,素天之气经于亢氐昴毕,玄天之气经于张翼娄胃。所谓戊己分者,奎壁角轸,则天地之门户也。夫候之所始,道之所生,不可不通也。

帝曰:善。论言,天地者万物之上下,左右者阴阳之道路,未知其所谓也。

岐伯曰:所谓上下者,岁上下见阴阳之所在也。左右者,诸上,见厥阴,左少阴,右太阳;见少阴,左太阴,右厥阴;见太阴,左少阳,右少阴;见少阳,左阳明,右太阴;见阳明,左太阳,右少阳;见太阳,左厥阴,右阳明。所谓面北而定其位,言其见也。

帝曰:何谓下?

岐伯曰:厥阴在上,则少阳在下,左阳明,右太阴;少阴在上,则阳明在下,左太阳,右少阳;太阴在上,则太阳在下,左厥阴,右阳明;少阳在上,则厥阴在下,左少阴,右太阳;阳明在上,则少阴在下,左太阴,右厥阴;太阳在上,则太阴在下,左少阳,右少阴。所谓面南而命其位,言其见也。上下相遘,寒暑相临,气相得则和,不相得则病。

帝曰:气相得而病者,何也?

岐伯曰:以下临上,不当位也。

帝曰:动静何如?

岐伯曰:上者右行,下者左行,左右周天,余而复会也。

帝曰:予闻鬼臾区曰:“应地者静。”今夫子乃言下者左行,不知其所谓也。愿闻何以生之乎?

岐伯曰:天地动静,五行迁复,虽鬼臾区其上候而已,犹不能遍明。夫变化之用,天垂象,地成形,七曜纬虚,五行丽地。地者,所以载生成之形类也。虚者,所以列应天之精气也。形精之动,犹根本之与枝叶也。仰观其象,虽远可知也。

帝曰:地之为下否乎?

岐伯曰:地为人之下,太虚之中者也。

帝曰:凭乎?

岐伯曰:大气举之也。燥以干之,暑以蒸之,风以动之,湿以润之,寒以坚之,火以温之。故风寒在下,燥热在上,湿气在中,火游行其间,寒暑六入,故令虚而化生也。故燥胜则地干,暑胜则地热,风胜则地动,湿胜则地泥,寒胜则地裂,火胜则地固矣。

帝曰:天地之气,何以候之?

岐伯曰:天地之气,胜复之作,不形于诊也。《脉法》曰:“天地之变,无以脉诊。”此之谓也。

帝曰:间气何如?

岐伯曰:随气所在,期于左右。

帝曰:期之奈何?

岐伯曰:从其气则和,逢其气则病。不当其位者病,迭移其位者病,失守其位者危,尺寸反者死,阴阳交者死。先立其年,以知其气,左右应见,然后乃可以言死生之逆顺。

帝曰:寒暑燥湿风火,在人合之奈何?其于万物何以生化?

岐伯曰:东方生风,风生木,木生酸,酸生肝,肝生筋,筋生心。其在天为玄,在人为道,在地为化。化生五味,道生智,玄生神,化生气。神在天为风,在地为木,在体为筋,在气为柔,在脏为肝。其性为暄,其德为和,其用为动,其色为苍,其化为荣,其虫毛,其政为散,其令宣发,其变摧拉,其眚为陨,其味为酸,其志为怒。怒伤肝,悲胜怒;风伤肝,燥胜风;酸伤筋,辛胜酸。

南方生热,热生火,火生苦,苦生心,心生血,血生脾。其在天为热,在地为火,在体为脉,在气为息,在脏为心。其性为暑,其德为显,其用为躁,其色为赤,其化为茂,其虫羽,其政为明,其令郁蒸,其变炎烁,其眚燔焫,其味为苦,其志为喜。喜伤心,恐胜喜;热伤气,寒胜热;苦伤气,咸胜苦。

中央生湿,湿生土,土生甘,甘生脾,脾生肉,肉生肺。其在天为湿,在地为土,在体为肉,在气为充,在脏为脾。其性静兼,其德为濡,其用为化,其色为黄,其化为盈,其虫倮,其政为谧,其令云雨,其变动注,其眚淫溃,其味为甘,其志为思。思伤脾,怒胜思;湿伤肉,风胜湿;甘伤脾,酸胜甘。

西方生燥,燥生金,金生辛,辛生肺,肺生皮毛,皮毛生肾。其在天为燥,在地为金,在体为皮毛,在气为成,在脏为肺。其性为凉,其德为清,其用为固,其色为白,其化为敛,其虫介,其政为劲,其令雾露,其变肃杀,其眚苍落,其味为辛,其志为忧。忧伤肺,喜胜忧;热伤皮毛,寒胜热;辛伤皮毛,苦胜辛。

北方生寒,寒生水,水生咸,咸生肾,肾生骨髓,髓生肝。其在天为寒,在地为水,在体为骨,在气为坚,在脏为肾。其性为凛,其德为寒,其用为脏,其色为黑,其化为肃,其虫鳞,其政为静,其令霰雪,其变凝冽,其眚冰雹,其味为咸,其志为恐。恐伤肾,思胜恐;寒伤血,燥胜寒;咸伤血,甘胜咸。

五气更立,各有所先,非其位则邪,当其位则正。

帝曰:病生之变何如?

岐伯曰:气相得则微,不相得则甚。

帝曰:主岁何如?

岐伯曰:气有余则制己所胜而侮所不胜,其不及则己所不胜侮而乘之,己所胜轻而侮之。侮反受邪,侮而受邪,寡于畏也。

帝曰:善。
\section{六微旨大论}
%%	68	%%
黄帝问曰:呜呼远哉,天之道也!如迎浮云,若视深渊,视深渊尚可测,迎浮云莫知其极。夫子数言谨奉天道,予闻而藏之,心私异之,不知其所谓也。愿夫子溢志尽言其事,令终不灭,久而不绝,天之道可得闻乎?

岐伯稽首再拜,对曰:明乎哉问天之道也!此因天之序,盛衰之时也。

帝曰:愿闻天道六六之节盛衰何也?

岐伯曰:上下有位,左右有纪。故少阳之右,阳明治之;阳明之右,太阳治之;太阳之右,厥阴治之;厥阴之右,少阴治之;少阴之右,太阴治之;太阴之右,少阳治之。此所谓气之标,盖南面而待之也。故曰:“因天之序,盛衰之时,移光定位,正立而待之。”此之谓也。

少阳之上,火气治之,中见厥阴;阳明之上,燥气治之,中见太阴;太阳之上,寒气治之,中见少阴;厥阴之上,风气治之,中见少阳;少阴之上,热气治之,中见太阳;太阴之上,湿气治之,中见阳明。所谓本也。本之下,中之见也。见之下,气之标也。本标不同,气应异象。

帝曰:其有至而至,有至而不至,有至而太过,何也?

岐伯曰:至而至者,和;至而不至,来气不及也;未至而至,来气有余也。

帝曰:至而不至,未至而至,何如?

岐伯曰:应则顺,否则逆,逆则变生,变生则病。

帝曰:善。请言其应。

岐伯曰:物,生其应也。气,脉其应也。

帝曰:善。愿闻地理之应六节气位何如?

岐伯曰:显明之右,君火之位也。君火之右,退行一步,相火治之;复行一步,土气治之;复行一步,金气治之;复行一步,水气治之;复行一步,木气治之;复行一步,君火治之。

相火之下,水气承之;水位之下,土气承之;土位之下,风气承之;风位之下,金气承之;金位之下,火气承之;君火之下,阴精承之。

帝曰:何也?

岐伯曰:亢则害,承乃制,制则生化,外列盛衰;害则败乱,生化大病。

帝曰:盛衰何如?

岐伯曰:非其位则邪,当其位则正。邪则变甚,正则微。

帝曰:何谓当位?

岐伯曰:木运临卯,火运临午,土运临四季,金运临酉,水运临子。所谓:“岁会”,气之平也。

帝曰:非其位何如?

岐伯曰:岁不与会也。

帝曰:土运之岁,上见太阴;火运之岁,上见少阳、少阴;金运之岁,上见阳明;木运之岁,上见厥阴;水运之岁,上见太阳。奈何?

岐伯曰:天与之会也。故《天元册》曰:“天符”。

帝曰:天符、岁会何如?

岐伯曰:“太一天符”之会也。

帝曰:其贵贱何如?

岐伯曰:天符为执法,岁会为行令,太一天符为贵人。

帝曰:邪之中也奈何?

岐伯曰:中执法者,其病速而危;中行令者,其病徐而持;中贵人者,其病暴而死。

帝曰:位之易也何如?

岐伯曰:君位臣则顺,臣位君则逆。逆则其病近,其害速;顺则其病远,其害微。所谓二火也。

帝曰:善。愿闻其步何如?

岐伯曰:所谓步者,六十度而有奇,故二十四步,积盈百刻而成日也。

帝曰:六气应五行之变何如?

岐伯曰:位有终始,气有初中上下不同,求之亦异也。

帝曰:求之奈何?

岐伯曰:天气始于甲,地气始于子,子甲相合,命曰岁立,谨候其时,气可与期。

帝曰:愿闻其岁,六气始终,早晏何如?

岐伯曰:明乎哉问也!

甲子之岁,初之气,天数始于水下一刻,终于八十七刻半;二之气,始于八十七刻六分,终于七十五刻;三之气,始于七十六刻,终于六十二刻半;四之气,始于六十二刻六分,终于五十刻;五之气,始于五十一刻,终于三十七刻半;六之气,始于三十七刻六分,终于二十五刻。所谓初六,天之数也。

乙丑岁,初之气,天数始于二十六刻,终于一十二刻半;二之气,始于一十二刻六分,终于水下百刻;三之气,始于一刻,终于八十七刻半;四之气,始于八十七刻六分,终于七十五刻;五之气,始于七十六刻,终于六十二刻半;六之气,始于六十二刻六分,终于五十刻。所谓六二,天之数也。

丙寅岁,初之气,天数始于五十一刻,终于三十七刻半;二之气,始于三十七刻六分,终于二十五刻;三之气,始于二十六刻,终于一十二刻半;四之气,始于一十二刻六分,终于水下百刻;五之气,始于一刻,终于八十七刻半;六之气,始于八十七刻六分,终于七十五刻。所谓六三,天之数也。

丁卯岁,初之气,天数始于七十六刻,终于六十二刻半;二之气,始于六十二刻六分,终于五十刻;三之气,始于五十一刻,终于三十七刻半;四之气,始于三十七刻六分,终于二十五刻;五之气,始于二十六刻,终于一十二刻半;六之气,始于一十二刻六分,终于水下百刻。所谓六四,天之数也。

次戊辰岁,初之气,复始于一刻,常如是无已,周而复始。

帝曰:愿闻其岁候何如?

岐伯曰:悉乎哉问也!日行一周,天气始于一刻;日行再周,天气始于二十六刻;日行三周,天气始于五十一刻;日行四周,天气始于七十六刻;日行五周,天气复始于一刻,所谓一纪也。是故寅午戌岁气会同,卯未亥岁气会同,辰申子岁气会同,巳酉丑岁气会同,终而复始。

帝曰:愿闻其用也。

岐伯曰:言天者求之本,言地者求之位,言人者求之气交。

帝曰:何谓气交?

岐伯曰:上下之位,气交之中,人之居也。故曰:“天枢之上,天气主之;天枢之下,地气主之;气交之分,人气从之,万物由之。”此之谓也。

帝曰:何谓初中?

岐伯曰:初凡三十度而有奇,中气同法。

帝曰:初中何也?

岐伯曰:所以分天地也。

帝曰:愿卒闻之!

岐伯曰:初者地气也,中者天气也。

帝曰:其升降何如?

岐伯曰:气之升降,天地之更用也。

帝曰:愿闻其用何如?

岐伯曰:升已而降,降者谓天;降已而升,升者谓地。天气下降,气流于地;地气上升,气腾于天。故高下相召,升降相因,而变作矣。

帝曰:善。寒湿相遘,燥热相临,风火相值,其有间乎?

岐伯曰:气有胜复,胜复之作,有德有化,有用有变,变则邪气居之。

帝曰:何谓邪乎?

岐伯曰:夫物之生从于化,物之极由乎变,变化之相薄,成败之所由也。故气有往复,用有迟速,四者之有,而化而变,风之来也。

帝曰:迟速往复,风所由生,而化而变,故因盛衰之变耳。成败倚伏游乎中,何也?

岐伯曰:成败倚伏生乎动,动而不已则变作矣。

帝曰:有期乎?

岐伯曰:不生不化,静之期也。

帝曰:不生化乎?

岐伯曰:出入废则神机化灭,升降息则气立孤危。故,非出入则无以生长壮老已,非升降则无以生长化收藏。是以升降出入,无器不有。故器者,生化之宇,器散则分之,生化息矣。故无不出入,无不升降,化有小大,期有近远,四者之有,而贵常守,反常则灾害至矣。故曰:“无形无患。”此之谓也。

帝曰:善。有不生不化乎?

岐伯曰:悉乎哉问也!与道合同,惟真人也。

帝曰:善。
\section{气交变大论}
%%	69	%%
黄帝问曰:五运更治,上应天期,阴阳往复,寒暑迎随,真邪相薄,内外分离,六经波荡,五气倾移,太过不及,专胜兼并,愿言其始,而有常名,可得闻乎?

岐伯稽首再拜对曰:昭乎哉问也!是明道也!此上帝所贵,先师传之,臣虽不敏,往闻其旨。

帝曰:余闻:“得其人不教,是谓失道,传非其人,慢泄天宝。”余诚菲德,未足以受至道,然而众子哀其不终,愿夫子保于无穷,流于无极,余司其事,则而行之,奈何?

岐伯曰:请遂言之也。《上经》曰:“夫道者,上知天文,下知地理,中知人事,可以长久。”此之谓也。

帝曰:何谓也?

岐伯曰:本气位也。位天者,天文也;位地者,地理也;通于人气之变化者,人事也。故太过者先天,不及者后天,所谓治化而人应之也。

帝曰:五运之化,太过何如?

岐伯曰:岁木太过,风气流行,脾土受邪。民病飧泄,食减,体重,烦冤,肠鸣,腹支满,上应岁星。甚则忽忽善怒,眩冒巅疾。化气不政,生气独治,云物飞动,草木不宁,甚而摇落,反胁痛而吐甚,冲阳绝者死不治,上应太白星。

岁火太过,炎暑流行,金肺受邪。民病疟,少气,咳喘,血溢、血泄、注下,嗌燥,耳聋,中热、肩背热,上应荧惑星。甚则胸中痛,胁支满,胁痛,膺背肩胛间痛,两臂内痛,身热骨痛而为浸淫。收气不行,长气独明,雨水霜寒,上应辰星。上临少阴少阳,火燔焫,水泉涸,物焦槁,病反谵妄狂越,咳喘,息呜,下甚,血溢泄不已,太渊绝者死不治,上应荧惑星。

岁土太过,雨湿流行,肾水受邪。民病腹痛,清厥,意不乐,体重,烦冤,上应镇星。甚则肌肉萎,足萎不收,行善瘈,脚下痛,饮发,中满,食减,四肢不举。变生得位,藏气伏,化气独治之,泉涌河衍,涸泽生鱼,风雨大至,土崩溃,鳞见于陆,病腹满,溏泄,肠鸣,反下甚而太溪绝者死不治,上应岁星。

岁金太过,燥气流行,肝木受邪。民病两胁下少腹痛,目赤痛,眦疡,耳无所闻。肃杀而甚,则体重,烦冤,胸痛引背,两胁满且痛引少腹,上应太白星。甚则喘咳,逆气,肩背痛,尻阴、股、膝、髀、腨、胻、足皆病,上应荧惑星。收气峻,生气下,草木敛,苍干凋陨,病反暴痛,胠胁不可反侧,咳逆甚而血溢,太冲绝者死不治,上应太白星。

岁水太过,寒气流行,邪害心火。民病身热,烦心,躁悸,阴厥,上下中寒,谵妄,心痛,寒气早至,上应辰星。甚则腹大胫肿,喘咳,寝汗出,憎风,大雨至,埃雾朦郁,上应镇星。上临太阳,雨冰雪,霜不时降,湿气变物,病反腹满,肠鸣,溏泄,食不化,渴而妄冒,神门绝者死不治,上应荧惑、辰星。

帝曰:善。其不及何如?

岐伯曰:悉乎哉问也!

岁木不及,燥乃大行,生气失应,草木晚荣,肃杀而甚,则刚木辟着,柔萎苍干,上应太白星。民病中清,胠胁痛,少腹痛,肠鸣,溏泄,凉雨时至,上应太白星(、岁星),其谷苍。上临阳明,生气失政,草木再荣,化气乃急,上应太白、镇星,其主苍早。复则炎暑流火,湿性燥,柔脆草木焦槁,下体再生,华实齐化,病寒热,疮疡,疿胗,痈痤,上应荧惑、太白,其谷白坚。白露早降,收杀气行,寒雨害物,虫食甘黄,脾上受邪,赤气后化,心气晚治,上胜肺金,白气乃屈,其谷不成,咳而鼽,上应荧惑、太白星。

岁火不及,寒乃大行,长政不用,物荣而下,凝惨而甚,则阳气不化,乃折荣美,上应辰星。民病胸中痛,胁支满,两胁痛,膺背肩胛间及两臂内痛,郁冒朦昧,心痛暴瘖,胸腹大,胁下与腰背相引而痛,甚则屈不能伸,髋髀如别,上应荧惑、辰星,其谷丹。复则埃郁,大雨且至,黑气乃辱,病骛溏,腹满,食饮不下,寒中,肠鸣,泄注,腹痛,暴挛痿痹,足不任身,上应镇星、辰星,玄谷不成。

岁土不及,风乃大行,化气不令,草木茂荣,飘扬而甚,秀而不实,上应岁星。民病飧泄霍乱,体重腹痛,筋骨繇复,肌肉瞤酸,善怒,藏气举事,蛰虫早附,咸病寒中,上应岁星、镇星,其谷黅。复则收政严峻,名木苍凋,胸胁暴痛,下引少腹,善太息,虫食甘黄,气客于脾,黅谷乃减,民食少失味,苍谷乃损,上应太白、岁星。上临厥阴,流水不冰,蛰虫来见,藏气不用,白乃不复,上应岁星,民乃康。

岁金不及,炎火乃行,生气乃用,长气专胜,庶物以茂,燥烁以行,上应荧惑星。民病肩背瞀重,鼽嚏,血便,注下,收气乃后,上应太白(、荧惑)星,其谷坚芒。复则寒雨暴至,乃零冰雹霜雪杀物,阴厥且格,阳反上行,头脑户痛,延及囟顶发热,上应辰星(、荧惑),丹谷不成,民病口疮,甚则心痛。

岁水不及,湿乃大行,长气反用,其化乃速,暑雨数至,上应镇星。民病腹满身重,濡泄,寒疡流水,腰股痛发,腘、腨、股、膝不便,烦冤,足痿,清厥,脚下痛,甚则跗肿,藏气不政,肾气不衡,上应(镇星、)辰星,其谷秬。上临太阴,则大寒数举,蛰虫早藏,地积坚冰,阳光不治,民病寒疾于下,甚则腹满浮肿,上应镇星(、荧惑),其主黅谷。复则大风暴发,草偃木零,生长不鲜,面色时变,筋骨并辟,肉瞤瘛,目视䀮䀮,物疏璺,肌肉胗发,气并膈中,痛于心腹,黄气乃损,其谷不登,上应岁星(、镇星)。

帝曰:善。愿闻其时也。

岐伯曰:悉乎哉问也!

木不及,春有鸣条律畅之化,则秋有雾露清凉之政;春有惨凄残贼之胜,则夏有炎暑燔烁之复。其眚东,其脏肝,其病内舍胠胁,外在关节。

火不及,夏有炳明光显之化,则冬有严肃霜寒之政;夏有惨凄凝冽之胜,则不时有埃昏大雨之复。其眚南,其脏心,其病内舍膺胁,外在经络。

土不及,四维有埃云润泽之化,则春有鸣条鼓拆之政;四维发振拉飘腾之变,则秋有肃杀霖霪之复。其眚四维,其脏脾,其病内舍心腹,外在肌肉四肢。

金不及,夏有光显郁蒸之令,则冬有严凝整肃之应;夏有炎烁燔燎之变,则秋有冰雹霜雪之复。其眚西,其脏肺,其病内舍膺胁肩背,外在皮毛。

水不及,四维有湍润埃云之化,则不时有和风生发之应;四维发埃昏骤注之变,则不时有飘荡振拉之复。其眚北,其脏肾,其病内舍腰脊骨髓,外在溪谷腨膝。

夫五运之政,犹权衡也,高者抑之,下者举之。化者应之,变者复之,此生长化成收藏之理,气之常也,失常则天地四塞矣。故曰:“天地之动静,神明为之纪,阴阳之往复,寒暑彰其兆。”此之谓也。

帝曰:夫子之言五气之变,四时之应,可谓悉矣。夫气之动乱,触遇而作,发无常会,卒然灾合,何以期之?

岐伯曰:夫气之动变,固不常在,而德化政令灾变,不同其候也。

帝曰:何谓也?

岐伯曰:东方生风,风生木,其德敷和,其化生荣,其政舒启,其令风,其变振发,其灾散落。南方生热,热生火,其德彰显,其化蕃茂,其政明曜,其令热,其变销烁,其灾燔焫。中央生湿,湿生土,其德溽蒸,其化丰备,其政安静,其令湿,其变骤注,其灾霖溃。西方生燥,燥生金,其德清洁,其化紧敛,其政劲切,其令燥,其变肃杀,其灾苍陨。北方生寒,寒生水,其德凄沧,其化清谧,其政凝肃,其令寒,其变凓冽,其灾冰雪霜雹。是以察其动也,有德有化,有政有令,有变有灾,而物由之,而人应之也。

帝曰:夫子之言岁候,其太过不及而上应五星。今夫德化政令,灾眚变易,非常而有也,卒然而动,其亦为之变乎?

岐伯曰:承天而行之,故无妄动,无不应也。卒然而动者,气之交变也,其不应焉。故曰:“应常不应卒。”此之谓也。

帝曰:其应奈何?

岐伯曰:各从其气化也。

帝曰:其行之徐疾逆顺何如?

岐伯曰:以道留久,逆守而小,是谓省下。以道而去,去而速来,曲而过之,是谓省遗过也。久留而环,或离或附,是谓议灾与其德也。应近则小,应远则大。芒而大,倍常之一,其化甚;大常之二,其眚即也;小常之一,其化减;小常之二,是谓临视,省下之过与其德也。德者福之,过者伐之,是以象之见也,高而远则小,下而近则大,故大则喜怒迩,小则祸福远。岁运太过,则运星北越,运气相得,则各行其道。故岁运太过,畏星失色而兼其母,不及则色兼其所不胜。肖者瞿瞿,莫知其妙。闵闵之当,孰者为良。妄行无征,示畏侯王。

帝曰:其灾应何如?

岐伯曰:亦各从其化也。故时至有盛衰,凌犯有逆顺,留守有多少,形见有善恶,宿属有胜负,征应有吉凶矣。

帝曰:其善恶何谓也?

岐伯曰:有喜有怒,有忧有丧,有泽有燥,此象之常也,必谨察之。

帝曰:六者高下异乎?

岐伯曰:象见高下,其应一也,故人亦应之。

帝曰:善。其德化政令之动静损益,皆何如?

岐伯曰:夫德、化、政、令、灾、变,不能相加也;胜复盛衰,不能相多也;往来小大,不能相过也;用之升降,不能相无也;各从其动而复之耳。

帝曰:其病生何如?

岐伯曰:德化者气之祥,政令者气之章,变易者复之纪,灾眚者伤之始,气相胜者和,不相胜者病,重感于邪则甚也。

帝曰:善。所谓精光之论,大圣之业,宣明大道,通于无穷,究于无极也。余闻之:“善言天者,必应于人;善言古者,必验于今;善言气者,必彰于物;善言应者,同天地之化;善言化言变者,通神明之理。”非夫子孰能言至道欤!

乃择良兆而藏之灵室,每旦读之,命曰:《气交变》,非斋戒不敢发,慎传也。
\section{五常政大论}
%%	70	%%
黄帝问曰:太虚寥廓,五运回薄,衰盛不同,损益相从,愿闻平气何如而名?何如而纪也?

岐伯对曰:昭乎哉问也!木曰敷和,火曰升明,土曰备化,金曰审平,水曰静顺。

帝曰:其不及奈何?

岐伯曰:木曰委和,火曰伏明,土曰卑监,金曰从革,水曰涸流。

帝曰:太过何谓?

岐伯曰:木曰发生,火曰赫曦,土曰敦阜,金曰坚成,水曰流衍。

帝曰:三气之纪,愿闻其候。

岐伯曰:悉乎哉问也!

敷和之纪,木德周行,阳舒阴布,五化宣平。其气端,其性随,其用曲直,其化生荣,其类草木,其政发散,其候温和,其令风,其脏肝,肝其畏清,其主目,其谷麻,其果李,其实核,其应春,其虫毛,其畜犬,其色苍,其养筋,其病里急支满,其味酸,其音角,其物中坚,其数八。

升明之纪,正阳而治,德施周普,五化均衡。其气高,其性速,其用燔灼,其化蕃茂,其类火,其政明曜,其候炎暑,其令热,其脏心,心其畏寒,其主舌,其谷麦,其果杏,其实络,其应夏,其虫羽,其畜马,其色赤,其养血,其病瞤瘈,其味苦,其音徵,其物脉,其数七。

备化之纪,气协天休,德流四政,五化齐修。其气平,其性顺,其用高下,其化丰满,其类土,其政安静,其候溽蒸,其令湿,其脏脾,脾其畏风,其主口,其谷稷,其果枣,其实肉,其应长夏,其虫倮,其畜牛,其色黄,其养肉,其病否,其味甘,其音宫,其物肤,其数五。

审平之纪,收而不争,杀而无犯,五化宣明。其气洁,其性刚,其用散落,其化坚敛,其类金,其政劲肃,其候清切,其令燥,其脏肺,肺其畏热,其主鼻,其谷稻,其果桃,其实壳,其应秋,其虫介,其畜鸡,其色白,其养皮毛,其病咳,其味辛,其音商,其物外坚,其数九。

静顺之纪,藏而勿害,治而善下,五化咸整。其气明,其性下,其用沃衍,其化凝坚,其类水,其政流演,其候凝肃,其令寒,其脏肾,肾其畏湿,其主二阴,其谷豆,其果栗,其实濡,其应冬,其虫鳞,其畜彘,其色黑,其养骨髓,其病厥,其味咸,其音羽,其物濡,其数六。

故生而勿杀,长而勿罚,化而勿制,收而勿害,藏而勿抑,是谓平气。

委和之纪,是谓胜生,生气不政,化气乃扬,长气不平,收令乃早,凉雨时降,风云并兴,草木晚荣,苍干凋落,物秀而实,肤肉内充。其气敛,其用聚,其动软戾拘缓,其发惊骇,其脏肝,其果枣李,其实核壳,其谷稷稻,其味酸辛,其色白苍,其畜犬鸡,其虫毛介,其主雾露凄沧,其声角商,其病摇动注恐,从金化也。少角与判商同,上角与正角同,上商与正商同。其病肢废、痈肿、疮疡,其甘虫,邪伤肝也。上宫与正宫同。萧瑟肃杀,则炎赫沸腾,眚于三,所谓复也,其主飞、蠹、蛆、雉,乃为雷霆。

伏明之纪,是谓胜长,长气不宣,藏气反布,收气自政,化令乃衡,寒清数举,暑令乃薄,承化物生,生而不长,成实而稚,遇化已老,阳气屈伏,蛰虫早藏。其气郁,其用暴,其动彰伏变易,其发痛,其脏心,其果栗桃,其实络濡,其谷豆稻,其味苦咸,其色玄丹,其畜马彘,其虫羽鳞,其主冰雪霜寒,其声徵羽,其病昏惑悲忘,从水化也。少徵与少羽同,上商与正商同。邪伤心也。凝惨凓冽,则暴雨霖霪,眚于九,其主骤注,雷霆震惊,沉霒淫雨。

卑监之纪,是谓减化,化气不令,生政独彰,长气整,雨乃衍,收气平,风寒并兴,草木荣美,秀而不实,成而粃也。其气散,其用静定,其动疡涌,分溃痈肿,其发濡滞,其脏脾,其果李栗,其实濡核,其谷豆麻,其味酸甘,其色苍黄,其畜牛犬,其虫倮毛,其主飘怒振发,其声宫角,其病留满否塞,从木化也。少宫与少角同,上宫与正宫同,上角与正角同。其病飧泄,邪伤脾也。振拉飘扬,则苍干散落,其眚四维,其主败折虎狼,清气乃用,生政乃辱。

从革之纪,是谓折收,收气乃后,生气乃扬,长化合德,火政乃宣,庶类以蕃。其气扬,其用躁切,其动铿禁瞀厥,其发欬喘,其脏肺,其果李杏,其实壳络,其谷麻麦,其味苦辛,其色白丹,其畜鸡羊,其虫介羽,其主明曜炎烁,其声商徵,其病嚏咳鼽衂,从火化也。少商与少徵同,上商与正商同,上角与正角同。邪伤肺也。炎光赫烈,则冰雪霜雹,眚于七,其主鳞伏彘鼠,藏气早至,乃生大寒。

涸流之纪,是谓反阳,藏令不举,化气乃昌,长气宣布,蛰虫不藏,土润水泉减,草木条茂,荣秀满盛。其气滞,其用渗泄,其动坚止,其发燥槁,其脏肾,其果枣杏,其实濡肉,其谷黍稷,其味甘咸,其色黅玄,其畜彘牛,其虫鳞倮,其主埃郁昏翳,其声羽宫,其病痿厥坚下,从土化也。少羽与少宫同,上官与正宫同。其病癃闭,邪伤肾也。埃昏骤雨,则振拉摧拔,眚于一,其主毛显狐狢,变化不藏。

故乘危而行,不速而至,暴疟无德,灾反及之,微者复微,甚者复甚,气之常也。

发生之纪,是谓启陈,土疏泄,苍气达,阳和布化,阴气乃随,生气淳化,万物以荣。其化生,其气美,其政散,其令条舒,其动掉眩巅疾,其德鸣靡启坼,其变振拉摧拔,其谷麻稻,其畜鸡犬,其果李桃,其色青黄白,其味酸甘辛,其象春,其经足厥阴少阳,其脏肝脾,其虫毛介,其物中坚外坚,其病怒。太角与上商同。上征则其气逆,其病吐利。不务其德,则收气复,秋气劲切,甚则肃杀,清气大至,草木凋零,邪乃伤肝。

赫曦之纪,是谓蕃茂,阴气内化,阳气外荣,炎暑施化,物得以昌。其化长,其气高,其政动,其令明显,其动炎灼妄扰,其德暄暑郁蒸,其变炎烈沸腾,其谷麦豆,其畜羊彘,其果杏栗,其色赤白玄,其味苦辛咸,其象夏,其经手少阴太阳、手厥阴少阳,其脏心肺,其虫羽鳞,其物脉濡,其病笑、疟、疮疡、血流、狂妄、目赤。上羽与正徵同。其收齐,其病痓,上徵而收气后也。暴烈其政,藏气乃复,时见凝惨,甚则雨水霜雹切寒,邪伤心也。

敦阜之纪,是谓广化,厚德清静,顺长以盈,至阴内实,物化充成,烟埃朦郁,见于厚土,大雨时行,湿气乃用,燥政乃辟。其化圆,其气丰,其政静,其令周备,其动濡积并稸,其德柔润重淖,其变震惊、飘骤、崩溃,其谷稷麻,其畜牛犬,其果枣李,其色黅玄苍,其味甘咸酸,其象长夏,其经足太阴、阳明,其脏脾肾,其虫倮毛,其物肌核,其病腹满,四肢不举,大风迅至,邪伤脾也。

坚成之纪,是谓收引,天气洁,地气明,阳气随,阴治化,燥行其政,物以司成,收气繁布,化治不终。其化成,其气削,其政肃,其令锐切,其动暴折疡疰,其德雾露萧瑟,其变肃杀凋零,其谷稻黍,其畜鸡马,其果桃杏,其色白青丹,其味辛酸苦,其象秋,其经手太阴、阳明,其脏肺肝,其虫介羽,其物壳络,其病喘喝、胸凭、仰息。上徵与正商同。其生齐,其病咳,政暴变则名木不荣,柔脆焦首,长气斯救,大火流,炎烁且至,蔓将槁,邪伤肺也。

流衍之纪,是谓封藏,寒司物化,天地严凝,藏政以布,长令不扬。其化凛,其气坚,其政谧,其令流注,其动漂泄沃涌,其德凝惨寒雰,其变冰雪霜雹,其谷豆稷,其畜彘牛,其果栗枣,其色黑丹黅,其味咸苦甘,其象冬,其经足少阴、太阳,其脏肾心,其虫鳞倮,其物濡满,其病胀。上羽而长,气不化也。政过则化气大举,而埃昏气交,大雨时降,邪伤肾也。

故曰:“不恒其德,则所胜来复;政恒其理,则所胜同化。”此之谓也。

帝曰:天不足西北,左寒而右凉;地不满东南,右热而左温,其故何也?

岐伯曰:阴阳之气,高下之理,太少之异也。东南方,阳也,阳者,其精降于下,故右热而左温。西北方,阴也,阴者,其精奉于上,故左寒而右凉。是以地有高下,气有温凉。高者气寒,下者气热,故适寒凉者胀之,温热者疮,下之则胀已,汗之则疮已,此腠理开闭之常,太少之异耳。

帝曰:其于寿夭何如?

岐伯曰:阴精所奉其人寿,阳精所降其人夭。

帝曰:善。其病也治之奈何?

岐伯曰:西北之气散而寒之,东南之气收而温之,所谓同病异治也。故曰:“气寒气凉,治以寒凉,行水渍之;气温气热,治以温热,强其内守,必同其气,可使平也,假者反之。”

帝曰:善。一州之气,生化寿夭不同,其故何也?

岐伯曰:高下之理,地势使然也。崇高则阴气治之,污下则阳气治之,阳胜者先天,阴胜者后天,此地理之常,生化之道也。

帝曰:其有寿夭乎?

岐伯曰:高者其气寿,下者其气夭,地之小大异也,小者小异,大者大异。故治病者,必明天道地理,阴阳更胜,气之先后,人之寿夭,生化之期,乃可以知人之形气矣。

帝曰:善。其岁有不病,而脏气不应不用者,何也?

岐伯曰:天气制之,气有所从也。

帝曰:愿卒闻之!

岐伯曰:少阳司天,火气下临,肺气上从,白起金用,草木眚,火见燔焫,革金且耗,大暑以行,欬嚏,鼽衂,鼻窒,疮疡,寒热胕肿。风行于地,尘沙飞扬,心痛、胃脘痛,厥逆鬲不通,其主暴速。

阳明司天,燥气下临,肝气上从,苍起木用而立,土乃眚,凄沧数至,木伐草萎,胁痛,目赤,掉振鼓栗,筋痿不能久立。暴热至,土乃暑,阳气郁发,小便变,寒热如疟,甚则心痛,火行于槁,流水不冰,蛰虫乃见。

太阳司天,寒气下临,心气上从,而火且明,丹起金乃眚,寒清时举,胜则水冰,火气高明,心热烦,嗌干善渴,鼽嚏,喜悲数欠,热气妄行,寒乃复,霜不时降,善忘,甚则心痛。土乃润,水丰衍,寒客至,沉阴化,湿气变物,水饮内稸,中满不食,皮𤸷肉苛,筋脉不利,甚则胕肿,身后痈。

厥阴司天,风气下临,脾气上从,而土且隆,黄起水乃眚,土用革,体重,肌肉萎,食减口爽,风行太虚,云物摇动,目转耳鸣。火纵其暴,地乃暑,大热消烁,赤沃下,蛰虫数见,流水不冰,其发机速。

少阴司天,热气下临,肺气上从,白起金用,草木眚,喘呕,寒热,嚏鼽衂,鼻窒,大暑流行,甚则疮疡燔灼,金烁石流。地乃燥清,凄沧数至,胁痛,善太息,肃杀行,草木变。

太阴司天,湿气下临,肾气上从,黑起水变,(火乃眚,)埃冒云雨,胸中不利,阴痿,气大衰,而不起不用。当其时,反腰脽痛动转不便也,厥逆。地乃藏阴,大寒且至,蛰虫早附,心下否痛,地裂冰坚,少腹痛,时害于食,乘金则止水增,味乃咸,行水减也。

帝曰:岁有胎孕不育,治之不全,何气使然?

岐伯曰:六气五类,有相胜制也,同者盛之,异者衰之,此天地之道,生化之常也。

故厥阴司天,毛虫静,羽虫育,介虫不成;在泉,毛虫育,倮虫耗,羽虫不育。

少阴司天,羽虫静,介虫育,毛虫不成;在泉,羽虫育,介虫耗不育。

太阴司天,倮虫静,鳞虫育,羽虫不成;在泉,倮虫育,鳞虫不成。

少阳司天,羽虫静,毛虫育,倮虫不成;在泉,羽虫育,介虫耗,毛虫不育。

阳明司天,介虫静,羽虫育,介虫不成;在泉,介虫育,毛虫耗,羽虫不成。

太阳司天,鳞虫静,倮虫育;在泉,鳞虫耗,倮虫不育。

诸乘所不成之运,则甚也。故气主有所制,岁立有所生,地气制己胜,天气制胜己,天制色,地制形,五类衰盛,各随其气之所宜也。故有胎孕不育,治之不全,此气之常也,所谓中根也。根于外者亦五,故生化之别,有五气、五味、五色、五类,五宜也。

帝曰:何谓也?

岐伯曰:根于中者,命曰神机,神去则机息。根于外者,命曰气立,气止则化绝。故各有制,各有胜,各有生,各有成。故曰“不知年之所加,气之同异,不足以言生化。”此之谓也。

帝曰:气始而生化,气散而有形,气布而蕃育,气终而象变,其致一也。然而五味所资,生化有薄厚,成熟有少多,终始不同,其故何也?

岐伯曰:地气制之也,非天不生,地不长也。

帝曰:愿闻其道。

岐伯曰:寒热燥湿,不同其化也。

故少阳在泉,寒毒不生,其味辛,其治苦酸,其谷苍丹。

阳明在泉,湿毒不生,其味酸,其气湿,其治辛苦甘,其谷丹素。

太阳在泉,热毒不生,其味苦,其治淡咸,其谷黅秬。

厥阴在泉,清毒不生,其味甘,其治酸苦,其谷苍赤,其气专,其味正。

少阴在泉,寒毒不生,其味辛,其治辛苦甘,其谷白丹。

太阴在泉,燥毒不生,其味咸,其气热,其治甘咸,其谷黅秬。化淳则咸守,气专则辛化而俱治。

故曰:“补上下者从之,治上下者逆之,以所在寒热盛衰而调之。”

故曰:“上取下取,内取外取,以求其过。能毒者以厚药,不胜毒者以薄药。”此之谓也。

气反者,病在上,取之下;病在下,取之上;病在中,旁取之。治热以寒,温而行之;治寒以热,凉而行之;治温以清,冷而行之;治清以温,热而行之。故消之削之,吐之下之,补之泻之,久新同法。

帝曰:病在中而不实不坚,且聚且散,奈何?

岐伯曰:悉乎哉问也!无积者求其脏,虚则补之,药以祛之,食以随之,行水渍之,和其中外,可使毕已。

帝曰:有毒无毒,服有约乎?

岐伯曰:病有久新,方有大小,有毒无毒,固宜常制矣。大毒治病,十去其六;常毒治病,十去其七;小毒治病,十去其八;无毒治病,十去其九。谷肉果菜,食养尽之。无使过之,伤其正也。不尽,行复如法,必先岁气,无伐天和,无盛盛,无虚虚,而遗人夭殃,无致邪,无失正,绝人长命。

帝曰:其久病者,有气从不康,病去而瘠,奈何?

岐伯曰:昭乎哉,圣人之问也!化不可代,时不可违。夫经络以通,血气以从,复其不足,与众齐同,养之和之,静以待时,谨守其气,无使倾移,其形乃彰,生气以长,命曰圣王。故《大要》曰:“无代化,无违时,必养必和,待其来复。”此之谓也。

帝曰:善。
\section{六元正纪大论}
%%	71	%%
黄帝问曰:六化六变,胜复淫治,甘苦辛咸酸淡先后,余知之矣。夫五运之化,或从天气,或逆天气,或从天气而逆地气,或从地气而逆天气,或相得,或不相得,余未能明其事。欲通天之纪,从地之理,和其运,调其化,使上下合德,无相夺伦,天地升降,不失其宜,五运宣行,勿乖其政,调之正味,从逆奈何?

岐伯稽首再拜对曰:昭乎哉问也!此天地之纲纪,变化之渊源,非圣帝孰能穷其至理欤!臣虽不敏,请陈其道,令终不灭,久而不易。

帝曰:愿夫子推而次之,从其类序,分其部主,别其宗司,昭其气数,明其正化,可得闻乎?

岐伯曰:先立其年以明其气,金木水火土运行之数,寒暑燥湿风火临御之化,则天道可见,民气可调,阴阳卷舒,近而无惑,数之可数者,请遂言之。

帝曰:太阳之政奈何?

岐伯曰:辰戌之纪也。

太阳,太角,太阴。壬辰,壬戌。其运风,其化鸣紊启拆,其变振拉摧拔,其病眩、掉、目瞑。太角(初正)、少徵、太宫、少商、太羽(终)。

太阳,太徵,太阴。戊辰,戊戌,同正徵。其运热,其化暄暑郁燠,其变炎烈沸腾,其病热郁。太徵、少宫、太商、少羽(终)、少角(初)。

太阳,太宫,太阴。甲辰岁会(同天符),甲戌岁会(同天符)。其运阴埃,其化柔润重泽,其变震惊飘骤,其病湿、下重。太宫、少商、太羽(终)、太角(初)、少徵。

太阳,太商,太阴。庚辰,庚戌。其运凉,其化雾露萧飋,其变肃杀凋零,其病燥、背瞀、胷满。太商、少羽(终)、少角(初)、太徵、少宫。

太阳,太羽,太阴。丙辰天符,丙戌天符。其运寒,其化凝惨凛冽,其变冰雪霜雹,其病大寒留于溪谷。太羽(终)、太角(初)、少徵、太宫、少商。

凡此太阳司天之政,气化运行先天,天气肃,地气静,寒临太虚,阳气不令,水土合德,上应辰星、镇星。其谷玄、黅,其政肃,其令徐。寒政大举,泽无阳焰,则火发待时。少阳中治,时雨乃涯,止极雨散,还于太阴,云朝北极,湿化乃布,泽流万物,寒敷于上,雷动于下,寒湿之气,持于气交。民病寒湿,发肌肉萎、足痿不收、濡泻、血溢。

初之气,地气迁,气乃大温,草乃早荣,民乃厉,温病乃作,身热、头痛、呕吐、肌腠疮疡。二之气,大凉反至,民乃惨,草乃遇寒,火气遂抑。民病气郁中满,寒乃始。三之气,天政布,寒气行,雨乃降。民病寒,反热中、痈疽、注下、心热、瞀闷,不治者死。四之气,风湿交争,风化为雨,乃长,乃化,乃成。民病大热,少气、肌肉萎、足痿、注下、赤白。五之气,阳复化,草乃长,乃化,乃成。民乃舒。终之气,地气正,湿令行,阴凝太虚,埃昏郊野。民乃惨凄,寒风以至,反者孕乃死。

必折其郁气,先资其化源,抑其运气,扶其不胜,无使暴过而生其疾,食岁谷以全其真,避虚邪以安其正,故岁宜苦以燥之温之(读注:此句经文原在本段首,为错简,调整至此)。适气同异,多少制之,同寒湿者燥热化,异寒湿者燥湿化,故同者多之,异者少之,用寒远寒,用凉远凉,用温远温,用热远热,食宜同法。有假者反常。反是者病,所谓时也。

帝曰:善。阳明之政奈何?

岐伯曰:卯酉之纪也。阳明,少角,少阴,清热胜复同,同正商。丁卯岁会,丁酉。其运风清热。少角(初正)、太徵、少宫、太商、少羽(终)。

阳明,少徵,少阴,寒雨胜复同,同正商。癸卯(同岁会),癸酉(同岁会)。其运热寒雨。少徵、太宫、少商、太羽(终)、太角(初)。

阳明,少宫,少阴,风凉胜复同。己卯,己酉。其运雨风凉。少宫、太商、少羽(终)、少角(初)、太徵。

阳明,少商,少阴,热寒胜复同,同正商。乙卯天符,乙酉岁会、太一天符,其运凉热寒。少商、太羽(终)、太角(初)、少徵、太宫。

阳明,少羽,少阴,雨风胜复同。辛卯,少宫同。辛酉、辛卯,其运寒雨风。少羽(终)、少角(初)、太徵、少宫、太商。

凡此阳明司天之政,气化运行后天,天气急,地气明,阳专其令,炎暑大行,物燥以坚,淳风乃治,风燥横逆,流于气交,多阳少阴,云趋雨府,湿化乃敷,燥极而泽。其谷白、丹,间谷命太者,其耗白甲品羽,金火合德,上应太白、荧惑。其政切,其令暴,蛰虫乃见,流水不冰。民病咳、嗌塞、寒热、发暴、振栗、癃闭。清先而劲,毛虫乃死,热后而暴,介虫乃殃,其发暴,胜复之作,扰而大乱,清热之气,持于气交。

初之气,地气迁,阴始凝,气始肃,水乃冰,寒雨化。其病中,热胀、面目浮肿、善眠、鼽、衂、嚏、欠、呕、小便黄赤甚则淋。二之气,阳乃布,民乃舒,物乃生荣。厉大至,民善暴死。三之气,天政布,凉乃行,燥热交合,燥极而泽。民病寒热。四之气,寒雨降,病暴仆、振栗、谵妄、少气、嗌干引饮,及为心痛、痈肿、疮疡、疟寒之疾、骨痿、血便。五之气,春令反行,草乃生荣。民气和。终之气,阳气布,候反温,蛰虫来见,流水不冰。民乃康平,其病温。

故食岁谷以安其气,食间谷以去其邪,岁宜以咸、以苦、以辛,汗之清之、散之,安其运气,无使受邪,折其郁气,资其化源。以寒热轻重少多其制,同热者多天化,同清者多地化。用凉远凉,用热远热,用寒远寒,用温远温,食宜同法。有假者反之,此其道也。反是者,乱天地之经,扰阴阳之纪也。

帝曰:善。少阳之政奈何?

岐伯曰:寅申之纪也。

少阳,太角,厥阴。壬寅(同天符),壬申(同天符)。其运风鼓,其化鸣紊启坼,其变振拉摧拔,其病掉、眩、支胁、惊骇。太角(初正)、少徵、太宫、少商、太羽(终)。

少阳,太徵,厥阴。戊寅天符,戊申天符。其运暑,其化暄嚣郁燠,其变炎烈沸腾,其病上热郁、血溢、血泄、心痛。太徵、少宫、太商、少羽(终)、少角(初)。

少阳,太宫,厥阴。甲寅,甲申。其运阴雨,其化柔润重泽,其变震惊飘骤,其病体重、胕肿、痞饮。太宫、少商、太羽(终)、太角(初)、少徵。

少阳,太商,厥阴。庚寅,庚申,同正商。其运凉,其化雾露清切,其变肃杀凋零,其病肩背胷中。太商、少羽(终)、少角(初)、太徵、少宫。

少阳,太羽,厥阴。丙寅,丙申。其运寒肃,其化凝惨凛冽,其变冰雪霜雹,其病寒,浮肿。太羽(终)、太角(初)、少徵、太宫、少商。

凡此少阳司天之政,气化运行先天,天气正,地气扰,风乃暴举,木偃沙飞,炎火乃流,阴行阳化,雨乃时应,火木同德,上应荧惑、岁星。其谷丹、苍,其政严,其令扰。故风热参布,云物沸腾,太阴横流,寒乃时至,凉雨并起。民病寒中,外发疮疡,内为泄满,故圣人遇之,和而不争。往复之作,民病寒热、疟、泄、聋、瞑、呕吐、上怫肿色变。

初之气,地气迁,风胜乃摇,寒乃去,候乃大温,草木早荣。寒来不杀,温病乃起,其病气怫于上,血溢、目赤、咳、逆、头痛、血崩、胁满、肤腠中疮。二之气,火反郁,白埃四起,云趋雨府,风不胜湿,雨乃零,民乃康。其病热郁于上,咳、逆、呕吐、疮发于中、胷嗌不利、头痛、身热、昏愦、脓疮。三之气,天政布,炎暑至,少阳临上,雨乃涯。民病热中,聋、瞑、血溢、脓疮、咳、呕、鼽、衂、渴、嚏、欠、喉痹、目赤,善暴死。四之气,凉乃至,炎暑间化,白露降,民气和平。其病满、身重。五之气,阳乃去,寒乃来,雨乃降,气门乃闭,刚木早凋。民避寒邪,君子周密。终之气,地气正,风乃至,万物反生,霿雾以行。其病关闭不禁、心痛、阳气不藏而咳。

抑其运气,赞所不胜,必折其郁气,先取化源,暴过不生,苛疾不起,故岁宜咸、宜辛、宜酸,渗之泄之;渍之发之。观气寒温以调其过,同风热者多寒化,异风热者少寒化。用热远热,用温远温,用寒远寒,用凉远凉,食宜同法,此其道也。有假者反之。反是者病之阶也。

帝曰:善。太阴之政奈何?

岐伯曰:丑未之纪也。

太阴,少角,太阳,清热胜复同,同正宫。丁丑,丁未。其运风清热。少角(初正)、太徵、少宫、太商、少羽(终)。

太阴,少徵,太阳,寒雨胜复同。癸丑,癸未。其运热寒雨。少徵、太宫、少商、太羽(终)、太角(初)。

太阴,少宫,太阳,风清胜复同,同正宫。己丑太一天符,己未太一天符。其运雨风清。少宫、太商、少羽(终)、少角(初)、太徵。

太阴,少商,太阳,热寒胜复同。乙丑,乙未。其运凉热寒。少商、太羽(终)、太角(初)、少徵、太宫。

太阴,少羽,太阳,雨风胜复同,同正宫。辛丑(同岁会),辛未(同岁会)。其运寒雨风。少羽(终)、少角(初)、太徵、少宫、太商。

凡此太阴司天之政,气化运行后天,阴专其政,阳气退避,大风时起,天气下降,地气上腾,原野昏霿,白埃四起,云奔南极,寒雨数至,物成于差夏。民病寒湿,腹满、身䐜愤、胕肿、痞逆、寒厥、拘急。湿寒合德,黄黑埃昏,流行气交,上应镇星、辰星。其政肃,其令寂,其谷黅、玄。故阴凝于上,寒积于下,寒水胜火,则为冰雹,阳光不治,杀气乃行。故有余宜高,不及宜下,有余宜晚,不及宜早,土之利,气之化也,民气亦从之,间谷命其太也。

初之气,地气迁,寒乃去,春气正,风乃来,生布万物以荣,民气条舒,风湿相薄,雨乃后。民病血溢、筋络拘强、关节不利、身重、筋痿。二之气,大火正,物承化,民乃和。其病温、厉大行,远近咸若。湿蒸相薄,雨乃时降。三之气,天政布,湿气降,地气腾,雨乃时降,寒乃随之。感于寒湿,则民病身重、胕肿、胷腹满。四之气,畏火临,溽蒸化,地气腾,天气否隔,寒风晓暮,蒸热相薄,草木凝烟,湿化不流,则白露阴布,以成秋令。民病腠理热、血暴溢、疟、心腹满热、胪胀,甚则胕肿。五之气,惨令已行,寒露下,霜乃早降,草木黄落,寒气及体,君子周密,民病皮腠。终之气,寒大举,湿大化,霜乃积,阴乃凝,水坚冰,阳光不治。感于寒,则病人关节禁固、腰脽痛,寒湿持于气交而为疾也。

必折其郁气而取化源,益其岁气无使邪胜。食岁谷以全其真,食间谷以保其精。故岁宜以苦燥之、温之,甚者发之、泄之。不发不泄则湿气外溢,肉溃皮拆而水血交流。必赞其阳火,令御甚寒,从气异同,少多其判也。同寒者以热化,同湿者以燥化,异者少之,同者多之。用凉远凉,用寒远寒,用温远温,用热远热,食宜同法。假者反之。此其道也,反是者病也。

帝曰:善。少阴之政奈何?

岐伯曰:子午之纪也。

少阴、太角、阳明。壬子、壬午。其运风鼓,其化鸣紊启拆(别本作坼),其变振拉摧拔,其病支满。太角(初正)、少徵、太宫、少商、太羽(终)。

少阴、太徵、阳明。戊子天符、戊午太一天符。其运炎暑,其化暄曜郁燠,其变灸烈沸腾,其病上热血溢。太徵、少宫、太商、少羽(终)、少角(初)。

少阴、太宫、阳明。甲子、甲午。其运阴雨,其化柔润(别本作顺)时雨,其变震惊飘骤,其病中满身重。太宫、少商、太羽(终)、太角(初)、少徵。

少阴、太商、阳明。庚子(同天符)、庚午(同天符),同正商。其运凉劲,其化雾露萧飋,其变肃杀凋零,其病下清。太商、少羽(终)、少角(初)、太徵、少宫。

少阴、太羽、阳明。丙子岁会、丙午。其运寒,其化凝惨栗冽,其变冰雪霜雹,其病寒下。太羽(终)、太角(初)、少徵、太宫、少商。

凡此少阴司天之政,气化运行先天,地气肃,天气明,寒交暑,热加燥,云驰雨府,湿化乃行,时雨乃降,金火合德,上应荧惑、太白。其政明,其令切,其谷丹、白。水火寒热,持于气交而为病始也,热病生于上,清病生于下,寒热凌犯而争于中,民病咳、喘、血溢、血泄、鼽、嚏、目赤、眦疡、寒厥入胃、心痛、腰痛、腹大、嗌干、肿上。

初之气,地气迁,燥将去,寒乃始,蛰复藏,水乃冰,霜复降,风乃至,阳气郁。民反周密,关节禁固,腰脽痛。炎暑将起,中外疮疡。二之气,阳气布,风乃行,春气以正,万物应荣,寒气时至。民乃和,其病淋、目瞑、目赤、气郁于上而热。三之气,天政布,大火行,庶类蕃鲜,寒气时至。民病气厥心痛,寒热更作,咳、喘、目赤。四之气,溽暑至,大雨时行,寒热互至。民病寒热,嗌干、黄瘅、鼽、衂、饮发。五之气,畏火临,暑反至,阳乃化,万物乃生、乃长荣,民乃康。其病温。终之气,燥令行,余火内格,肿于上、咳、喘,甚则血溢。寒气数举,则霿雾翳,病生皮腠,内舍于胁,下连少腹而作寒中。地将易也。

必抑其运气,资其岁胜,折其郁发,先取化源,无使暴过而生其病也。食岁谷以全真气,食间谷以避虚邪,岁宜咸以软之而调其上,甚则以苦发之以酸收之而安其下,甚则以苦泄之。适气同异而多少之,同天气者以寒清化,同地气者以温热化。用热远热,用凉远凉,用温远温,用寒远寒,食宜同法。有假则反。此其道也,反是者病作矣。

帝曰:善。厥阴之政奈何?

岐伯曰:巳亥之纪也。

厥阴、少角、少阳,清热胜复同,同正角。丁巳天符、丁亥天符。其运风清热。少角(初正)、太徵、少宫、太商、少羽(终)。

厥阴、少徵、少阳,寒雨胜复同。癸巳(同岁会)、癸亥(同岁会)。其运热寒雨。少徵、太宫、少商、太羽(终)、太角(初)。

厥阴、少宫、少阳,风清胜复同,同正角。己巳、己亥。其运雨风清。少宫、太商、少羽(终)、少角(初)、太徵。

厥阴、少商、少阳,热寒胜复同,同正角。乙巳、乙亥。其运凉热寒。少商、太羽(终)、太角(初)、少徵、太宫。

厥阴、少羽、少阳,雨风胜复同。辛巳、辛亥。其运寒雨风。少羽(终)、少角(初)、太徵、少宫、太商。

凡此厥阴司天之政,气化运行后天,诸同正岁,气化运行同天。天气扰,地气正,风生高远,炎热从之,云趋雨府,湿化乃行,风火同德,上应岁星、荧惑。其政挠,其令速,其谷苍、丹,间谷言太者,其耗文角、品羽。风燥火热,胜复更作,蛰虫来见,流水不冰,热病行于下,风病行于上,风燥胜复形于中。

初之气,寒始肃,杀气方至。民病寒于右之下。二之气,寒不去,华雪水冰,杀气施化,霜乃降,名草上焦,寒雨数至,阳复化。民病热于中。三之气,天政布,风乃时举。民病泣出、耳鸣、掉、眩。四之气,溽暑湿热相薄,争于左之上。民病黄瘅而为胕肿。五之气,燥湿更胜,沉阴乃布,寒气及体,风雨乃行。终之气,畏火司令,阳乃大化,蛰虫出见,流水不冰,地气大发,草乃生,人乃舒,其病温、厉。

必折其郁气,资其化源,赞其运气,无使邪胜。岁宜以辛调上,以咸调下,畏火之气,无妄犯之。用温远温,用热远热,用凉远凉,用寒远寒,食宜同法。有假反常。此之道也,反是者病。

帝曰:善。夫子之言,可谓悉矣,然何以明其应乎?

岐伯曰:昭乎哉问也!夫六气者,行有次,止有位,故常以正月朔日平旦视之,覩其位而知其所在矣。运有余,其至先,运不及,其至后,此天之道,气之常也。运非有余,非不足,是谓正岁,其至当其时也。

帝曰:胜复之气,其常在也,灾眚时至,候也奈何?

岐伯曰:非气化者,是谓灾也。

帝曰:天地之数,终始奈何?

岐伯曰:悉乎哉问也!是明道也。数之始,起于上而终于下,岁半之前天气主之,岁半之后地气主之,上下交互气交主之,岁纪毕矣。故曰:“位明气月,可知乎。”所谓气也。

帝曰:余司其事,则而行之,不合其数,何也?

岐伯曰:气用有多少,化洽有盛衰,衰盛多少,同其化也。

帝曰:愿闻同化何如?

岐伯曰:风温春化同,热曛昏火夏化同,胜与复同,燥清烟露秋化同,云雨昏暝埃长夏化同,寒气霜雪冰冬化同,此天地五运六气之化,更用盛衰之常也。

帝曰:五运行同天化者,命曰天符,余知之矣。愿闻同地化者何谓也?

岐伯曰:太过而同天化者三,不及而同天化者亦三;太过而同地化者三,不及而同地化者亦三。此凡二十四岁也。

帝曰:愿闻其所谓也?

岐伯曰:甲辰、甲戌,太宫,下加太阴;壬寅、壬申,太角,下加厥阴;庚子、庚午,太商,下加阳明,如是者三。癸巳、癸亥,少徵,下加少阳;辛丑、辛未,少羽,下加太阳;癸卯、癸酉,少徵,下加少阴,如是者三。戊子、戊午,太徵,上临少阴;戊寅、戊申,太徵,上临少阳;丙辰、丙戌,太羽,上临太阳,如是者三。丁巳、丁亥,少角,上临厥阴;乙卯、乙酉,少商,上临阳明;己丑、己未,少宫,上临太阴,如是者三。除此二十四岁,则不加不临也。

帝曰:加者何谓?

岐伯曰:太过而加同天符,不及而加同岁会也。

帝曰:临者何谓?

岐伯曰:太过不及,皆曰天符,而变行有多少,病形有微甚,生死有早晏耳。

帝曰:夫子言用寒远寒,用热远热,余未知其然也,愿闻何谓远?

岐伯曰:热无犯热,寒无犯寒,从者和,逆者病,不可不敬畏而远之,所谓时兴六位也。

帝曰:温凉何如?

岐伯曰:司气以热,用热无犯;司气以寒,用寒无犯;司气以凉,用凉无犯;司气以温,用温无犯。间气同其主无犯,异其主则小犯之,是谓四畏,必谨察之。

帝曰:善。其犯者何如?

岐伯曰:天气反时则可依时,及胜其主则可犯,以平为期而不可过,是谓邪气反胜者。故曰:“无失天信,无逆气宜,无翼其胜,无赞其复,是谓至治。”

帝曰:善。五运气行,主岁之纪,其有常数乎?

岐伯曰:臣请次之。

甲子、甲午岁:上少阴火,中太宫土运,下阳明金。热化二,雨化五,燥化四。所谓正化日也。其化上咸寒,中苦热,下酸热,所谓药食宜也。

乙丑、乙未岁:上太阴土,中少商金运,下太阳水,热化寒化胜复同。所谓邪气化日也。灾七宫。湿化五,清化四,寒化六,所谓正化日也。其化上苦热,中酸和,下甘热,所谓药食宜也。

丙寅、丙申岁:上少阳相火,中太羽水运,下厥阴木。火化二,寒化六,风化三,所谓正化日也。其化上咸寒,中咸温,下辛温,所谓药食宜也。

丁卯(岁会)、丁酉岁:上阳明金,中少角木运,下少阴火,清化热化胜复同。所谓邪气化日也。灾三宫。燥化九,风化三,热化七,所谓正化日也。其化上苦小温,中辛和,下咸寒,所谓药食宜也。

戊辰、戊戌岁:上太阳水,中太徵火运,下太阴土。寒化六,热化七,湿化五,所谓正化日也。其化上苦温,中甘和,下甘温,所谓药食宜也。

己巳、己亥岁:上厥阴木,中少宫土运,下少阳相火,风化清化胜复同。所谓邪气化日也。灾五宫。风化三,湿化五,火化七,所谓正化日也。其化上辛凉,中甘和,下咸寒,所谓药食宜也。

庚午(同天符)、庚子岁(同天符):上少阴火,中太商金运,下阳明金。热化七,清化九,燥化九,所谓正化日也。其化上咸寒,中辛温酸温,所谓药食宜也。

辛未(同岁会)、辛丑岁(同岁会):上太阴土,中少羽水运,下太阳水,雨化风化胜复同。所谓邪气化日也。灾一宫。雨化五,寒化一,所谓正化日也。其化上苦热,中苦和,下苦热,所谓药食宜也。

壬申(同天符)、壬寅岁(同天符):上少阳相火,中太角木运,下厥阴木。火化二,风化八,所谓正化日也。其化上咸寒,中酸和,下辛凉,所谓药食宜也。

癸酉(同岁会)、癸卯岁(同岁会):上阳明金,中少徵火运,下少阴火,寒化雨化胜复同。所谓邪气化日也。灾九宫。燥化九,热化二,所谓正化日也。其化上苦小温,中咸温,下咸寒,所谓药食宜也。

甲戌(岁会、同天符)、甲辰岁(岁会、同天符):上太阳水,中太宫土运,下太阴土。寒化六,湿化五,正化日也。其化上苦热,中苦温,下苦温,药食宜也。

乙亥、乙巳岁:上厥阴木,中少商金运,下少阳相火,热化寒化胜复同。邪气化日也。灾七宫。风化八,清化四,火化二,正化度也。其化上辛凉,中酸和,下咸寒,药食宜也。

丙子(岁会)、丙午岁:上少阴火,中太羽水运,下阳明金。热化二,寒化六,清化四,正化度也。其化上咸寒,中咸热,下酸温,药食宜也。

丁丑、丁未岁:上太阴土,中少角木运,下太阳水,清化热化胜复同。邪气化度也。灾三宫。雨化五,风化三,寒化一,正化度也。其化上苦温,中辛温,下甘热,药食宜也。

戊寅(天符)、戊申岁(天符):上少阳相火,中太徵火运,下厥阴木。火化七,风化三,正化度也。其化上咸寒,中甘和,下辛凉,药食宜也。

己卯、己酉岁:上阳明金,中少宫土运,下少阴火,风化清化胜复同。邪气化度也。灾五宫。清化九,雨化五,热化七,正化度也。其化上苦小温,中甘和,下咸寒,药食宜也。

庚辰、庚戌岁:上太阳水,中太商金运,下太阴土。寒化一,清化九,雨化五,正化度也。其化上苦热,中辛温,下甘热,药食宜也。

辛巳、辛亥岁:上厥阴木,中少羽水运,下少阳相火,雨化风化胜复同。邪气化度也。灾一宫。风化三,寒化一,火化七,正化度也。其化上辛凉,中苦和,下咸寒,药食宜也。

壬午、壬子岁:上少阴火,中太角木运,下阳明金。热化二,风化八,清化四,正化度也。其化上咸寒,中酸凉,下酸温,药食宜也。

癸未、癸丑岁:上太阴土,中少徵火运,下太阳水,寒化雨化胜复同。邪气化度也。灾九宫。雨化五,火化二,寒化一,正化度也。其化上苦温,中咸温,下甘热,药食宜也。

甲申、甲寅岁:上少阳相火,中太宫土运,下厥阴木。火化二,雨化五,风化八,正化度也。其化上咸寒,中咸和,下辛凉,药食宜也。

乙酉(太一天符)、乙卯岁(天符):上阳明金,中少商金运,下少阴火,热化寒化胜复同。邪气化度也。灾七宫。燥化四,清化四,热化二,正化度也。其化上苦小温,中苦和,下咸寒,药食宜也。

丙戌(天符)、丙辰岁(天符):上太阳水,中太羽水运,下太阴土。寒化六,雨化五,正化度也。其化上苦热,中咸温,下甘热,药食宜也。

丁亥(天符)、丁巳岁(天符):上厥阴木,中少角木运,下少阳相火,清化热化胜复同。邪气化度也。灾三宫。风化三,火化七,正化度也。其化上辛凉,中辛和,下咸寒,药食宜也。

戊子(天符)、戊午岁(太一天符):上少阴火,中太徵火运,下阳明金。热化七,清化九,正化度也。其化上咸寒,中甘寒,下酸温,药食宜也。

己丑(太一天符)、己未岁(太一天符):上太阴土,中少宫土运,下太阳水,风化清化胜复同。邪气化度也。灾五宫。雨化五,寒化一,正化度也。其化上苦热,中甘和,下甘热,药食宜也。

庚寅、庚申岁:上少阳相火,中太商金运,下厥阴木。火化七,清化九,风化三,正化度也。其化上咸寒,中辛温,辛凉,药食宜也。

辛卯、辛酉岁:上阳明金,中少羽水运,下少阴火,雨化风化胜复同。邪气化度也。灾一宫。清化九,寒化一,热化七,正化度也。其化上苦小温,中苦和,下咸寒,药食宜也。

壬辰、壬戌岁:上太阳水,中太角木运,下太阴土。寒化六,风化八,雨化五,正化度也。其化上苦温,中酸温,下甘温,药食宜也。

癸巳(同岁会)、癸亥(岁同岁会):上厥阴木,中少徵火运,下少阳相火,寒化雨化胜复同。邪气化度也。灾九宫。风化八,火化二,正化度也。其化上辛凉,中咸和,下咸寒,药食宜也。

凡此定期之纪,胜复正化,皆有常数,不可不察。故知其要者,一言而终,不知其要,流散无穷,此之谓也。

帝曰:善。五运之气亦复岁乎?

岐伯曰:郁极乃发,待时而作也。

帝曰:请问其所谓也?

岐伯曰:五常之气,太过不及,其发异也。

帝曰:愿卒闻之。

岐伯曰:太过者暴,不及者徐,暴者为病甚,徐者为病持。

帝曰:太过不及,其数何如?

岐伯曰:太过者其数成,不及者其数生,土常以生也。

帝曰:其发也何如?

岐伯曰:土郁之发,岩谷震惊,雷殷气交,埃昏黄黑,化为白气,飘骤高深,击石飞空,洪水乃从,川流漫衍,田牧土驹。化气乃敷,善为时雨,始生始长,始化始成。故民病心腹胀,肠鸣而为数后,甚则心痛,胁䐜,呕吐,霍乱,饮发,注下,胕肿,身重。云奔雨府,霞拥朝阳,山泽埃昏,其乃发也,以其四气。云横天山,浮游生灭,怫之先兆。

金郁之发,天洁地明,风清气切,大凉乃举,草树浮烟,燥气以行,霿雾数起,杀气来至,草木苍干,金乃有声。故民病咳,逆,心胁满引少腹,善暴痛,不可反侧,嗌干,面陈色恶。山泽焦枯,土凝霜卤,怫乃发也,其气五。夜零白露,林莽声凄,怫之兆也。

水郁之发,阳气乃辟,阴气暴举,大寒乃至,川泽严凝,寒雾结为霜雪,甚则黄黑昏翳,流行气交,乃为霜杀,水乃见祥。故民病寒客,心痛,腰脽痛,大关节不利,屈伸不便,善厥逆,痞坚,腹满。阳光不治,空积沉阴,白埃昏暝而乃发也。其气二火前后。太虚深玄,气犹麻散,微见而隐,色黑微黄,怫之先兆也。

木郁之发,太虚埃昏,云物以扰,大风乃至,屋发折木,木有变。故民病胃脘当心而痛、上支两胁、鬲咽不通、食饮不下,甚则耳鸣,眩转,目不识人,善暴僵仆。太虚苍埃,天山一色,或气浊色,黄黑郁若,横云不起雨,而乃发也,其气无常。长川草偃,柔叶呈阴,松吟高山,虎啸岩岫,怫之先兆也。

火郁之发,太虚曛翳,大明不彰,炎火行,大暑至,山泽燔燎,材木流津,广厦腾烟,土浮霜卤,止水乃减,蔓草焦黄,风行惑言,湿化乃后。故民病少气,疮疡,痈肿,胁腹、胷背、面首、四肢䐜愤,胪胀,疡疿,呕逆,瘈瘲,骨痛,节乃有动,注下,温疟,腹中暴痛,血溢流注,精液乃少,目赤,心热,甚则瞀闷懊憹,善暴死。刻终大温,汗濡玄府,其乃发也,其气四。动复则静,阳极反阴,湿令乃化乃成,华发水凝,山川冰雪,焰阳午泽,怫之先兆也。

有怫之应而后报也,皆观其极而乃发也。木发无时,水随火也。谨候其时,病可与期,失时反岁,五气不行,生化收藏,政无恒也。

帝曰:水发而雹雪,土发而飘骤,木发而毁折,金发而清明,火发而曛昧,何气使然?

岐伯曰:气有多少,发有微甚,微者当其气,甚者兼其下,徵其下气而见可知也。

帝曰:善。五气之发,不当位者,何也?

岐伯曰:命其差。

帝曰:差有数乎?

岐伯曰:后皆三十度而有奇也。

帝曰:气至而先后者何?

岐伯曰:运太过则其至先,运不及则其至后,此候之常也。

帝曰:当时而至者,何也?

岐伯曰:非太过非不及则至当时,非是者眚也。

帝曰:善。气有非时而化者,何也?

岐伯曰:太过者当其时,不及者归其己胜也。

帝曰:四时之气,至有早晏高下左右,其候何如?

岐伯曰:行有逆顺,至有迟速,故太过者化先天,不及者化后天。

帝曰:愿闻其行何谓也?

岐伯曰:春气西行,夏气北行,秋气东行,冬气南行。故春气始于下,秋气始于上,夏气始于中,冬气始于标;春气始于左,秋气始于右,冬气始于后,夏气始于前,此四时正化之常。故至高之地,冬气常在,至下之地,春气常在,必谨察之。

帝曰:善。

黄帝问曰:五运六气之应见,六化之正,六变之纪,何如?

岐伯对曰:夫六气正纪,有化有变,有胜有复,有用有病,不同其候,帝欲何乎?

帝曰:愿尽闻之。

岐伯曰:请遂言之。夫气之所至也,厥阴所至为和平,少阴所至为暄,太阴所至为埃溽,少阳所至为炎暑,阳明所至为清劲,太阳所至为寒雰。时化之常也。

厥阴所至为风府、为璺启;少阴所至为火府、为舒荣;太阴所至为雨府、为员盈;少阳所至为热府、为行出;阳明所至为司杀府、为庚苍;太阳所至为寒府、为归藏。司化之常也。

厥阴所至为生、为风摇;少阴所至为荣、为形见;太阴所至为化、为云雨;少阳所至为长、为蕃鲜;阳明所至为收、为雾露;太阳所至为藏、为周密。气化之常也。

厥阴所至为风生,终为肃;少阴所至为热生,中为寒;太阴所至为湿生,终为注雨;少阳所至为火生,终为蒸溽;阳明所至为燥生,终为凉;太阳所至为寒生,中为温。德化之常也。

厥阴所至为毛化,少阴所至为翮化,太阴所至为倮化,少阳所至为羽化,阳明所至为介化,太阳所至为鳞化。德化之常也。

厥阴所至为生化,少阴所至为荣化,太阴所至为濡化,少阳所至为茂化,阳明所至为坚化,太阳所至为藏化。布政之常也。

厥阴所至为飘怒大凉,少阴所至为大暄寒,太阴所至为雷霆骤注烈风,少阳所至为飘风燔燎霜凝,阳明所至为散落温,太阳所至为寒雪冰雹白埃。气变之常也。

厥阴所至为挠动、为迎随;少阴所至为高明焰、为曛;太阴所至为沉阴、为白埃、为晦暝;少阳所至为光显、为彤云、为曛;阳明所至为烟埃、为霜、为劲切、为凄鸣;太阳所至为刚固、为坚芒、为立。令行之常也。

厥阴所至为里急,少阴所至为疡胗、身热,太阴所至为积饮否隔,少阳所至为嚏、呕、为疮疡,阳明所至为浮虚,太阳所至为屈伸不利。病之常也。

厥阴所至为支痛,少阴所至为惊惑、恶寒、战栗、谵妄,太阴所至为稸满,少阳所至为惊躁、瞀昧、暴病,阳明所至为鼽、尻阴股膝髀腨胻足病,太阳所至为腰痛。病之常也。

厥阴所至为緛戾,少阴所至为悲、妄、衂、蔑,太阴所至为中满、霍乱、吐下,少阳所至为喉痹、耳鸣、呕涌,阳明所至为皴揭,太阳所至为寝汗痉。病之常也。

厥阴所至为胁痛、呕泄,少阴所至为语、笑,太阴所至为重、胕肿,少阳所至为暴注、瞤、瘛、暴死,阳明所至为鼽、嚏,太阳所至为流泄、禁止。病之常也。

凡此十二变者,报德以德,报化以化,报政以政,报令以令,气高则高,气下则下,气后则后,气前则前,气中则中,气外则外,位之常也。故风胜则动,热胜则肿,燥胜则干,寒胜则浮,湿胜则濡泄甚则水闭胕肿,随气所在,以言其变耳。

帝曰:愿闻其用也。

岐伯曰:夫六气之用,各归不胜而为化,故太阴雨化,施于太阳;太阳寒化,施于少阴;少阴热化,施于阳明;阳明燥化,施于厥阴;厥阴风化,施于太阴。各命其所在以征之也。

帝曰:自得其位何如?

岐伯曰:自得其位常化也。

帝曰:愿闻所在也。

岐伯曰:命其位而方月可知也。

帝曰:六位之气,盈虚何如?

岐伯曰:太少异也。太者之至徐而常,少者暴而亡。

帝曰:天地之气盈虚如何?

岐伯曰:天气不足,地气随之;地气不足,天气从之,运居其中而常先也。恶所不胜,归所同和,随运归从而生其病也。故上胜则天气降而下,下胜则地气迁而上。多少而差其分,微者小差,甚者大差,甚则位易气交,易则大变生而病作矣。《大要》曰:“甚纪五分,微纪七分,其差可见。”此之谓也。

帝曰:善。论言热无犯热,寒无犯寒,余欲不远寒,不远热,奈何?

岐伯曰:悉乎哉问也!发表不远热,攻里不远寒。

帝曰:不发不攻,而犯寒犯热何如?

岐伯曰:寒热内贼,其病益甚。

帝曰:愿闻无病者何如?

岐伯曰:无者生之,有者甚之。

帝曰:生者何如?

岐伯曰:不远热则热至,不远寒则寒至。寒至则坚否、腹满、痛急、下利之病生矣。热至则身热、吐下、霍乱、痈疽、疮疡、瞀郁、注下、瞤、瘛、肿胀、呕、鼽、衂、头痛、骨节变、肉痛、血溢、血泄、淋闭之病生矣。

帝曰:治之奈何?

岐伯曰:时必顺之,犯者治以胜也。

黄帝问曰:妇人重身,毒之何如?

岐伯曰:有故无殒,亦无殒也。

帝曰:愿闻其故,何谓也?

岐伯曰:大积大聚,其可犯也,衰其大半而止,过者死。

帝曰:善。郁之甚者,治之奈何?

岐伯曰:木郁达之,火郁发之,土郁夺之,金郁泄之,水郁折之,然调其气,过者折之,以其畏也,所谓泻之。

帝曰:假者何如?

岐伯曰:有假其气则无禁也。所谓:“主气不足,客气胜也。”

帝曰:至哉圣人之道!天地大化,运行之节,临御之纪,阴阳之政,寒暑之令,非夫子孰能通之!请藏之灵兰之室,署曰《六元正纪》,非斋戒不敢示,慎传也。
\section{刺法论}
%%	72	%%
黄帝问曰:升降不前,气交有变,即成暴郁,余已知之。何如预救生灵,可得却乎?

岐伯稽首再拜对曰:昭乎哉问!臣闻夫子言,既明天元,须穷刺法,可以折郁扶运,补弱全真,泻盛蠲余,令除斯苦。

帝曰:愿卒闻之。

岐伯曰:升之不前,即有甚凶也。木欲升而天柱窒抑之,木欲发郁亦须待时,当刺足厥阴之井。火欲升而天蓬窒抑之,火欲发郁亦须待时,君火相火同刺包络之荥。土欲升而天冲窒抑之,土欲发郁亦须待时,当刺足太阴之俞。金欲升而天英窒抑之,金欲发郁亦须待时,当刺手太阴之经。水欲升而天芮窒抑之,水欲发郁亦须待时,当刺足少阴之合。

帝曰:升之不前,可以预备。愿闻其降,可以先防。

岐伯曰:既明其升,必达其降也。升降之道,皆可先治也。木欲降而地皛窒抑之,降而不入,抑之郁发,散而可得位,降而郁发,暴如天间之待时也,降而不下,郁可速矣,降可折其所胜也。当刺手太阴之所出,刺手阳明之所入。火欲降而地玄窒抑之,降而不入,抑之郁发,散而可矣,当折其所胜,可散其郁。当刺足少阴之所出,刺足太阳之所入。土欲降而地苍窒抑之,降而不下,抑之郁发,散而可入,当折其胜,可散其郁。当刺足厥阴之所出,刺足少阳之所入。金欲降而地彤窒抑之,降而不下,抑之郁发,散而可入,当折其胜,可散其郁。当刺心包络所出,刺手少阳所入也。水欲降而地阜窒抑之,降而不下,抑之郁发,散而可入,当折其土,可散其郁。当刺足太阴之所出,刺足阳明之所入。

帝曰:五运之至,有前后与升降往来,有所承抑之,可得闻刺法乎?

岐伯曰:当取其化源也。是故太过取之,不及资之。太过取之,次抑其郁,取其运之化源,令折郁气,不及扶资,以扶运气,以避虚邪也。资取之法,令出《密语》。

黄帝问曰:升降之刺,以知其要。愿闻司天未得迁正,使司化之失其常政,即万化之或其皆妄。然与民为病,可得先除。欲济羣生,愿闻其说。

岐伯稽首再拜曰:悉乎哉问!言其至理。圣念慈悯,欲济群生,臣乃尽陈斯道,可申洞微。太阳复布,即厥阴不迁正,不迁正气塞于上,当泻足厥阴之所流。厥阴复布,少阴不迁正,不迁正即气塞于上,当刺心包络脉之所流。少阴复布,太阴不迁正,不迁正即气留于上,当刺足太阴之所流。太阴复布,少阳不迁正,不迁正则气塞未通,当刺手少阳之所流。少阳复布,则阳明不迁正,不迁正则气未通上,当刺手太阴之所流。阳明复布,太阳不迁正,不迁正则复塞其气,当刺足少阴之所流。

帝曰:迁正不前,以通其要。愿闻不退,欲折其余,无令过失,可得明乎?

岐伯曰:气过有余,复作布正,是名不退位也。使地气不得后化,新司天未可迁正,故复布化令如故也。巳亥之岁,天数有余,故厥阴不退位也。风行于上,木化布天,当刺足厥阴之所入。子午之岁,天数有余,故少阴不退位也。热行于上,火余化布天,当刺手厥阴之所入。丑未之岁,天数有余,故太阴不退位也。湿行于上,雨化布天,当刺足太阴之所入。寅申之岁,天数有余,故少阳不退位也。热行于上,火化布天,当刺手少阳之所入。卯酉之岁,天数有余,故阳明不退位也。金行于上,燥化布天,当刺手太阴之所入。辰戌之岁,天数有余,故太阳不退位也。寒行于上,凛水化布天,当刺足少阴之所入。故天地气逆,化成民病,以法刺之,预可平疴。

黄帝问曰:刚柔二干,失守其位,使天运之气皆虚乎?与民为病,可得平乎?

岐伯曰:深乎哉问!明其奥旨,天地迭移,三年化疫,是谓根之可见,必有逃门。

假令甲子,刚柔失守,刚未正,柔孤而有亏,时序不令,即音律非从,如此三年,变大疫也。详其微甚,察其浅深,欲至而可刺,刺之。当先补肾俞,次三日,可刺足太阴之所注。又有下位己卯不至,而甲子孤立者,次三年作土疠,其法补泻,一如甲子同法也。其刺以毕,又不须夜行及远行,令七日洁,清净斋戒,所有自来。肾有久病者,可以寅时面向南,净神不乱思,闭气不息七遍;以引颈咽气顺之,如咽甚硬物,如此七遍后,饵舌下津令无数。

假令丙寅,刚柔失守,上刚干失守,下柔不可独主之,中水运非太过,不可执法而定之,布天有余,而失守上正,天地不合,即律吕音异,如此即天运失序,后三年变疫。详其微甚,差有大小,徐至即后三年,至甚即首三年,当先补心俞,次五日,可刺肾之所入。又有下位地甲子辛巳柔不附刚,亦名失守,即地运皆虚,后三年变水疠,即刺法皆如此矣。其刺如毕,慎其大喜欲情于中,如不忌,即其气复散也,令静七日,心欲实,令少思。

假令庚辰,刚柔失守,上位失守,下位无合,乙庚金运,故非相招,布天未退,中运胜来,上下相错,谓之失守,姑洗林钟,商音不应也。如此则天运化易,三年变大疫。详其天数,差有微甚,微即微,三年至;甚即甚,三年至。当先补肝俞,次三日,可刺肺之所行。刺毕,可静神七日,慎勿大怒,怒必真气却散之。又或在下地甲子乙未失守者,即乙柔干,即上庚独治之,亦名失守者,即天运孤主之,三年变疠,名曰金疠,其至待时也。详其地数之等差,亦推其微甚,可知迟速耳。诸位乙庚失守,刺法同,肝欲平,即勿怒。

假令壬午,刚柔失守,上壬未迁正,下丁独然,即虽阳年,亏及不同,上下失守,相招其有期,差之微甚,各有其数也。律吕二角,失而不和,同音有日,微甚如见,三年大疫。当刺脾之俞,次三日,可刺肝之所出也。刺毕,静神七日,勿大醉歌乐,其气复散,又勿饱食,勿食生物。欲令脾实,气无滞饱,无久坐,食无太酸,无食一切生物,宜甘宜淡。又或地下甲子丁酉失守其位,未得中司,即气不当位,下不与壬奉合者,亦名失守,非名合德,故柔不附刚,即地运不合,三年变疠,其刺法一如木疫之法。

假令戊申,刚柔失守,戊癸虽火运,阳年不太过也,上失其刚,柔地独主,其气不正,故有邪干,迭移其位,差有浅深,欲至将合,音律先同,如此天运失时,三年之中,火疫至矣。当刺肺之俞。刺毕,静神七日,勿大悲伤也,悲伤即肺动,而其气复散也。人欲实肺者,要在息气也。又或地下甲子癸亥失守者,即柔失守位也,即上失其刚也,即亦名戊癸不相合德者也,即运与地虚,后三年变疠,即名火疠。

是故立地五年,以明失守,以穷法刺,于是疫之与疠,即是上下刚柔之名也,穷归一体也,即刺疫法,只有五法,是总其诸位失守,故只归五行而统之也。

黄帝曰:余闻五疫之至,皆相染易,无问大小,病状相似,不施救疗,如何可得不相移易者?

岐伯曰:不相染者,正气存内,邪不可干,避其毒气,天牝从来,复得其往,气出于脑,即不邪干。气出于脑,即室先想心如日,欲将入于疫室,先想青气自肝而出,左行于东,化作林木;次想白气自肺而出,右行于西,化作戈甲;次想赤气自心而出,南行于上,化作焰明;次想黑气自肾而出,北行于下,化作水;次想黄气自脾而出,存于中央,化作土。五气护身之毕,以想头上如北斗之煌煌,然后可入于疫室。

又一法,于春分之日,日未出而吐之。又一法,于雨水日后,三浴以药泄汗。又一法,小金丹方:辰砂二两,水磨雄黄一两,叶子雌黄一两,紫金半两,同入合中,外固,了地一尺筑地实,不用炉,不须药制,用火二十斤煅之也,七日终。候冷七日取,次日出合子,埋药地中七日取出。顺日研之三日,炼白沙蜜为丸,如梧桐子大,每日望东吸日华气一口,冰水下一丸,和气咽之,服十粒,无疫干也。

黄帝问曰:人虚即神游失守位,使鬼神外干,是致夭亡,何以全真?愿闻刺法。

岐伯稽首再拜曰:昭乎哉问!谓神移失守,虽在其体,然不致死,或有邪干,故令夭寿。只如厥阴失守,天以虚,人气肝虚,感天重虚,即魂游于上,邪干厥大气,身温犹可刺之,刺其足少阳之所过,复刺肝之俞。人病心虚,又遇君相二火司天失守,感而三虚,遇火不及,黑尸鬼犯之,令人暴亡,可刺手少阳之所过,复刺心俞。人脾病,又遇太阴司天失守,感而三虚,又遇土不及,青尸鬼邪犯之于人,令人暴亡,可刺足阳明之所过,复刺脾之俞。人肺病,遇阳明司天失守,感而三虚,又遇金不及,有赤尸鬼干人,令人暴亡,可刺手阳明之所过,复刺肺俞。人肾病,又遇太阳司天失守,感而三虚,又遇水运不及之年,有黄尸鬼干犯人正气,吸人神魂,致暴亡,可刺足太阳之所过,复刺肾俞。

黄帝问曰:十二脏之相使,神失位,使神彩之不圆,恐邪干犯,治之可刺,愿闻其要。

岐伯稽首再拜曰:悉乎哉问!至理道真宗,此非圣帝,焉究斯源,是谓气神合道,契符上天。心者,君主之官,神明出焉,可刺手少阴之源。肺者,相傅之官,治节出焉,可刺手太阴之源。肝者,将军之官,谋虑出焉,可刺足厥阴之源。胆者,中正之官,决断出焉,可刺足少阳之源。膻中者,臣使之官,喜乐出焉,可刺心包络所流。脾为谏议之官,知周出焉,可刺脾之源。胃为仓廪之官,五味出焉,可刺胃之源。大肠者传道之官,变化出焉,可刺大肠之源。小肠者受盛之官,化物出焉,可刺小肠之源。肾者作强之官,伎巧出焉,刺其肾之源。三焦者,决渎之官,水道出焉,刺三焦之源。膀胱者,州都之官,精液藏焉,气化则能出矣,刺膀胱之源。凡此十二官者,不得相失也。是故刺法有全神养真之旨,亦法有修真之道,非治疾也。故要修养和神也,道贵常存,补神固根,精气不散,神守不分,然即神守而虽不去,亦全真,人神不守,非达至真,至真之要,在乎天玄,神守天息,复入本元,命曰归宗。
\section{本病论}
%%	73	%%
黄帝问曰:天元九窒,余已知之。愿闻气交何名失守?

岐伯曰:谓其上下升降,迁正退位,各有经论,上下各有不前,故名失守也。是故气交失易位,气交乃变,变易非常,即四时失序,万化不安,变民病也。

帝曰:升降不前,愿闻其故,气交有变,何以明知?

岐伯曰:昭乎问哉!明乎道矣。气交有变,是为天地机,但欲降而不得降者,地窒刑之。又有五运太过,而先天而至者,即交不前,但欲升而不得其升,中运抑之,但欲降而不得其降,中运抑之。于是有升之不前,降之不下者;有降之不下,升而至天者;有升降俱不前,作如此之分别,即气交之变,变之有异,常各各不同,灾有微甚者也。

帝曰:愿闻气交遇会胜抑之由,变成民病,轻重何如?

岐伯曰:胜相会,抑伏使然。是故辰戌之岁,木气升之,主逢天柱,胜而不前。又遇庚戌,金运先天,中运胜之,忽然不前。木运升天,金乃抑之,升而不前,即清生风少,肃杀于春,露霜复降,草木乃萎。民病瘟疫早发,咽嗌乃干,四肢满,肢节皆痛。久而化郁,即大风摧拉,折陨鸣紊。民病卒中,偏痹,手足不仁。

是故巳亥之岁,君火升天,主窒天蓬,胜之不前。又厥阴未迁正,则少阴未得升天,水运以至其中者。君火欲升,而中水运抑之,升之不前,即清寒复作,冷生旦暮。民病伏阳而内生烦热,心神惊悸,寒热间作。日久成郁,即暴热乃至,赤风肿翳,化疫,温疠暖作,赤气彰而化火疫,皆烦而躁渴,渴甚治之,以泄之可止。

是故子午之岁,太阴升天,主窒天冲,胜之不前,又或遇壬子,木运先天而至者,中木运抑之也,升天不前,即风埃四起,时举埃昏,雨湿不化。民病风厥、涎潮、偏痹不随、胀满。久而伏郁,即黄埃化疫也,民病夭亡,脸支府黄疸满闭,湿令弗布,雨化乃微。

是故丑未之年,少阳升天,主窒天蓬,胜之不前。又或遇太阴未迁正者,即少阴未升天也,水运以至者。升天不前,即寒雰反布,凛冽如冬,水复涸,冰再结,喧暖乍作,冷复布之,寒暄不时。民病伏阳在内,烦热生中、心神惊骇、寒热间争。以久成郁,即暴热乃生,赤风气肿翳,化成郁疠,乃化作伏热内烦、痹而生厥,甚则血溢。

是故寅申之年,阳明升天,主窒天英。胜之不前,又或遇戊申戊寅,火运先天而至,金欲升天,火运抑之,升之不前,实时雨不降,西风数举,咸卤燥生。民病上热,喘、嗽,血溢。久而化郁,即白埃翳雾,清生杀气,民病胁满、悲伤、寒、鼽、嚏、嗌干、手坼、皮肤燥。

是故卯酉之年,太阳升天,主窒天芮,胜之不前。又遇阳明未迁正者,即太阳未升天也。土运以至,水欲升天,土运抑之,升之不前,即湿而热蒸,寒生两间。民病注下、食不及化。久而成郁,冷来客热,冰雹卒至。民病厥逆而哕,热生于内,气痹于外,足胫酸疼、反生心悸燠热、暴烦而复厥。

黄帝曰:升之不前,余已尽知其旨。愿闻降之不下,可得明乎?

岐伯曰:悉乎哉问也!是谓天地微旨,可以尽陈斯道。所谓升已必降也,至天三年,次岁必降,降而入地,始为左间也,如此升降往来,命之六纪也。

是故丑未之岁,厥阴降地,主窒地晶,胜而不前。又或遇少阴未退位,即厥阴未降下,金运以至中,金运承之,降之未下,抑之变郁,木欲降下,金承之,降而不下。苍埃远见,白气承之,风举埃昏,清燥行杀,霜露复下,肃杀布令。久而不降,抑之化郁,即作风燥相伏,暄而反清,草木萌动,杀霜乃下,蛰虫未见,惧清伤藏。

是故寅申之岁,少阴降地,主窒地玄,胜之不入。又或遇丙申、丙寅,水运太过,先天而至,君火欲降,水运承之,降而不下。即彤云才见,黑气反生,暄暖如舒,寒常布雪,凛冽复作,天云惨凄。久而不降,伏之化郁,寒胜复热,赤风化疫,民病面赤、心烦、头痛、目眩也。赤气彰而温病欲作也。

是故卯酉之岁,太阴降地,主窒地苍,胜之不入。又或少阳未退位者,即太阴未得降也;或木运以至,木运承之,降而不下。即黄云见而青霞彰,郁蒸作而大风,雾翳埃胜,折损乃作。久而不降也,伏之化郁,天埃黄气,地布湿蒸,民病四肢不举、昏眩、支节痛、腹满、䐜臆。

是故辰戌之岁,少阳降地,主窒地玄,胜之不入。又或遇水运太过,先天而至也,水运承之,降而不下。即彤云才见,黑气反生,暄暖欲生,冷气卒至,甚即冰雹也。久而不降,伏之化郁,冷气复热,赤风化疫。民病面赤、心烦、头痛、目眩也。赤气彰而热病欲作也。

是故巳亥之岁,阳明降地,主窒地彤,胜而不入。又或遇太阳未退位,即阳明未得降,即火运以至之,火运承之,降而不下。即天清而肃,赤气乃彰,暄热反作。民皆昏倦,夜卧不安,咽干、引饮、懊热内烦。大清朝暮,暄还复作。久而不降,伏之化郁,天清薄寒,远生白气。民病掉、眩、手足直而不仁、两胁作痛、满目䀮䀮。

是故子午之年,太阳降地,主窒地阜胜之,降而不入。又或遇土运太过,先天而至,土运承之,降而不入。即天彰黑气,暝暗凄惨,才施黄埃而布湿,寒化令气,蒸湿复令。久而不降,伏之化郁。民病大厥、四肢重怠、阴痿、少力。天布沉阴,蒸湿间作。

帝曰:升降不前,晰知其宗。愿闻迁正,可得明乎?

岐伯曰:正司中位,是谓迁正位。司天不得其迁正者,即前司天以过交司之日,即遇司天太过有余日也,即仍旧治天数,新司天未得迁正也。厥阴不迁正,即风暄不时,花卉萎瘁,民病淋溲,目系转,转筋,喜怒,小便赤。风欲令而寒由不去,温暄不正,春正失时。少阴不迁正,即冷气不退,春冷后寒,暄暖不时。民病寒热,四肢烦痛、腰脊强直。木气虽有余,而位不过于君火也。太阴不迁正,即云雨失令,万物枯焦,当生不发。民病手足肢节肿满、大腹水肿、䐜臆、不食、飧泄、胁满、四肢不举。雨化欲令,热犹治之,温煦于气,亢而不泽。少阳不迁正,即炎灼弗令,苗莠不荣,酷暑于秋,肃杀晚至,霜露不时。民病痎疟、骨热、心悸、惊骇,甚时血溢。阳明不迁正,则暑化于前,肃杀于后,草木反荣。民病寒热,鼽、嚏、皮毛折、爪甲枯焦,甚则喘嗽息高,悲伤不乐。热化乃布,燥化未令,即清劲未行,肺金复病。太阳不迁正,即冬清反寒,易令于春,杀霜在前,寒冰于后,阳光复治,凛冽不作,雾云待时。民病温、疠至,喉闭、嗌干、烦躁而渴、喘息而有音也。寒化待燥,犹治天气,过失序,与民作灾。

帝曰:迁正早晚,以命其旨,愿闻退位,可得明哉?

岐伯曰:所谓不退者,即天数未终,即天数有余,名曰复布政故名曰再治天也,即天令如故而不退位也。厥阴不退位,即大风早举,时雨不降,湿令不化。民病温疫、疵、废;风生,民病皆肢节痛,头目痛,伏热内烦,咽喉干,引饮。少阴不退位,即温生春冬,蛰虫早至,草木发生。民病膈热、咽干、血溢、惊骇、小便赤涩、丹瘤疹、疮疡、留毒。太阴不退位,而取寒暑不时,埃昏布作,湿令不去。民病四肢少力、食饮不下、泄注、淋满、足胫寒、阴痿、闭塞、失溺、小便数。少阳不退位,即热生于春,暑乃后化,冬温不冻,流水不冰,蛰虫出见。民病少气、寒热更作、便血、上热、小腹坚满、小便赤沃,甚则血溢。阳明不退位,即春生清冷,草木晚荣,寒热间作。民病呕吐、暴注、食饮不下、大便干燥、四肢不举、目瞑、掉、眩。太阳不退位,即春寒夏作,冷雹乃降,沉阴昏翳,二之气寒犹不去。民病痹厥,阴痿,失溺,腰膝皆痛,温疠晚发。

帝曰:天岁早晚,余以知之,愿闻地数,可得闻乎?

岐伯曰:地下迁正、升天及退位不前之法,即地上产化,万物失时之化也。

帝曰:余闻天地二甲子,十干十二支。上下经纬天地,数有迭移,失守其位,可得昭乎?

岐伯曰:失之迭位者,谓虽得岁正,未得正位之司,即四时不节,即生大疫。

假令甲子阳年,土运太窒,如癸亥天数有余者,年虽交得甲子,厥阴犹尚治天;地已迁正,阳明在泉,去岁少阳以作右间,即厥阴之地阳明,故不相和奉者也。癸己相会,土运太过,虚反受木胜,故非太过也,何以言土运太过,况黄钟不应太窒,木既胜而金还复,金既复而少阴如至,即木胜如火而金复微,如此则甲己失守,后三年化成土疫,晚至丁卯,早至丙寅,土疫至也,大小善恶,推其天地,详乎太一。又只如甲子年,如甲至子而合,应交司而治天,即下己卯未迁正,而戊寅少阳未退位者,亦甲己未合德也,即土运非太过,而木乃乘虚而胜土也,金次又行复胜之,即反邪化也。阴阳天地殊异尔,故其大小善恶,一如天地之法旨也。

假令丙寅阳年太过,如乙丑天数有余者,虽交(别本作然)得丙寅,太阴尚治天也;地已迁正,厥阴司地,去岁太阳以作右间,即天太阴而地厥阴,故地不奉天化也。乙辛相会,水运太虚,反受土胜,故非太过,即太簇之管,太羽不应,土胜而雨化,水复即风。此者丙辛失守其会,后三年化成水疫,晚至己巳,早至戊辰,甚即速,微即徐,水疫至也。大小善恶,推其天地数,乃太一游宫。又只如丙寅年,丙至寅且合,应交司而治天,即辛巳未得迁正,而庚辰太阳未退位者,亦丙辛不合德也,即水运亦小虚而小胜,或有复,后三年化疠,名曰水疠,其状如水疫,治法如前。

假令庚辰阳年太过,如己卯天数有余者,虽交得庚辰年也,阳明犹尚治天;地以迁正,太阴司地,去岁少阴以作右间,即天阳明而地太阴也,故地下奉天也。乙己相会,金运太虚,反受火胜,故非太过也,即姑洗之管,太商不应,火胜热化,水复寒刑,此乙庚失守,其后三年化成金疫也,速至壬午,徐至癸未,金疫至也。大小善恶,推本年天数及太一也。又只如庚辰,如庚至辰,且应交司而治天,即下乙未未得迁正者,即地甲午少阴未退位者,且乙庚不合德也,即下乙未柔干失刚,亦金运小虚也,有小胜或无复,后三年化疠,名曰金疠,其状如金疫也,治法如前。

假令壬午阳年太过,如辛巳天数有余者,虽交得壬午年也,厥阴犹尚治天;地已迁正,阳明在泉,去岁丙申少阳以作右间,即天厥阴而地阳明,故地不奉天者也。丁辛相合会,木运太虚,反受金胜,故非太过也,即蕤宾之管,太角不应,金行燥胜,火化热复,甚即速,微即徐,疫至大小善恶,推疫至之年天数及太一。又只如壬午,如壬至午,且应交司而治天,即下丁酉未得迁正者,即地下丙申少阳未得退位者,且丁壬不合德也,即丁柔干失刚,亦木运小虚也,有小胜小复,后三年化疠,名曰木疠,其状如风疫也,治法如前。

假令戊申阳年太过,如丁未天数太过者,虽交后戊申年也,太阴犹尚治天;地已迁正,厥阴在泉,去岁壬戌太阳以退位作右间,即天丁未,地癸亥,故地不奉天化也。丁癸相会,火运太虚,反受水胜,故非太过也,即夷则之管,上太徵不应。此戊癸失守其会,后三年化疫也,速至庚戌,大小善恶,推疫至之年天数及太一。又只如戊申,如戊至申,且迁交司而治天,即下癸亥未得迁正者,即地下壬戌太阳未退位者,见戊癸未合德也,即下癸柔干失刚,见火运小虚也,有小胜或无复也,后三年化疠,名曰火疠也,治法如前。治之法,可寒之泄之。

黄帝曰:人气不足,天气如虚,人神失守,神光不聚,邪鬼干人,致有夭亡,可得闻乎?

岐伯曰:人之五藏,一藏不足,又会天虚,感邪之至也。人忧愁思虑即伤心,又或遇少阴司天,天数不及,太阴作接间至,即谓天虚也,此即人气天气同虚也。又遇惊而夺精,汗出于心,因而三虚,神明失守。心为君主之官,神明出焉,神失守位,即神游上丹田,在帝太一帝君泥丸宫下,神既失守,神光不聚,却遇火不及之岁,有黑尸鬼见之,令人暴亡。

人饮食劳倦即伤脾,又或遇太阴司天,天数不及,即少阳作接间至,即谓天虚也,此即人气虚而天气虚也。又遇饮食饱甚,汗出于胃,醉饱行房,汗出于脾,因而三虚,脾神失守。脾为谏议之官,智周出焉,神既失守,神光失位而不聚也,却遇土不及之年,或己年或甲年失守,或太阴天虚;青尸鬼见之,令人卒亡。

人久坐湿地,强力入水即伤肾。肾为作强之官,伎巧出焉,因而三虚,肾神失守,神志失位,神光不聚,却遇水不及之年,或辛不会符,或丙年失守,或太阳司天虚,有黄尸鬼至见之,令人暴亡。

人或恚怒,气逆上而不下,即伤肝也,又遇厥阴司天,天数不及,即少阴作接间至,是谓天虚也,此谓天虚人虚也。又遇疾走恐惧,汗出于肝,肝为将军之官,谋虑出焉,神位失守,神光不聚,又遇木不及年,或丁年不符,或壬年失守,或厥阴司天虚也,有白尸鬼见之,令人暴亡也。

已上五失守者,天虚而人虚也,神游失守其位,即有五尸鬼干人,令人暴亡也,谓之曰尸厥。人犯五神易位,即神光不圆也,非但尸鬼,即一切邪犯者,皆是神失守位故也。此谓:“得守者生,失守者死;得神者昌,失神者亡。”
\section{至真要大论}
%%	74	%%
黄帝问曰:五气交合,盈虚更作,余知之矣。六气分治,司天地者,其至何如?

岐伯再拜对曰:明乎哉问也!天地之大纪,人神之通应也。

帝曰:愿闻上合昭昭,下合冥冥奈何?

岐伯曰:此道之所主,工之所疑也。

帝曰:愿闻其道也。

岐伯曰:厥阴司天,其化以风;少阴司天,其化以热;太阴司天,其化以湿;少阳司天,其化以火;阳明司天,其化以燥;太阳司天,其化以寒。以所临藏位,命其病者也。

帝曰:地化奈何?

岐伯曰:司天同候,间气皆然。

帝曰:间气何谓?

岐伯曰:司左右者,是谓间气也。

帝曰:何以异之?

岐伯曰:主岁者纪岁,间气者纪步也。

帝曰:善。岁主奈何?

岐伯曰:厥阴司天为风化,在泉为酸化,司气为苍化,间气为动化。少阴司天为热化,在泉为苦化,不司气化,居气为灼化。太阴司天为湿化,在泉为甘化,司气为黅化,间气为柔化。少阳司天为火化,在泉苦化,司气为丹化,间气为明化。阳明司天为燥化,在泉为辛化,司气为素化,间气为清化。太阳司天为寒化,在泉为咸化,司气为玄化,间气为藏化。故治病者,必明六化分治,五味五色所生,五藏所宜,乃可以言盈虚病生之绪也。

帝曰:厥阴在泉而酸化先,余知之矣。风化之行也,何如?

岐伯曰:风行于地,所谓本也,余气同法。本乎天者,天之气也,本乎地者,地之气也,天地合气,六节分而万物化生矣。故曰:谨候气宜,无失病机,此之谓也。

帝曰:其主病何如?

岐伯曰:司岁备物,则无遗主矣。

帝曰:先岁物何也?

岐伯曰:天地之专精也。

帝曰:司气者何如?

岐伯曰:司气者主岁同,然有余不足也。

帝曰:非司岁物何谓也?

岐伯曰:散也,故质同而异等也,气味有薄厚,性用有躁静,治保有多少,力化有浅深,此之谓也。

帝曰:岁主藏害何谓?

岐伯曰:以所不胜命之,则其要也。

帝曰:治之奈何?

岐伯曰:上淫于下,所胜平之,外淫于内,所胜治之。

帝曰:善。平气何如?

岐伯曰:谨察阴阳所在而调之,以平为期,正者正治,反者反治。

帝曰:夫子言察阴阳所在而调之,论言人迎与寸口相应,若引绳小大齐等,命曰平,阴之所在寸口何如?

岐伯曰:视岁南北,可知之矣。

帝曰:愿卒闻之。

岐伯曰:北政之岁,少阴在泉,则寸口不应;厥阴在泉,则右不应;太阴在泉,则左不应。南政之岁,少阴司天,则寸口不应;厥阴司天,则右不应;太阴司天,则左不应。诸不应者,反其诊则见矣。

帝曰:尺候何如?

岐伯曰:北政之岁,三阴在下,则寸不应;三阴在上,则尺不应。南政之岁,三阴在天,则寸不应;三阴在泉,则尺不应,左右同。故曰:知其要者,一言而终,不知其要,流散无穷,此之谓也。

帝曰:善。天地之气,内淫而病何如?

岐伯曰:岁厥阴在泉,风淫所胜,则地气不明,平野昧,草乃早秀。民病洒洒振寒,善伸数欠,心痛支满,两胁里急,饮食不下,鬲咽不通,食则呕,腹胀善噫,得后与气,则快然如衰,身体皆重。

岁少阴在泉,热淫所胜,则焰浮川泽,阴处反明。民病腹中常鸣,气上冲胸,喘不能久立,寒热皮肤痛,目瞑齿痛(出页)肿,恶寒发热如疟,少腹中痛,腹大,蛰虫不藏。

岁太阴在泉,草乃早荣,湿淫所胜,则埃昏岩谷,黄反见黑,至阴之交。民病饮积,心痛,耳聋,浑浑焞焞,嗌肿喉痹,阴病血见,少腹痛肿,不得小便,病冲头痛,目似脱,项似拔,腰似折,髀不可以回,膕如结,(月耑)如别。

岁少阳在泉,火淫所胜,则焰明郊野,寒热更至。民病注泄赤白,少腹痛溺赤,甚则血便,少阴同候。

岁阳明在泉,燥淫所胜,则霿雾清瞑。民病喜呕,呕有苦,善太息,心胁痛不能反侧,甚则嗌干面尘,身无膏泽,足外反热。

岁太阳在泉,寒淫所胜,则凝肃惨慄。民病少腹控睾,引腰脊,上冲心痛,血见,嗌痛颔肿。

帝曰:善。治之奈何?

岐伯曰:诸气在泉,风淫于内,治以辛凉,佐以苦,以甘缓之,以辛散之。热淫于内,治以咸寒,佐以甘苦,以酸收之,以苦发之。湿淫于内,治以苦热,佐以酸淡,以苦燥之,以淡泄之。火淫于内,治以咸冷,佐以苦辛,以酸收之,以苦发之。燥淫于内,治以苦温,佐以甘辛,以苦下之。寒淫于内,治以甘热,佐以苦辛,以咸泻之,以辛润之,以苦坚之。

帝曰:善。天气之变何如?

岐伯曰: 厥阴司天,风淫所胜,则太虚埃昏,云物以扰,寒生春气,流水不冰,民病胃脘当心而痛,上支两胁,鬲咽不通,饮食不下,舌本强,食则呕,冷泄腹胀,溏泄,瘕水闭,蛰虫不去,病本于脾。冲阳绝,死不治。 少阴司天,热淫所胜,怫热至,火行其政,民病胸中烦热,嗌干,右胠满,皮肤痛,寒热咳喘,大雨且至,唾血血泄,鼽衄嚏呕,溺色变,甚则疮疡胕肿,肩背臂臑及缺盆中痛,心痛肺(月真),腹大满,膨膨而喘咳,病本于肺。尺泽绝,死不治。

太阴司天,湿淫所胜,则沉阴且布,雨变枯槁,胕肿骨痛,阴痹,阴痹者,按之不得,腰脊头项痛,时眩,大便难,阴气不用,饥不欲食,咳唾则有血,心如悬,病本于肾。太谿绝,死不治。

少阳司天,火淫所胜,则温气流行,金政不平,民病头痛,发热恶寒而疟,热上皮肤痛,色变黄赤,传而为水,身面胕肿,腹满仰息,泄注赤白,疮疡咳唾血,烦心,胸中热,甚则鼽衄,病本于肺。天府绝,死不治。

阳明司天,燥淫所胜,则木乃晚荣,草乃晚生,筋骨内变,民病左胠胁痛,寒清于中,感而疟,大凉革候,咳,腹中鸣,注泄鶩溏,名木敛,生菀于下,草焦上首,心胁暴痛,不可反侧,嗌干面尘,腰痛,丈夫颓疝,妇人少腹痛,目昧眥,疡疮痤痈,蛰虫来见,病本于肝。太冲绝,死不治。

太阳司天,寒淫所胜,则寒气反至,水且冰,血变于中,发为痈疡,民病厥心痛,呕血血泄鼽衄,善悲,时眩仆,运火炎烈,雨暴乃雹,胸腹满,手热肘挛,掖肿,心澹澹大动,胸胁胃脘不安,面赤目黄,善噫嗌干,甚则色(火台),渴而欲饮,病本于心。神门绝,死不治。所谓动气知其藏也。

帝曰:善。治之奈何?

岐伯曰:司天之气,风淫所胜,平以辛凉,佐以苦甘,以甘缓之,以酸泻之。热淫所胜,平以咸寒,佐以苦甘,以酸收之。湿淫所胜,平以苦热,佐以酸辛,以苦燥之,以淡泄之。湿上甚而热,治以苦温,佐以甘辛,以汗为故而止。火淫所胜,平以酸冷,佐以苦甘,以酸收之,以苦发之,以酸复之,热淫同。燥淫所胜,平以苦湿,佐以酸辛,以苦下之。寒淫所胜,平以辛热,佐以甘苦,以咸泻之。

帝曰:善。邪气反胜,治之奈何?

岐伯曰:风司于地,清反胜之,治以酸温,佐以苦甘,以辛平之。热司于地,寒反胜之,治以甘热,佐以苦辛,以咸平之。湿司于地,热反胜之,治以苦冷,佐以咸甘,以苦平之。火司于地,寒反胜之,治以甘热,佐以苦辛,以咸平之。燥司于地,热反胜之,治以平寒,佐以苦甘,以酸平之,以和为利。寒司于地,热反胜之,治以咸冷,佐以甘辛,以苦平之。

帝曰:其司天邪胜何如?

岐伯曰:风化于天,清反胜之,治以酸温,佐以甘苦。热化于天,寒反胜之,治以甘温,佐以苦酸辛。湿化于天,热反胜之,治以苦寒,佐以苦酸。火化于天,寒反胜之,治以甘热,佐以苦辛。燥火于天,热反胜之,治以辛寒,佐以苦甘。寒化于天,热反胜之,治以咸冷,佐以苦辛。

帝曰:六气相胜奈何?

岐伯曰:厥阴之胜,耳鸣头眩,愦愦欲吐,胃鬲如寒,大风数举,倮虫不滋,胠胁气并,化而为热,小便黄赤,胃脘当心而痛,上支两胁,肠鸣飧泄,少腹痛,注下赤白,甚则呕吐,鬲咽不通。

少阴之胜,心下热,善饥,齐下反动,气游三焦,炎暑至,木乃津,草乃萎,呕逆躁烦,腹满痛,溏泄,传为赤沃。

太阴之胜,火气内郁,疮疡于中,流散于外,病在胠胁,甚则心痛,热格,头痛喉痹项强,独胜则湿气内郁,寒迫下焦,痛留顶,互引眉间,胃满,雨数至,燥化乃见,少腹满,腰(月隹)重强,内不便,善注泄,足下温,头重,足胫胕肿,饮发于中,胕肿于上。

少阳之胜,热客于胃,烦心心痛,目赤欲呕,呕酸善饥,耳痛溺赤,善惊谵妄,暴热消烁,草萎水涸,介虫乃屈,少腹痛,下沃赤白。

阳明之胜,清发于中,左胠胁痛,溏泄,内为嗌塞,外发颓疝,大凉肃杀,华英改容,毛虫乃殃,胸中不便,嗌塞而咳。

太阳之胜,凝凓且至,非时水冰,羽乃后化,痔疟发,寒厥入胃,则内生心痛,阴中乃疡,隐曲不利,互引阴股,筋肉拘苛,血脉凝泣,络满色变,或为血泄,皮肤否肿,腹满食减,热反上行,头项囟顶脑户中痛,目如脱,寒入下焦,传为濡泻。

帝曰:治之奈何?

岐伯曰:厥阴之胜,治以甘清,佐以苦辛,以酸泻之。少阴之胜,治以辛寒,佐以苦咸,以甘泻之。太阴之胜,治以咸热,佐以辛甘,以苦泻之。少阳之胜,治以辛寒,佐以甘咸,以甘泻之。阳明之胜,治以酸温,佐以辛甘,以苦泄之。太阳之胜,治以甘热,佐以辛酸,以咸泻之。

帝曰:六气之复何如?

岐伯曰:悉乎哉问也!

厥阴之复,少腹坚满,里急暴痛,偃木飞沙,倮虫不荣,厥心痛,汗发呕吐,饮食不入,入而复出,筋骨掉眩,清厥,甚则入脾,食痹而吐。冲阳绝,死不治。 少阴之复,燠热内作,烦躁鼽嚏,少腹绞痛,火见燔(火芮),嗌燥,分注时止,气动于左,上行于右,咳,皮肤痛,暴瘖心痛,郁冒不知人,乃洒淅恶寒,振慄谵妄,寒已而热,渴而欲饮,少气骨痿,隔肠不便,外为浮肿,哕噫,赤气后化,流水不冰,热气大行,介虫不复,病疿胗疮疡,痈疽痤痔,甚则入肺,咳而鼻渊。天府绝,死不治。

太阴之复,湿变乃举,体重中满,食饮不化,阴气上厥,胸中不便,饮发于中,咳喘有声,大雨时行,鳞见于陆,头顶痛重,而掉瘛尤甚,呕而密默,唾吐清液,甚则入肾窍,泻无度。太谿绝,死不治。

少阳之复,大热将至,枯燥燔(艹热),介虫乃耗,惊瘛咳衄,心热烦躁,便数憎风,厥气上行,面如浮埃,目乃(目閏)瘛,火气内发,上为口麋呕逆,血溢血泄,发而为疟,恶寒鼓慄,寒极反热,嗌络焦槁,渴引水浆,色变黄赤,少气脉萎,化而为水,传为胕肿,甚则入肺,咳而血泄。尺泽绝,死不治。

阳明之复,清气大举,森木苍干,毛虫乃厉,病生胠胁,气归于左,善太息,甚则心痛否满,腹胀而泄,呕苦咳哕,烦心,病在鬲中,头痛,甚则入肝,惊骇筋挛。太冲绝,死不治。

太阳之复,厥气上行,水凝雨冰,羽虫乃死。心胃生寒,胸膈不利,心痛否满,头痛善悲,时眩仆,食减,腰(月隹)反痛,屈伸不便,地裂冰坚,阳光不治,少腹控睾,引腰脊,上冲心,唾出清水,及为哕噫,甚则入心,善忘善悲。神门绝,死不治。

帝曰:善,治之奈何?

岐伯曰:厥阴之复,治以酸寒,佐以甘辛,以酸泻之,以甘缓之。少阴之复,治以咸寒,佐以苦辛,以甘泻之,以酸收之,辛苦发之,以咸軟之。太阴之复,治以苦热,佐以酸辛,以苦泻之,燥之,泄之。少阳之复,治以咸冷,佐以苦辛,以咸軟之,以酸收之,辛苦发之,发不远热,无犯温凉,少阴同法。阳明之复,治以辛温,佐以苦甘,以苦泄之,以苦下之,以酸补之。太阳之复,治以咸热,佐以甘辛,以苦坚之。治诸胜复,寒者热之,热者寒之,温者清之,清者温之,散者收之,抑者散之,燥者润之,急者缓之,坚者耎之,脆者坚之,衰者补之,强者泻之,各安其气,必清必静,则病气衰去,归其所宗,此治之大体也。

帝曰:善。气之上下,何谓也?

岐伯曰:身半以上,其气三矣,天之分也,天气主之。身半以下,其气三矣,地之分也,地气主之。以名命气,以气命处,而言其病。半,所谓天枢也。故上胜而下俱病者,以地名之,下胜而上俱病者,以天名之。所谓胜至,报气屈伏而未发也,复至则不以天地异名,皆如复气为法也。

帝曰:胜复之动,时有常乎?气有必乎?

岐伯曰:时有常位,而气无必也。

帝曰:愿闻其道也。

岐伯曰:初气终三气,天气主之,胜之常也。四气尽终气,地气主之,复之常也。有胜则复,无胜则否。

帝曰:善。复已而胜何如?

岐伯曰:胜至则复,无常数也,衰乃止耳。复已而胜,不复则害,此伤生也。

帝曰:复而反病何也?

岐伯曰:居非其位,不相得也,大复其胜则主胜之,故反病也,所谓火燥热也。

帝曰:治之何如?

岐伯曰:夫气之胜也,微者随之,甚者制之。气之复也,和者平之,暴者夺之,皆随胜气,安其屈伏,无问其数,以平为期,此其道也。

帝曰:善。客主之胜复奈何?

岐伯曰:客主之气,胜而无复也。

帝曰:其逆从何如?

岐伯曰:主胜逆,客胜从,天之道也。

帝曰:其生病何如?

岐伯曰:厥阴司天,客胜则耳鸣掉眩,甚则咳;主胜则胸胁痛,舌难以言。少阴司天,客胜则鼽嚏颈项强,肩背瞀热,头痛少气,发热耳聋目暝,甚则胕肿血溢,疮疡咳喘;主胜则心热烦躁,甚则胁痛支满。太阴司天,客胜则首面胕肿,呼吸气喘;主胜则胸腹满,食已而瞀。少阳司天,客胜则丹胗外发,及为丹熛疮疡,呕逆喉痹,头痛嗌肿,耳聋血溢,内为瘛瘲;主胜则胸满咳仰息,甚而有血,手热。阳明司天,清复内余,则咳衄嗌塞,心鬲中热,咳不止而白血出者死。太阳司天,客胜则胸中不利,出清涕,感寒则咳;主胜则喉嗌中鸣。 厥阴在泉,客胜则大关节不利,内为痉强拘瘛,外为不便;主胜则筋骨繇并,腰腹时痛。少阴在泉,客胜则腰痛,尻股膝髀(月耑)(骨行)足病,瞀热以酸,胕肿不能久立,溲便变;主胜则厥气上行,心痛发热,鬲中,众痹皆作,发于胠胁,魄汗不藏,四逆而起。太阴在泉,客胜则足痿下重,便溲不时,湿客下焦,发而濡泻,及为肿,隐曲之疾;主胜则寒气逆满,食饮不下,甚则为疝。少阳在泉,客胜则腰腹痛而反恶寒,甚则下白溺白;主胜则热反上行而客于心,心痛发热,格中而呕。少阴同候。阳明在泉,客胜则清气动下,少腹坚满而数便泻;主胜则腰重腹痛,少腹生寒,下为鶩溏,则寒厥于肠,上冲胸中,甚则喘,不能久立。太阳在泉,寒复内余,则腰尻痛,屈伸不利,股胫足膝中痛。

帝曰:善,治之奈何?

岐伯曰:高者抑之,下者举之,有余折之,不足补之,佐以所利,和以所宜,必安其主客,适其寒温,同者逆之,异者从之。

帝曰:治寒以热,治热以寒,气相得者逆之,不相得者从之,余己知之矣。其于正味何如?

岐伯曰:木位之主,其泻以酸,其补以辛。火位之主,其泻以甘,其补以咸。土位之主,其泻以苦,其补以甘。金位之主,其泻以辛,其补以酸。水位之主,其泻以咸,其补以苦。厥阴之客,以辛补之,以酸泻之,以甘缓之。少阴之客,以咸补之,以甘泻之,以咸收之。太阴之客,以甘补之,以苦泻之,以甘缓之。少阳之客,以咸补之,以甘泻之,以咸軟之。阳明之客,以酸补之。以辛泻之,以苦泄之。太阳之客,以苦补之,以咸泻之,以苦坚之,以辛润之。开发腠理,致津液通气也。

帝曰:善。愿闻阴阳之三也何谓?

岐伯曰:气有多少,异用也。

帝曰:阳明何谓也?

岐伯曰:两阳合明也。

帝曰:厥阴何也?

岐伯曰:两阴交尽也。

帝曰:气有多少,病有盛衰,治有缓急,方有大小,愿闻约奈何?

岐伯曰:气有高下,病有远近,证有中外,治有轻重,适其至所为故也。《大要》曰:君一臣二,奇之制也;君二臣四,偶之制也;君二臣三,奇之制也;君二臣六,偶之制也。故曰:近者奇之,远者偶之,汗者不以奇,下者不以偶,补上治上制以缓,补下治下制以急,急则气味厚,缓则气味薄,适其至所,此之谓也。病所远而中道气味之者,食而过之,无越其制度也。是故平气之道,近而奇偶,制小其服也。远而奇偶,制大其服也。大则数少,小则数多。多则九之,少则二之。奇之不去则偶之,是谓重方。偶之不去,则反佐以取之,所谓寒热温凉,反从其病也。

帝曰:善。病生于本,余知之矣。生于标者,治之奈何?

岐伯曰:病反其本,得标之病,治反其本,得标之方。

帝曰:善。六气之胜,何以候之?

岐伯曰:乘其至也。清气大来,燥之胜也,风木受邪,肝病生焉。热气大来,火之胜也,金燥受邪,肺病生焉。寒气大来,水之胜也,火热受邪,心病生焉。湿气大来,土之胜也,寒水受邪,肾病生焉。风气大来,木之胜也,土湿受邪,脾病生焉。所谓感邪而生病也。乘年之虚,则邪甚也。失时之和,亦邪甚也。遇月之空,亦邪甚也。重感于邪,则病危矣。有胜之气,其必来复也。

帝曰:其脉至何如?

岐伯曰:厥阴之至,其脉弦,少阴之至,其脉钩,太阴之至,其脉沉,少阳之至,大而浮,阳明之至,短而濇,太阳之至,大而长。至而和则平,至而甚则病,至而反者病,至而不至者病,未至而至者病,阴阳易者危。

帝曰:六气标本,所从不同,奈何?

岐伯曰:气有从本者,有从标本者,有不从标本者也。

帝曰:愿卒闻之。

岐伯曰:少阳太阴从本,少阴太阳从本从标,阳明厥阴,不从标本,从乎中也。故从本者,化生于本,从标本者,有标本之化,从中者,以中气为化也。

帝曰:脉从而病反者,其诊何如?

岐伯曰:脉至而从,按之不鼓,诸阳皆然。

帝曰:诸阴之反,其脉何如?

岐伯曰:脉至而从,按之鼓甚而盛也。 是故百病之起,有生于本者,有生于标者,有生于中气者,有取本而得者,有取标而得者,有取中气而得者,有取标本而得者,有逆取而得者,有从取而得者。逆,正顺也。若顺,逆也。故曰:知标与本,用之不殆,明知逆顺,正行无问。此之谓也。不知是者,不足以言诊,足以乱经。故《大要》曰:粗工嘻嘻,以为可知,言热未已,寒病复始,同气异形,迷诊乱经,此之谓也,夫标本之道,要而博,小而大,可以言一而知百病之害,言标与本,易而勿损,察本与标,气可令调,明知胜复,为万民式,天之道毕矣。

帝曰:胜复之变,早晏何如?

岐伯曰:夫所胜者,胜至已病,病已愠愠,而复已萌也。夫所复者,胜尽而起,得位而甚,胜有微甚,复有少多,胜和而和,胜虚而虚,天之常也。

帝曰:胜复之作,动不当位,或后时而至,其故何也?

岐伯曰:夫气之生,与其化衰盛异也。寒暑温凉盛衰之用,其在四维。故阳之动,始于温,盛于暑;阴之动,始于清,盛于寒。春夏秋冬,各差其分。故《大要》曰:彼春之暖,为夏之暑,彼秋之忿,为冬之怒,谨按四维,斥候皆归,其终可见,其始可知。此之谓也。

帝曰:差有数乎?

岐伯曰:又凡三十度也。

帝曰:其脉应皆何如?

岐伯曰:差同正法,待时而去也。《脉要》曰:春不沉,夏不弦,冬不濇,秋不数,是谓四塞。沉甚曰病,弦甚曰病,涩甚曰病,数甚曰病,参见曰病,复见曰病,未去而去曰病,去而不去曰病,反者死。故曰:气之相守司也,如权衡之不得相失也。夫阴阳之气,清静则生化治,动则苛疾起,此之谓也。

帝曰:幽明何如?

岐伯曰:两阴交尽故曰幽,两阳合明故曰明,幽明之配,寒暑之异也。

帝曰:分至何如?

岐伯曰:气至之谓至,气分之谓分,至则气同,分则气异,所谓天地之正纪也。

帝曰:夫子言春秋气始于前,冬夏气始于后,余已知之矣。然六气往复,主岁不常也,其补泻奈何?

岐伯曰:上下所主,随其攸利,正其味,则其要也,左右同法。《大要》曰:少阳之主,先甘后咸;阳明之主,先辛后酸;太阳之主,先咸后苦;厥阴之主,先酸后辛;少阴之主,先甘后咸;太阴之主,先苦后甘。佐以所利,资以所生,是谓得气。

帝曰:善。夫百病之生也,皆生于风寒暑湿燥火,以之化之变也。经言盛者泻之,虚者补之,余錫以方士,而方士用之,尚未能十全,余欲令要道必行,桴鼓相应,犹拔刺雪汙,工巧神圣,可得闻乎?

岐伯曰:审察病机,无失气宜,此之谓也。

帝曰:愿闻病机何如?

岐伯曰:诸风掉眩,皆属于肝。诸寒收引,皆属于肾。诸气膹郁,皆属于肺。诸湿肿满,皆属于脾。诸热瞀瘈,皆属于火。诸痛痒疮,皆属于心。诸厥固泄,皆属于下。诸痿喘呕,皆属于上。诸禁鼓慄,如丧神守,皆属于火。诸痉项强,皆属于湿。诸逆冲上,皆属于火。诸胀腹大,皆属于热。诸躁狂越,皆属于火。诸暴强直,皆属于风。诸病有声,鼓之如鼓,皆属于热。诸病胕肿,疼酸惊骇,皆属于火。诸转反戾,水液浑浊,皆属于热。诸病水液,澄澈清冷,皆属于寒。诸呕吐酸,暴注下迫,皆属于热。故《大要》曰:谨守病机,各司其属,有者求之,无者求之,盛者责之,虚者责之,必先五胜,疏其血气,令其调达,而致和平,此之谓也。

帝曰:善,五味阴阳之用何如?

岐伯曰:辛甘发散为阳,酸苦涌泄为阴,咸味涌泄为阴,淡味渗泄为阳。六者或收或散,或缓或急,或燥或润,或軟或坚,以所利而行之,调其气,使其平也。

帝曰:非调气而得者,治之奈何?有毒无毒,何先何后?愿闻其道。

岐伯曰:有毒无毒,所治为主,适大小为制也。

帝曰:请言其制。

岐伯曰:君一臣二,制之小也;君一臣三佐五,制之中也;君一臣三佐九,制之大也。寒者热之,热者寒之,微者逆之,甚者从之,坚者削之,客者除之,劳者温之,结者散之,留者攻之,燥者濡之,急者缓之,散者收之,损者温之,逸者行之,惊者平之,上之下之,摩之浴之,薄之劫之,开之发之,适事为故。

帝曰:何谓逆从?

岐伯曰:逆者正治,从者反治,从少从多,观其事也。

帝曰:反治何谓?

岐伯曰:热因寒用,寒因热用,塞因塞用,通因通用,必伏其所主,而先其所因,其始则同,其终则异,可使破积,可使溃坚,可使气和,可使必已。

帝曰:善。气调而得者何如?

岐伯曰:逆之从之,逆而从之,从而逆之,疏气令调,则其道也。

帝曰:善。病之中外何如?

岐伯曰:从内之外者调其内;从外之内者治其外;从内之外而盛于外者,先调其内而后治其外;从外之内而盛于内者,先治其外,而后调其内;中外不相及,则治主病。

帝曰:善。火热复,恶寒发热,有如疟状,或一日发,或间数日发,其故何也?

岐伯曰:胜复之气,会遇之时,有多少也。阴气多而阳气少,则其发日远;阳气多而阴气少,则其发日近。此胜复相薄,盛衰之节,疟亦同法。

帝曰:论言治寒以热,治热以寒,而方士不能废绳墨而更其道也。有病热者,寒之而热,有病寒者,热之而寒,二者皆在,新病复起,奈何治?

岐伯曰:诸寒之而热者取之阴,热之而寒者取之阳,所谓求其属也。

帝曰:善。服寒而反热,服热而反寒,其故何也?

岐伯曰:治其王气,是以反也。

帝曰:不治王而然者何也?

岐伯曰:悉乎哉问也!不治五味属也。夫五味入胃,各归所喜,攻酸先入肝,苦先入心,甘先入脾,辛先入肺,咸先入肾,久而增气,物化之常也。气增而久,夭之由也。

帝曰:善。方制君臣何谓也?

岐伯曰:主病之谓君,佐君之谓臣,应臣之谓使,非上下三品之谓也。

帝曰:三品何谓/

岐伯曰:所以明善恶之殊贯也。

帝曰:善。病之中外何如?

岐伯曰:调气之方,必别阴阳,定其中外,各守其乡。内者内治,外者外治,微者调之,其次平之,盛者夺之,汗之下之,寒热温凉,衰之以属,随其攸利,谨道如法,万举万全,气血正平,长有天命。

帝曰:善。
\section{著至教论}
%%	75	%%
黄帝坐明堂,召雷公而问之曰:子知医之道乎?

雷公对曰:诵而颇能解,解而未能别,别而未能明,明而未能彰,足以治群僚,不足至侯王。愿得受树天之度,四时阴阳合之,别星辰与日月光,以彰经术,后世益明,上通神农,著至教,疑于二皇。

帝曰:善!无失之,此皆阴阳表里上下雌雄相輸应也,而道上知天文,下知地理,中知人事,可以长久,以教众庶,亦不疑殆,医道论篇,可传后世,可以为宝。

雷公曰:请受道,讽诵用解。

帝曰:子不闻《阴阳传》乎!

曰:不知。

曰:夫三阳天为业,上下无常,合而病至,偏害阴阳。

雷公曰:三阳莫当,请闻其解。

帝曰:三阳独至者,是三阳并至,并至如风雨,上为巅疾,下为漏病,外无期,内无正,不中经纪,诊无上下,以书别。

雷公曰:臣治疏愈,说意而已。

帝曰:三阳者,至阳也,积并则为惊,病起疾风,至如礔砺,九窍皆塞,阳气滂溢,干嗌喉塞,并于阴,则上下无常,薄为肠澼,此谓三阳直心,坐不得起,卧者便身全。三阳之病,且以知天下,何以别阴阳,应四时,合之五行。

雷公曰:阳言不别,阴言不理,请起受解,以为至道。

帝曰:子若受传,不知合至道以惑师教,语子至道之要。病伤五藏,筋骨以消,子言不明不别,是世主学尽矣。肾且绝,惋惋日暮,从容不出,人事不殷。
\section{示从容论}
%%	76	%%
黄帝燕坐,召雷公而问之曰:汝受术诵书者,若能览观杂学,及于比类,通合道理,为余言子所长,五脏六腑,胆、胃、大小肠、脾、胞、膀胱、脑髓,涕、唾、哭泣、悲哀,水所从行,此皆人之所生,治之过失,子务明之,可以十全,即不能知,为世所怨。

雷公曰:臣请诵《脉经·上下篇》甚众多矣,别异比类,犹未能以十全,又安足以明之?

帝曰:子别试通五脏之过,六腑之所不和,针石之败,毒药所宜,汤液滋味,具言其状,悉言以对,请问不知。

雷公曰:肝虚、肾虚、脾虚,皆令人体重烦冤,当投毒药、刺灸、砭石、汤液,或已或不已,愿闻其解。

帝曰:公何年之长而问之少?余真问以自谬也。吾问子窈冥,子言上下篇以对,何也?夫脾虚浮似肺,肾小浮似脾,肝急沉散似肾,此皆工之所时乱也,然从容得之。若夫三脏,土木水参居,此童子之所知,问之何也!

雷公曰:于此有人头痛,筋挛,骨重,怯然少气,哕,噫,腹满,时惊,不嗜卧,此何脏之发也?脉浮而弦,切之石坚,不知其解,复问所以三脏者,以知其比类也。

帝曰:夫从容之谓也。夫年长则求之于腑,年少则求之于经,年壮则求之于脏。今子所言皆失,八风菀热,五脏消烁,传邪相受。夫浮而弦者,是肾不足也;沉而石者,是肾气内着也;怯然少气者,是水道不行,形气消索也;咳嗽烦冤者,是肾气之逆也。一人之气,病在一脏也,若言三脏俱行,不在法也。

雷公曰:于此有人四肢懈堕,喘嗽,血泄,而愚诊之以为伤肺,切脉浮大而紧,愚不敢治。粗工下砭石,病愈多出血,血止身轻,此何物也?

帝曰:子所能治,知亦众多,与此病失矣。譬以鸿飞,亦冲于天。夫圣人之治病,循法守度,援物比类,化之冥冥,循上及下,何必守经?今夫脉浮大虚者,是脾气之外绝,去胃外归阳明也。夫二火不胜三水,是以脉乱而无常也。四肢懈堕,此脾精之不行也。喘咳者,是水气并阳明也。血泄者,脉急,血无所行也。若夫以为伤肺者,由失以狂也。不引比类,是知不明也。夫伤肺者,脾气不守,胃气不清,经气不为使,真脏坏决,经脉旁绝,五脏漏泄,不衂则呕。此二者不相类也,譬如天之无形,地之无理,白与黑相去远矣。是失吾过矣,以子知之,故不告子,明引比类《从容》,是以名曰诊轻(《太素》作经),是谓至道也。
\section{疏五过论}
%%	77	%%
黄帝曰:呜呼远哉!闵闵乎若视深渊,若迎浮云,视深渊尚可测,迎浮云莫知其际。圣人之术,为万民式,论裁志意,必有法则,循经守数,接循医事,为万民副,故事有五过四德,汝知之乎?

雷公避席再拜曰:臣年幼小,蒙愚以惑,不闻五过与四德,比类形名,虚引其经,心无所对。

帝曰:凡未诊病者,必问尝贵后贱,虽不中邪,病从内生,名曰脱营。尝富后贫,名曰失精,五气留连,病有所并。医工诊之,不在藏府,不变躯形,诊之而疑,不知病名。身体日减,气虚无精,病深无气,洒洒然时惊,病深者,以其外耗于卫,内夺于荣。良工所失,不知病情,此亦治之一过也。

凡欲诊病者,必问饮食居处,暴乐暴苦,始乐后苦,皆伤精气,精气竭绝,形体毁沮。暴怒伤阴,暴喜伤阳,厥气上行,满脉去形。愚医治之,不知补泻,不知病情,精华日脱,邪气乃并,此治之二过也。

善为脉者,必以比类奇恒,从容知之,为工而不知道,此诊之不足贵,此治之三过也。

诊有三常,必问贵贱,封君败伤,及欲侯王。故贵脱势,虽不中邪,精神内伤,身必败亡。始富后贫,虽不伤邪,皮焦筋屈,痿躄为挛。医不能严,不能动神,外为柔弱,乱至失常,病不能移,则医事不行,此治之四过也。

凡诊者必知终始,有知余绪,切脉问名,当合男女。离绝菀结,忧恐喜怒,五藏空虚,血气离守,工不能知,何术之语。尝富大伤,斩筋绝脉,身体复行,令泽不息。故伤败结积,留薄归阳,脓积寒炅。粗工治之,亟刺阴阳,身体解散,四支转筋,死日有期,医不能明,不问所发,唯言死日,亦为粗工,此治之五过也。

凡此五者,皆受术不通,人事不明也。故曰:圣人之治病也,必知天地阴阳,四时经纪,五藏六府,雌雄表里,刺灸砭石,毒药所主,从容人事,以明经道,贵贱贫富,各异品理,问年少长,勇怯之理,审于分部,知病本始,八正九候,诊必副矣。

治病之道,气内为宝,循求其理,求之不得,过在表里。守数据治,无失俞理,能行此术,终身不殆。不知俞理,五藏菀熟,痈发六府,诊病不审,是谓失常。谨守此治,与经相明,《上经》《下经》,揆度阴阳,奇恒五中,决以明堂,审于终始,可以横行。
\section{徵四失论}
%%	78	%%
黄帝在明堂,雷公侍坐,黄帝曰:夫子所通书受事众多矣,试言得失之意,所以得之,所以失之。

雷公对曰:循经受业,皆言十全,其时有过失者,请闻其事解也。

帝曰:子年少,智未及邪,将言以杂合耶?夫经脉十二,络脉三百六十五,此皆人之所明知,工之所循用也。所以不十全者,精神不专,志意不理,外内相失,故时疑殆。诊不知阴阳逆从之理,此治之一失矣。

受师不卒,妄作杂术,谬言为道,更名自功,妄用砭石,后遗身咎,此治之二失也。

不适贫富贵贱之居,坐之薄厚,形之寒温,不适饮食之宜,不别人之勇怯,不知比类,足以自乱,不足以自明,此治之三失也。

诊病不问其始,忧患饮食之失节,起居之过度,或伤于毒,不先言此,卒持寸口,何病能中,妄言作名,为所穷,此治之四失也。

是以世人之语者,驰千里之外,不明尺寸之论,诊无人事。治数之道,从容之葆,坐持寸口,诊不中五脉,百病所起,始以自怨,遗师其咎。是故治不能循理,弃术于市,妄治时愈,愚心自得。呜呼!窈窈冥冥,熟知其道?道之大者,拟于天地,配于四海,汝不知道之谕,受以明为晦。
\section{阴阳类论}
%%	79	%%
孟春始至,黄帝燕坐,临观八极,正八风之气,而问雷公曰:阴阳之类,经脉之道,五中所主,何藏最贵?

雷公对曰:春甲乙青,中主肝,治七十二日,是脉之主时,臣以其藏最贵。

帝曰:却念上下经,阴阳从容,子所言最贵,其下也。

雷公致斋七日,旦复侍坐。

帝曰:三阳为经,二阳为维,一阳为游部,此知五藏终始。三阴为表,二阴为里,一阴至绝,作朔晦,却具合以正其理。

雷公曰:受业未能明。

帝曰:所谓三阳者,太阳为经,三阳脉,至手太阴,弦浮而不沉,决以度,察以心,合之阴阳之论。所谓二阳者,阳明也,至手太阴,弦而沉急不鼓,炅至以病皆死。一阳者,少阳也,至手太阴,上连人迎,弦急悬不绝,此少阳之病也,专阴则死。

三阴者,六经之所主也,交于太阴,伏鼓不浮,上空志心。二阴至肺,其气归膀胱,外连脾胃。一阴独至,经绝,气浮不鼓,钩而滑。此六脉者,乍阴乍阳,交属相并,缪通五藏,合于阴阳,先至为主,后至为客。

雷公曰:臣悉尽意,受传经脉,颂得从容之道,以合《从容》,不知阴阳,不知雌雄。

帝曰:三阳为父,二阳为卫,一阳为纪。三阴为母,二阴为雌,一阴为独使。

二阳一阴,阳明主病,不胜一阴,软而动,九窍皆沉。三阳一阴,太阳脉胜,一阴不能止,内乱五藏,外为惊骇。二阴二阳,病在肺,少阴脉沉,胜肺伤脾,外伤四支。二阴二阳皆交至,病在肾,骂詈妄行,巅疾为狂。二阴一阳,病出于肾,阴气客游于心脘,下空窍堤,闭塞不通,四支别离。一阴一阳代绝,此阴气至心,上下无常,出入不知,喉咽干燥,病在土脾。二阳三阴,至阴皆在,阴不过阳,阳气不能止阴,阴阳并绝,浮为血瘕,沉为脓胕。阴阳皆壮,下至阴阳。上合昭昭,下合冥冥,诊决生死之期,遂合岁首。

雷公曰:请问短期。黄帝不应。雷公复问。

黄帝曰:在经论中。

雷公曰:请闻短期。

黄帝曰:冬三月之病,病合于阳者,至春正月脉有死徵,皆归出春。冬三月之病,在理已尽,草与柳叶皆杀,春阴阳皆绝,期在孟春。春三月之病,曰阳杀,阴阳皆绝,期在草干。夏三月之病,至阴不过十日,阴阳交,期在溓水。秋三月之病,三阳俱起,不治自已。阴阳交合者,立不能坐,坐不能起。三阳独至,期在石水。二阴独至,期在盛水。
\section{方盛衰论}
%%	80	%%
雷公请问:气之多少,何者为逆,何者为从。

黄帝答曰:阳从左,阴从右,老从上,少从下。是以春夏归阳为生,归秋冬为死,反之则归秋冬为生,是以气多少,逆皆为厥。

问曰:有余者厥耶?

答曰:一上不下,寒厥到膝,少者秋冬死,老者秋冬生。气上不下,头痛巅疾,求阳不得,求阴不审,五部隔无徵,若居旷野,若伏空室,绵绵乎属不满日。 是以少气之厥,令人妄梦,其极至迷。三阳绝,三阴微,是为少气。

是以肺气虚,则使人梦见白物,见人斩血藉藉,得其时,则梦见兵战。肾气虚,则使人梦见舟舩溺人,得其时,则梦伏水中,若有畏恐。肝气虚,则梦见菌香生草,得其时,则梦伏树下不敢起。心气虚,则梦救火阳物,得其时,则梦燔灼。脾气虚,则梦饮食不足,得其时,则梦筑垣盖屋。此皆五藏气虚,阳气有余,阴气不足。合之五诊,调之阴阳,以在经脉。

诊有十度度人:脉度,藏度,肉度,筋度,俞度。阴阳气尽,人病自具。脉动无常,散阴颇阳,脉脱不具,诊无常行,诊必上下,度民君卿。受师不卒,使术不明,不察逆从,是为妄行,持雌失雄,弃阴附阳,不知并合,诊故不明,传之后世,反论自章。

至阴虚,天气绝;至阳盛,地气不足。阴阳并交,至人之所行。阴阳交并者,阳气先至,阴气后至。是以圣人持诊之道,先后阴阳而持之,《奇恒之势》乃六十首,诊合微之事,追阴阳之变,章五中之情,其中之论,取虚实之要,定五度之事,知此乃足以诊。是以切阴不得阳,诊消亡,得阳不得阴,守学不湛,知左不知右,知右不知左,知上不知下,知先不知后,故治不久。知醜知善,知病知不病,知高知下,知坐知起,知行知止,用之有纪,诊道乃具,万世不殆。起所有余,知所不足。度事上下,脉事因格。是以形弱气虚,死;形气有余,脉气不足,死。脉气有余,形气不足,生。 是以诊有大方,坐起有常,出入有行,以转神明,必清必净,上观下观,司八正邪,别五中部,按脉动静,循尺滑濇,寒温之意,视其大小,合之病能,逆从以得,复知病名,诊可十全,不失人情。故诊之,或视息视意,故不失条理,道甚明察,故能长久。不知此道,失经绝理,亡言妄期,此谓失道。
\section{解精微论}
%%	81	%%
黄帝在明堂,雷公请曰:臣授业,传之行教以经论,从容形法,阴阳刺灸,汤药所滋,行治有贤不肖,未必能十全。若先言悲哀喜怒,燥湿寒暑,阴阳妇女,请问其所以然者,卑贱富贵,人之形体,所从群下,通使临事以适道术,谨闻命矣。请问有毚愚仆漏之问,不在经者,欲闻其状。

帝曰:大矣。

公请问:哭泣而泪不出者,若出而少涕,其故何也?

帝曰:在经有也。

复问:不知水所从生,涕所出也。

帝曰:若问此者,无益于治也,工之所知,道之所生也。

夫心者,五藏之专精也,目者,其窍也,华色者,其荣也,是以人有德也,则气和于目,有亡,忧知于色。是以悲哀则泣下,泣下水所由生。水宗者,积水也,积水者,至阴也,至阴者,肾之精也。宗精之水所以不出者,是精持之也。辅者裹之,故水不行也。夫水之精为志,火之精为神,水火相感,神志俱悲,是以目之水生也。故谚言曰:心悲名曰志悲,志与心精共凑于目也。是以俱悲则神气传于心,精上不传于志,而志独悲,故泣出也。泣涕者,脑也,脑者,阴也,髓者,骨之充也,故脑渗为涕。志者骨之主也,是以水流而涕从之者,其行类也。夫涕之与泣者,譬如人之兄弟,急则俱死,生则俱生,其志以神悲,是以涕泣俱出而横行也。夫人涕泣俱出而相从者,所属之类也。

雷公曰:大矣。请问人哭泣而泪不出者,若出而少,涕不从之何也?

帝曰:夫泣不出者,哭不悲也。不泣者,神不慈也。神不慈则志不悲,阴阳相持,泣安能独来。夫志悲者惋,惋则冲阴,冲阴则志去目,志去则神不守精,精神去目,涕泣出也。 且子独不诵不念夫经言乎,厥则目无所见。夫人厥则阳气并于上,阴气并于下。阳并于上,则火独光也;阴并于下则足寒,足寒则胀也。夫一水不胜五火,故目盲。是以冲风,泣下而不止。夫风之中目也,阳气内守于精,是火气燔目,故见风则泣下也。有以比之,夫火疾风生乃能雨,此之类也。


%%%%%%%%%%%%%%% 颜色  %%%%%%%%%%%%%%%%%%%%%%%%%%%%%%%%
% yellow: 黄色
% gray: 灰色
% brown: 棕色
% orange: 橙色
% pink: 粉红
% purple: 紫色
%%%%%%%%%%%%%%%%%%%%%%%%%%%%%%
% cyan: 青色
% magenta: 洋红
% lime: 石灰绿
% olive: 橄榄绿
% teal: 蓝绿色
% violet: 紫罗兰色
%\textcolor{blue}{P(x, y)}
%\textcolor{magenta}{P(x, y)}
%\textcolor{lime}{P(x, y)}
%\textcolor{olive}{P(x, y)}
%\textcolor{teal}{P(x, y)}
%\textcolor{violet}{P(x, y)}

\end{document}